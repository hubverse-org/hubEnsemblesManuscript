% Options for packages loaded elsewhere
\PassOptionsToPackage{unicode}{hyperref}
\PassOptionsToPackage{hyphens}{url}
\PassOptionsToPackage{dvipsnames,svgnames,x11names}{xcolor}
%
\documentclass[
  letterpaper,
  DIV=11,
  numbers=noendperiod]{scrartcl}

\usepackage{amsmath,amssymb}
\usepackage{iftex}
\ifPDFTeX
  \usepackage[T1]{fontenc}
  \usepackage[utf8]{inputenc}
  \usepackage{textcomp} % provide euro and other symbols
\else % if luatex or xetex
  \usepackage{unicode-math}
  \defaultfontfeatures{Scale=MatchLowercase}
  \defaultfontfeatures[\rmfamily]{Ligatures=TeX,Scale=1}
\fi
\usepackage{lmodern}
\ifPDFTeX\else  
    % xetex/luatex font selection
\fi
% Use upquote if available, for straight quotes in verbatim environments
\IfFileExists{upquote.sty}{\usepackage{upquote}}{}
\IfFileExists{microtype.sty}{% use microtype if available
  \usepackage[]{microtype}
  \UseMicrotypeSet[protrusion]{basicmath} % disable protrusion for tt fonts
}{}
\makeatletter
\@ifundefined{KOMAClassName}{% if non-KOMA class
  \IfFileExists{parskip.sty}{%
    \usepackage{parskip}
  }{% else
    \setlength{\parindent}{0pt}
    \setlength{\parskip}{6pt plus 2pt minus 1pt}}
}{% if KOMA class
  \KOMAoptions{parskip=half}}
\makeatother
\usepackage{xcolor}
\setlength{\emergencystretch}{3em} % prevent overfull lines
\setcounter{secnumdepth}{5}
% Make \paragraph and \subparagraph free-standing
\ifx\paragraph\undefined\else
  \let\oldparagraph\paragraph
  \renewcommand{\paragraph}[1]{\oldparagraph{#1}\mbox{}}
\fi
\ifx\subparagraph\undefined\else
  \let\oldsubparagraph\subparagraph
  \renewcommand{\subparagraph}[1]{\oldsubparagraph{#1}\mbox{}}
\fi

\usepackage{color}
\usepackage{fancyvrb}
\newcommand{\VerbBar}{|}
\newcommand{\VERB}{\Verb[commandchars=\\\{\}]}
\DefineVerbatimEnvironment{Highlighting}{Verbatim}{commandchars=\\\{\}}
% Add ',fontsize=\small' for more characters per line
\usepackage{framed}
\definecolor{shadecolor}{RGB}{241,243,245}
\newenvironment{Shaded}{\begin{snugshade}}{\end{snugshade}}
\newcommand{\AlertTok}[1]{\textcolor[rgb]{0.68,0.00,0.00}{#1}}
\newcommand{\AnnotationTok}[1]{\textcolor[rgb]{0.37,0.37,0.37}{#1}}
\newcommand{\AttributeTok}[1]{\textcolor[rgb]{0.40,0.45,0.13}{#1}}
\newcommand{\BaseNTok}[1]{\textcolor[rgb]{0.68,0.00,0.00}{#1}}
\newcommand{\BuiltInTok}[1]{\textcolor[rgb]{0.00,0.23,0.31}{#1}}
\newcommand{\CharTok}[1]{\textcolor[rgb]{0.13,0.47,0.30}{#1}}
\newcommand{\CommentTok}[1]{\textcolor[rgb]{0.37,0.37,0.37}{#1}}
\newcommand{\CommentVarTok}[1]{\textcolor[rgb]{0.37,0.37,0.37}{\textit{#1}}}
\newcommand{\ConstantTok}[1]{\textcolor[rgb]{0.56,0.35,0.01}{#1}}
\newcommand{\ControlFlowTok}[1]{\textcolor[rgb]{0.00,0.23,0.31}{#1}}
\newcommand{\DataTypeTok}[1]{\textcolor[rgb]{0.68,0.00,0.00}{#1}}
\newcommand{\DecValTok}[1]{\textcolor[rgb]{0.68,0.00,0.00}{#1}}
\newcommand{\DocumentationTok}[1]{\textcolor[rgb]{0.37,0.37,0.37}{\textit{#1}}}
\newcommand{\ErrorTok}[1]{\textcolor[rgb]{0.68,0.00,0.00}{#1}}
\newcommand{\ExtensionTok}[1]{\textcolor[rgb]{0.00,0.23,0.31}{#1}}
\newcommand{\FloatTok}[1]{\textcolor[rgb]{0.68,0.00,0.00}{#1}}
\newcommand{\FunctionTok}[1]{\textcolor[rgb]{0.28,0.35,0.67}{#1}}
\newcommand{\ImportTok}[1]{\textcolor[rgb]{0.00,0.46,0.62}{#1}}
\newcommand{\InformationTok}[1]{\textcolor[rgb]{0.37,0.37,0.37}{#1}}
\newcommand{\KeywordTok}[1]{\textcolor[rgb]{0.00,0.23,0.31}{#1}}
\newcommand{\NormalTok}[1]{\textcolor[rgb]{0.00,0.23,0.31}{#1}}
\newcommand{\OperatorTok}[1]{\textcolor[rgb]{0.37,0.37,0.37}{#1}}
\newcommand{\OtherTok}[1]{\textcolor[rgb]{0.00,0.23,0.31}{#1}}
\newcommand{\PreprocessorTok}[1]{\textcolor[rgb]{0.68,0.00,0.00}{#1}}
\newcommand{\RegionMarkerTok}[1]{\textcolor[rgb]{0.00,0.23,0.31}{#1}}
\newcommand{\SpecialCharTok}[1]{\textcolor[rgb]{0.37,0.37,0.37}{#1}}
\newcommand{\SpecialStringTok}[1]{\textcolor[rgb]{0.13,0.47,0.30}{#1}}
\newcommand{\StringTok}[1]{\textcolor[rgb]{0.13,0.47,0.30}{#1}}
\newcommand{\VariableTok}[1]{\textcolor[rgb]{0.07,0.07,0.07}{#1}}
\newcommand{\VerbatimStringTok}[1]{\textcolor[rgb]{0.13,0.47,0.30}{#1}}
\newcommand{\WarningTok}[1]{\textcolor[rgb]{0.37,0.37,0.37}{\textit{#1}}}

\providecommand{\tightlist}{%
  \setlength{\itemsep}{0pt}\setlength{\parskip}{0pt}}\usepackage{longtable,booktabs,array}
\usepackage{calc} % for calculating minipage widths
% Correct order of tables after \paragraph or \subparagraph
\usepackage{etoolbox}
\makeatletter
\patchcmd\longtable{\par}{\if@noskipsec\mbox{}\fi\par}{}{}
\makeatother
% Allow footnotes in longtable head/foot
\IfFileExists{footnotehyper.sty}{\usepackage{footnotehyper}}{\usepackage{footnote}}
\makesavenoteenv{longtable}
\usepackage{graphicx}
\makeatletter
\def\maxwidth{\ifdim\Gin@nat@width>\linewidth\linewidth\else\Gin@nat@width\fi}
\def\maxheight{\ifdim\Gin@nat@height>\textheight\textheight\else\Gin@nat@height\fi}
\makeatother
% Scale images if necessary, so that they will not overflow the page
% margins by default, and it is still possible to overwrite the defaults
% using explicit options in \includegraphics[width, height, ...]{}
\setkeys{Gin}{width=\maxwidth,height=\maxheight,keepaspectratio}
% Set default figure placement to htbp
\makeatletter
\def\fps@figure{htbp}
\makeatother
% definitions for citeproc citations
\NewDocumentCommand\citeproctext{}{}
\NewDocumentCommand\citeproc{mm}{%
  \begingroup\def\citeproctext{#2}\cite{#1}\endgroup}
\makeatletter
 % allow citations to break across lines
 \let\@cite@ofmt\@firstofone
 % avoid brackets around text for \cite:
 \def\@biblabel#1{}
 \def\@cite#1#2{{#1\if@tempswa , #2\fi}}
\makeatother
\newlength{\cslhangindent}
\setlength{\cslhangindent}{1.5em}
\newlength{\csllabelwidth}
\setlength{\csllabelwidth}{3em}
\newenvironment{CSLReferences}[2] % #1 hanging-indent, #2 entry-spacing
 {\begin{list}{}{%
  \setlength{\itemindent}{0pt}
  \setlength{\leftmargin}{0pt}
  \setlength{\parsep}{0pt}
  % turn on hanging indent if param 1 is 1
  \ifodd #1
   \setlength{\leftmargin}{\cslhangindent}
   \setlength{\itemindent}{-1\cslhangindent}
  \fi
  % set entry spacing
  \setlength{\itemsep}{#2\baselineskip}}}
 {\end{list}}
\usepackage{calc}
\newcommand{\CSLBlock}[1]{\hfill\break\parbox[t]{\linewidth}{\strut\ignorespaces#1\strut}}
\newcommand{\CSLLeftMargin}[1]{\parbox[t]{\csllabelwidth}{\strut#1\strut}}
\newcommand{\CSLRightInline}[1]{\parbox[t]{\linewidth - \csllabelwidth}{\strut#1\strut}}
\newcommand{\CSLIndent}[1]{\hspace{\cslhangindent}#1}

\KOMAoption{captions}{tableheading}
\makeatletter
\@ifpackageloaded{caption}{}{\usepackage{caption}}
\AtBeginDocument{%
\ifdefined\contentsname
  \renewcommand*\contentsname{Table of contents}
\else
  \newcommand\contentsname{Table of contents}
\fi
\ifdefined\listfigurename
  \renewcommand*\listfigurename{List of Figures}
\else
  \newcommand\listfigurename{List of Figures}
\fi
\ifdefined\listtablename
  \renewcommand*\listtablename{List of Tables}
\else
  \newcommand\listtablename{List of Tables}
\fi
\ifdefined\figurename
  \renewcommand*\figurename{Figure}
\else
  \newcommand\figurename{Figure}
\fi
\ifdefined\tablename
  \renewcommand*\tablename{Table}
\else
  \newcommand\tablename{Table}
\fi
}
\@ifpackageloaded{float}{}{\usepackage{float}}
\floatstyle{ruled}
\@ifundefined{c@chapter}{\newfloat{codelisting}{h}{lop}}{\newfloat{codelisting}{h}{lop}[chapter]}
\floatname{codelisting}{Listing}
\newcommand*\listoflistings{\listof{codelisting}{List of Listings}}
\makeatother
\makeatletter
\makeatother
\makeatletter
\@ifpackageloaded{caption}{}{\usepackage{caption}}
\@ifpackageloaded{subcaption}{}{\usepackage{subcaption}}
\makeatother
\ifLuaTeX
  \usepackage{selnolig}  % disable illegal ligatures
\fi
\usepackage{bookmark}

\IfFileExists{xurl.sty}{\usepackage{xurl}}{} % add URL line breaks if available
\urlstyle{same} % disable monospaced font for URLs
\hypersetup{
  pdftitle={hubEnsembles: Ensembling Methods in R},
  pdfauthor={Li Shandross; Emily Howerton; Lucie Contamin; Harry Hochheiser; Anna Krystalli; Consortium of Infectious Disease Modeling Hubs; Nicholas G. Reich; Evan L. Ray},
  pdfkeywords={multiple models, aggregation, forecast, prediction},
  colorlinks=true,
  linkcolor={blue},
  filecolor={Maroon},
  citecolor={Blue},
  urlcolor={Blue},
  pdfcreator={LaTeX via pandoc}}

\title{{hubEnsembles}: Ensembling Methods in {R}}
\author{Li Shandross \and Emily Howerton \and Lucie Contamin \and Harry
Hochheiser \and Anna Krystalli \and Consortium of Infectious Disease
Modeling Hubs \and Nicholas G. Reich \and Evan L. Ray}
\date{}

\begin{document}
\maketitle
\begin{abstract}
Combining predictions from multiple models into an ensemble is a widely
used practice across many fields with demonstrated performance benefits.
The {R} package {hubEnsembles} provides a flexible framework for
ensembling various types of predictions, including point estimates and
probabilistic predictions. A range of common methods for generating
ensembles are supported, including weighted averages, quantile averages,
and linear pools. The {hubEnsembles} package fits within a broader
framework of open-source software and data tools called the
``hubverse'', which facilitates the development and management of
collaborative modelling exercises.
\end{abstract}

\section{Introduction}\label{sec-intro}

Predictions of future outcomes are essential to planning and decision
making, yet generating reliable predictions of the future is
challenging. One method for overcoming this challenge is combining
predictions across multiple, independent models. These combination
methods (also called aggregation or ensembling) have been repeatedly
shown to produce predictions that are more accurate\textsuperscript{1,2}
and more consistent\textsuperscript{3} than individual models. Because
of the clear performance benefits, multi-model ensembles are commonplace
across fields, including weather\textsuperscript{4},
climate\textsuperscript{5}, and economics\textsuperscript{6}. More
recently, multi-model ensembles have been used to improve predictions of
infectious disease outbreaks\textsuperscript{7--11}.

In the rapidly growing field of outbreak forecasting, there are many
proposed methods for generating ensembles. Generally, these methods
differ in at least one of two ways: (1) the function used to combine or
``average'' predictions, and (2) how predictions are weighted when
performing the combination. No one method is universally ``the best''; a
simple average of predictions works surprisingly well across a range of
settings\textsuperscript{9,12,13} for established theoretical
reasons\textsuperscript{14}. However, more complex approaches have also
been shown to have benefits in some settings\textsuperscript{10,15--17}.
Here, we present the {hubEnsembles} package, which provides a flexible
framework for generating ensemble predictions from multiple models.
Complementing other software for combining predictions from multiple
models
(e.g.,\textsuperscript{18};\textsuperscript{19};\textsuperscript{20};\textsuperscript{21}),
{hubEnsembles} supports multiple types of predictions, including point
estimates and different kinds of probabilistic predictions. Throughout,
we will use the term ``prediction'' to refer to any kind of model output
that may be combined including a forecast, a scenario projection, or a
parameter estimate.

The {hubEnsembles} package is part of the ``hubverse'' collection of
open-source software and data tools. The hubverse project facilitates
the development and management of collaborative modelling
exercises\textsuperscript{22}. The broader hubverse initiative is
motivated by the demonstrated benefits of collaborative
hubs\textsuperscript{23,24}, including performance benefits of
multi-model ensembles and the desire for standardization across such
hubs. In this paper, we focus specifically on the functionality
encompassed in {hubEnsembles}. We provide an overview of the methods
implemented, including mathematical definitions and properties
(Section~\ref{sec-defs}) as well as implementation details
(Section~\ref{sec-implementation}), a basic demonstration of
functionality with simple examples (Section~\ref{sec-simple-ex}), and a
more in-depth analysis using real influenza forecasts
(Section~\ref{sec-case-study}) that motivates a discussion and
comparison of the various methods (Section~\ref{sec-conclusions}).

\section{Mathematical definitions and properties of ensemble
methods}\label{sec-defs}

The {hubEnsembles} package supports both point predictions and
probabilistic predictions of different formats. A point prediction gives
a single estimate of a future outcome while a probabilistic prediction
provides an estimated probability distribution over a set of future
outcomes. We use \(N\) to denote the total number of individual
predictions that the ensemble will combine. For example, these
predictions will often be produced by different statistical or
mathematical models, and \(N\) is the total number of models that have
provided predictions. Individual predictions will be indexed by the
subscript \(i\). Optionally, the package allows for calculating
ensembles that use a weight \(w_i\) for each prediction; we define the
set of model-specific weights as
\(\pmb{w} = \{w_i | i \in 1, ..., N\}\). Informally, predictions with a
larger weight have a greater influence on the ensemble prediction,
though the details of this depend on the ensemble method (described
further below).

For a set of \(N\) point predictions,
\(\pmb{p} = \{p_i|i \in 1, ..., N\}\), each from a distinct model \(i\),
the {hubEnsembles} package can compute an ensemble of these predictions

\[
p_E = C(\pmb{p}, \pmb{w}) 
\]

using any function \(C\) and any set of model-specific weights
\(\pmb{w}\). For example, an arithmetic average of predictions yields
\(p_E = \sum_{i=1}^Np_iw_i\), where the weights are non-negative and sum
to 1. If \(w_i = 1/N\) for all \(i\), all predictions will be equally
weighted. This framework can also support more complex functions for
aggregation, such as a (weighted) median or geometric mean.

For probabilistic predictions, there are two commonly used classes of
methods to average or ensemble multiple predictions: quantile averaging
(also called a Vincent average\textsuperscript{25}) and probability
averaging (also called a distributional mixture or linear opinion
pool\textsuperscript{26})\textsuperscript{27}. To define these two
classes of methods, let \(F(x)\) be a cumulative density function (CDF)
defined over values \(x\) of the target variable for the prediction, and
\(F^{-1}(\theta)\) be the corresponding quantile function defined over
quantile levels \(\theta \in [0, 1]\). Throughout this article, we may
refer to \(x\) as either a `value of the target variable' or a
`quantile' depending on the context, and similarly we may refer to
\(\theta\) as either a `quantile level' or a `(cumulative) probability'.
Additionally, we will use \(f(x)\) to denote a probability mass function
(PMF) for a prediction of a discrete variable or a discretization (such
as binned values) of a continuous variable.

The quantile average combines a set of quantile functions,
\(\mathcal{Q} = \{F_i^{-1}(\theta)| i \in 1,...,N \}\), with a given set
of weights, \(\pmb{w}\), as \[
F^{-1}_Q(\theta) = C_Q(\mathcal{Q}, \pmb{w}) = \sum_{i = 1}^Nw_iF^{-1}_i(\theta).
\]

This computes the average value of predictions across different models
for each fixed quantile level \(\theta\). For a normal distribution or
any distribution with a shape and scale parameter, the resulting
quantile average will be the same type of distribution, with shape and
scale parameters that are the average of the shape and scale parameters
from the individual distributions
(Figure~\ref{fig-example-quantile-average-and-linear-pool}, panel B). In
other words, this method inteprets the predictive probability
distributions that are being combined as uncertain estimates of a single
true distribution. It is also possible to use other combination
functions, such as a weighted median, to combine quantile predictions.

The probability average or linear pool is calculated by averaging
probabilities across predictions for a fixed value of the target
variable, \(x\). In other words, for a set
\(\mathcal{F} = \{F_i(x)| i \in 1,...,N \}\) containing the values of
CDFs at the point \(x\) and weights \(\pmb{w}\), the linear pool is
calculated as

\[
F_{LOP}(x) = C_{LOP}(\mathcal{F}, \pmb{w}) = \sum_{i = 1}^Nw_iF_i(x). 
\]

For a set of PMF values, \(\{f_i(x)|i \in 1, ..., N\}\), the linear pool
can be equivalently calculated:
\(f_{LOP}(x) = \sum_{i = 1}^N w_i f_i(x)\). Statistically this amounts
to a mixture of the probability distributions, and the resulting
probability distribution can be interpreted as one where the constituent
probability distributions represent alternative predictions of the
future, each of which has a probability \(w_i\) of being the true one.
For a visual depiction of these equations, see
Figure~\ref{fig-example-quantile-average-and-linear-pool} below.

The different averaging methods for probabilistic predictions yield
different properties of the resulting ensemble distribution. For
example, the variance of the linear pool is
\(\sigma^2_{LOP} = \sum_{i=1}^Nw_i\sigma_i^2 + \sum_{i=1}^Nw_i(\mu_i-\mu_{LOP})^2\),
where \(\mu_i\) is the mean and \(\sigma^2_i\) is the variance of
individual prediction \(i\), and although there is no closed-form
variance for the quantile average, the variance of the quantile average
will always be less than or equal to that of the linear
pool\textsuperscript{27}. Both methods generate distributions with the
same mean, \(\mu_Q = \mu_{LOP} = \sum_{i=1}^Nw_i\mu_i\), which is the
mean of individual model means\textsuperscript{27}. The linear pool
method preserves variation between individual models, whereas the
quantile average cancels away this variation under the assumption it
constitutes sampling error\textsuperscript{28}.

\section{Model implementation details}\label{sec-implementation}

To understand how these methods are implemented in {hubEnsembles}, we
first must define the conventions employed by the hubverse and its
packages for representing and working with model predictions. We begin
with a short overview of concepts and conventions needed to utilize the
{hubEnsembles} package, supplemented by example predictions provided by
the hubverse, then explain the implementation of the two ensembling
functions included in the package, \texttt{simple\_ensemble()} and
\texttt{linear\_pool()}.

\subsection{Hubverse terminology and
conventions}\label{hubverse-terminology-and-conventions}

A central concept in the hubverse effort is ``model output''. Model
output is a specially formatted tabular representation of predictions.
Each row represents a single, unique prediction with each column
providing information about what is being predicted, its scope, and its
value. Per hubverse convention, each column serves one of three
purposes: (i) denote which model has produced the prediction (called the
``model ID''), (ii) provide details about what is being predicted
(called the ``task IDs''), or (iii) specify the prediction itself and
how it is represented (called the ``model output
representation'')\textsuperscript{22}.

Predictions are assumed to be generated by distinct models, typically
developed and run by a modeling team of one or more individuals. Each
model should have a unique identifier that is stored in the
\texttt{model\_id} column. Then, the details of the outcome being
predicted can be stored in a series of task ID columns, the second type
of column. These task ID columns may also include additional
information, such as any conditions or assumptions that were used to
generate the predictions\textsuperscript{22}. For example, short-term
forecasts of incident influenza hospitalizations in the US at different
locations and amounts of time in the future might represent this
information using a \texttt{target} column with the value ``wk inc flu
hosp'', a \texttt{location} column identifying the location being
predicted (not shown), a \texttt{reference\_date} column with the
``starting point'' of the forecasts (not shown), and a \texttt{horizon}
column with the number of steps ahead that the forecast is predicting
relative to the \texttt{reference\_date}
(Table~\ref{tbl-example-forecasts}). All these variables make up the
task ID columns.

\begin{longtable}[]{@{}
  >{\raggedright\arraybackslash}p{(\columnwidth - 10\tabcolsep) * \real{0.1447}}
  >{\raggedright\arraybackslash}p{(\columnwidth - 10\tabcolsep) * \real{0.2105}}
  >{\raggedleft\arraybackslash}p{(\columnwidth - 10\tabcolsep) * \real{0.1316}}
  >{\raggedright\arraybackslash}p{(\columnwidth - 10\tabcolsep) * \real{0.1842}}
  >{\raggedright\arraybackslash}p{(\columnwidth - 10\tabcolsep) * \real{0.2237}}
  >{\raggedleft\arraybackslash}p{(\columnwidth - 10\tabcolsep) * \real{0.1053}}@{}}

\caption{\label{tbl-example-forecasts}Example of forecasts for incident
influenza hospitalizations, formatted according to hubverse standards.
Quantile predictions for the median and 50\% prediction intervals from a
single model are shown for four distinct horizons. The \texttt{location}
and \texttt{reference\_date} columns have been omitted for brevity; all
forecasts in this example were made on 2022-12-17 for Massachusetts.
These predictions are a modified subset of the
\texttt{forecast\_outputs} data provided by the {hubExamples} package.}

\tabularnewline

\toprule\noalign{}
\begin{minipage}[b]{\linewidth}\raggedright
\texttt{model\_id}
\end{minipage} & \begin{minipage}[b]{\linewidth}\raggedright
\texttt{target}
\end{minipage} & \begin{minipage}[b]{\linewidth}\raggedleft
\texttt{horizon}
\end{minipage} & \begin{minipage}[b]{\linewidth}\raggedright
\texttt{output\_type}
\end{minipage} & \begin{minipage}[b]{\linewidth}\raggedright
\texttt{output\_type\_id}
\end{minipage} & \begin{minipage}[b]{\linewidth}\raggedleft
\texttt{value}
\end{minipage} \\
\midrule\noalign{}
\endhead
\bottomrule\noalign{}
\endlastfoot
model-X & wk inc flu hosp & 0 & quantile & 0.25 & 514 \\
model-X & wk inc flu hosp & 0 & quantile & 0.5 & 596 \\
model-X & wk inc flu hosp & 0 & quantile & 0.75 & 713 \\
model-X & wk inc flu hosp & 1 & quantile & 0.25 & 563 \\
model-X & wk inc flu hosp & 1 & quantile & 0.5 & 664 \\
model-X & wk inc flu hosp & 1 & quantile & 0.75 & 803 \\
model-X & wk inc flu hosp & 2 & quantile & 0.25 & 469 \\
model-X & wk inc flu hosp & 2 & quantile & 0.5 & 575 \\
model-X & wk inc flu hosp & 2 & quantile & 0.75 & 705 \\
model-X & wk inc flu hosp & 3 & quantile & 0.25 & 324 \\
model-X & wk inc flu hosp & 3 & quantile & 0.5 & 408 \\
model-X & wk inc flu hosp & 3 & quantile & 0.75 & 512 \\

\end{longtable}

Alternatively, longer-term scenario projections may require different
task ID columns. For example, projections of incident COVID-19 cases in
the US at different locations, amounts of time in the future, and under
different assumed conditions may use a \texttt{target} column of ``inc
case'', a \texttt{location} column specifying the location being
predicted (not shown), an \texttt{origin\_date} column specifying the
date on which the projections were made (not shown), a \texttt{horizon}
column describing the number of steps ahead that the projection is
predicting relative to the \texttt{origin\_date}, and a
\texttt{scenario\_id} column denoting the future conditions that were
modeled and are projected to result in the specified number of incident
cases (Table~\ref{tbl-example-scenarios}). Different modeling efforts
may use different sets of task ID columns and values to specify their
prediction goals, or may simply choose distinct names to represent the
same concept (e.g., \texttt{reference\_date} versus
\texttt{origin\_date} for a date task ID). Additional examples of task
ID variables are available on the hubverse documentation
website\textsuperscript{22}.

\begin{longtable}[]{@{}
  >{\raggedright\arraybackslash}p{(\columnwidth - 12\tabcolsep) * \real{0.1294}}
  >{\raggedright\arraybackslash}p{(\columnwidth - 12\tabcolsep) * \real{0.1059}}
  >{\raggedleft\arraybackslash}p{(\columnwidth - 12\tabcolsep) * \real{0.1176}}
  >{\raggedright\arraybackslash}p{(\columnwidth - 12\tabcolsep) * \real{0.1647}}
  >{\raggedright\arraybackslash}p{(\columnwidth - 12\tabcolsep) * \real{0.1647}}
  >{\raggedleft\arraybackslash}p{(\columnwidth - 12\tabcolsep) * \real{0.2000}}
  >{\raggedleft\arraybackslash}p{(\columnwidth - 12\tabcolsep) * \real{0.1176}}@{}}

\caption{\label{tbl-example-scenarios}Example of scenario projections
for incident COVID-19 cases, formatted according to hubverse standards.
Quantile predictions for the median and 50\% prediction intervals from a
single model are shown for four distinct scenarios. The
\texttt{location} and \texttt{origin\_date} columns have been omitted
for brevity; all forecasts in this example were made on 2021-03-07 for
the US. These predictions are a modified subset of the
\texttt{scenario\_outputs} data provided by the {hubExamples} package.}

\tabularnewline

\toprule\noalign{}
\begin{minipage}[b]{\linewidth}\raggedright
\texttt{model\_id}
\end{minipage} & \begin{minipage}[b]{\linewidth}\raggedright
\texttt{target}
\end{minipage} & \begin{minipage}[b]{\linewidth}\raggedleft
\texttt{horizon}
\end{minipage} & \begin{minipage}[b]{\linewidth}\raggedright
\texttt{scenario\_id}
\end{minipage} & \begin{minipage}[b]{\linewidth}\raggedright
\texttt{output\_type}
\end{minipage} & \begin{minipage}[b]{\linewidth}\raggedleft
\texttt{output\_type\_id}
\end{minipage} & \begin{minipage}[b]{\linewidth}\raggedleft
\texttt{value}
\end{minipage} \\
\midrule\noalign{}
\endhead
\bottomrule\noalign{}
\endlastfoot
model-Y & inc case & 26 & A & quantile & 0.25 & 1147.00 \\
model-Y & inc case & 26 & A & quantile & 0.50 & 1516.00 \\
model-Y & inc case & 26 & A & quantile & 0.75 & 1929.00 \\
model-Y & inc case & 26 & B & quantile & 0.25 & 4241.75 \\
model-Y & inc case & 26 & B & quantile & 0.50 & 4952.50 \\
model-Y & inc case & 26 & B & quantile & 0.75 & 6002.25 \\
model-Y & inc case & 26 & C & quantile & 0.25 & 32478.75 \\
model-Y & inc case & 26 & C & quantile & 0.50 & 38594.50 \\
model-Y & inc case & 26 & C & quantile & 0.75 & 44975.50 \\
model-Y & inc case & 26 & D & quantile & 0.25 & 85811.75 \\
model-Y & inc case & 26 & D & quantile & 0.50 & 99841.50 \\
model-Y & inc case & 26 & D & quantile & 0.75 & 113963.50 \\

\end{longtable}

The third group of columns in model output specify the model predictions
and details about how the predictions are represented. This ``model
output representation'' includes the predicted values along with
metadata that specifies how the predictions are conveyed and always
consists of the same three columns: (1) \texttt{output\_type}, (2)
\texttt{output\_type\_id}, and (3) \texttt{value}. The
\texttt{output\_type} column defines how the prediction is represented
and may be one of \texttt{"mean"} or \texttt{"median"} for point
predictions, or \texttt{"quantile"}, \texttt{"cdf"}, \texttt{"pmf"}, or
\texttt{"sample"} for probabilistic predictions. The
\texttt{output\_type\_id} provides additional identifying information
for a prediction and is specific to the particular \texttt{output\_type}
(see Table~\ref{tbl-model-output-rep}). For quantile predictions, the
\texttt{output\_type\_id} is a numeric value between 0 and 1 specifying
the cumulative probability associated with the quantile prediction. In
the notation we defined above, the \texttt{output\_type\_id} corresponds
to \(\theta\) and the \texttt{value} is the quantile prediction
\(F^{-1}(\theta)\). For CDF or PMF predictions, the
\texttt{output\_type\_id} is the target variable value \(x\) at which
the cumulative distribution function or probability mass function for
the predictive distribution should be evaluated, and the \texttt{value}
column contains the predicted \(F(x)\) or \(f(x)\), respectively.
Requirements for the values of the \texttt{output\_type\_id} and
\texttt{value} columns associated with each valid output type are
summarized in Table~\ref{tbl-model-output-rep}.

This representation of predictive model output is codified by the
\texttt{model\_out\_tbl} S3 class in the {hubUtils} package, one of the
foundational hubverse packages. Although this S3 class is required for
all {hubEnsembles} functions, model predictions in other formats can
easily be transformed using the \texttt{as\_model\_out\_tbl()} function
from {hubUtils}. An example of this transformation is provided in
Section~\ref{sec-case-study}.

\begin{longtable}[]{@{}
  >{\raggedright\arraybackslash}p{(\columnwidth - 4\tabcolsep) * \real{0.1899}}
  >{\raggedright\arraybackslash}p{(\columnwidth - 4\tabcolsep) * \real{0.3544}}
  >{\raggedright\arraybackslash}p{(\columnwidth - 4\tabcolsep) * \real{0.4557}}@{}}
\caption{A table summarizing how the model output representation columns
are used for predictions of different output types. Adapted from
\url{https://hubverse.io/en/latest/user-guide/model-output.html\#formats-of-model-output}}\label{tbl-model-output-rep}\tabularnewline
\toprule\noalign{}
\begin{minipage}[b]{\linewidth}\raggedright
\texttt{output\_type}
\end{minipage} & \begin{minipage}[b]{\linewidth}\raggedright
\texttt{output\_type\_id}
\end{minipage} & \begin{minipage}[b]{\linewidth}\raggedright
\texttt{value}
\end{minipage} \\
\midrule\noalign{}
\endfirsthead
\toprule\noalign{}
\begin{minipage}[b]{\linewidth}\raggedright
\texttt{output\_type}
\end{minipage} & \begin{minipage}[b]{\linewidth}\raggedright
\texttt{output\_type\_id}
\end{minipage} & \begin{minipage}[b]{\linewidth}\raggedright
\texttt{value}
\end{minipage} \\
\midrule\noalign{}
\endhead
\bottomrule\noalign{}
\endlastfoot
\texttt{mean} & NA (not used for mean predictions) & Numeric: The mean
of the predictive distribution \\
\texttt{median} & NA (not used for median predictions) & Numeric: The
median of the predictive distribution \\
\texttt{quantile} & Numeric between 0.0 and 1.0: A quantile level &
Numeric: The quantile of the predictive distribution at the quantile
level specified by the \texttt{output\_type\_id} \\
\texttt{cdf} & String or numeric naming a possible value of the target
variable & Numeric between 0.0 and 1.0: The cumulative probability of
the predictive distribution at the value of the outcome variable
specified by the \texttt{output\_type\_id} \\
\texttt{pmf} & String naming a possible category of a discrete outcome
variable & Numeric between 0.0 and 1.0: The probability mass of the
predictive distribution when evaluated at a specified level of a
discrete outcome variable \\
\texttt{sample} & Integer or string specifying the sample index &
Numeric: A sample from the predictive distribution \\
\end{longtable}

\subsection{Ensemble functions in hubEnsembles}\label{sec-ens-fns}

The {hubEnsembles} package includes two functions that perform ensemble
calculations: \texttt{simple\_ensemble()}, which applies some function
to each model prediction, and \texttt{linear\_pool()}, which computes an
ensemble using the linear opinion pool method. In the following
sections, we outline the implementation details for each function and
how these implementations correspond to the statistical ensembling
methods described in Section~\ref{sec-defs}. A short description of the
calculation performed by each function is summarized by output type in
Table~\ref{tbl-fns-by-output-type}.

\begin{longtable}[]{@{}
  >{\raggedright\arraybackslash}p{(\columnwidth - 4\tabcolsep) * \real{0.1400}}
  >{\raggedright\arraybackslash}p{(\columnwidth - 4\tabcolsep) * \real{0.5000}}
  >{\raggedright\arraybackslash}p{(\columnwidth - 4\tabcolsep) * \real{0.3600}}@{}}
\caption{Summary of ensemble function calculations for each output type.
The ensemble function determines the operation that is performed, and in
the case of probabilistic output types (quantile, CDF, PMF), this also
determines what ensemble distribution is generated (quantile average,
\(F_{Q}^{-1}(\theta)\), or linear pool, \(F_{LOP}(x)\)). The ensembled
predictions are returned in the same output type as the inputs. Thus,
the output type determines how the resulting ensemble distribution is
summarized (as a quantile, \(F^{-1}(\theta)\), cumulative probability,
\(F(x)\), or probability \(f(x)\)). Estimating individual model
cumulative probabilities is required to compute a
\texttt{linear\_pool()} for predictions of \texttt{quantile} output
type; see Section~\ref{sec-linear-pool} on the linear pool operation for
details. In the case of \texttt{simple\_ensemble()}, we show the
calculations for the default case where \texttt{agg\_fun\ =\ mean};
however, if another aggregation function is chosen (e.g.,
\texttt{agg\_fun\ =\ median}), that calculation would be performed
instead. For example,
\texttt{simple\_ensemble(...,\ agg\_fun\ =\ median)} applied to
predictions of mean output type would return the median of individual
model means.}\label{tbl-fns-by-output-type}\tabularnewline
\toprule\noalign{}
\begin{minipage}[b]{\linewidth}\raggedright
\texttt{output\_type}
\end{minipage} & \begin{minipage}[b]{\linewidth}\raggedright
\texttt{simple\_ensemble(...,\ agg\_fun=mean)}
\end{minipage} & \begin{minipage}[b]{\linewidth}\raggedright
\texttt{linear\_pool()}
\end{minipage} \\
\midrule\noalign{}
\endfirsthead
\toprule\noalign{}
\begin{minipage}[b]{\linewidth}\raggedright
\texttt{output\_type}
\end{minipage} & \begin{minipage}[b]{\linewidth}\raggedright
\texttt{simple\_ensemble(...,\ agg\_fun=mean)}
\end{minipage} & \begin{minipage}[b]{\linewidth}\raggedright
\texttt{linear\_pool()}
\end{minipage} \\
\midrule\noalign{}
\endhead
\bottomrule\noalign{}
\endlastfoot
\texttt{mean} & mean of individual model means & mean of individual
model means \\
\texttt{median} & mean of individual model medians & NA \\
\texttt{quantile} & mean of individual model target variable values at
each quantile level, \(F^{-1}_Q(\theta)\) & quantiles of the
distribution are obtained by computing the mean of estimated individual
model cumulative probabilities at each target variable value,
\(F^{-1}_{LOP}(\theta)\) \\
\texttt{cdf} & mean of individual model cumulative probabilities at each
target variable value, \(F_{LOP}(x)\) & mean of individual model
cumulative probabilities at each target variable value,
\(F_{LOP}(x)\) \\
\texttt{pmf} & mean of individual model bin or category probabilities at
each target variable value, \(f_{LOP}(x)\) & mean of individual model
bin or category probabilities at each target variable value,
\(f_{LOP}(x)\) \\
\texttt{sample} & NA & samples of the distribution are obtained by
stratified draw from individual models' samples for equal
representation \\
\end{longtable}

\subsubsection{Simple ensemble}\label{sec-simple-ensemble}

The \texttt{simple\_ensemble()} function directly computes an ensemble
from component model outputs by combining them via some function (\(C\))
within each unique combination of task ID variables, output types, and
output type IDs. This function can be used to summarize predictions of
output types mean, median, quantile, CDF, and PMF. The mechanics of the
ensemble calculations are the same for each of the output types, though
the resulting statistical ensembling method differs for different output
types (Table~\ref{tbl-fns-by-output-type}).

By default, \texttt{simple\_ensemble()} uses the mean for the
aggregation function \(C\) and equal weights for all models. For point
predictions with a mean or median output type, the resulting ensemble
prediction is an equally weighted average of the individual models'
predictions. For probabilistic predictions in a quantile format, by
default \texttt{simple\_ensemble()} calculates an equally weighted
average of individual model target variable values at each quantile
level, which is equivalent to a quantile average. For model outputs in a
CDF or PMF format, by default \texttt{simple\_ensemble()} computes an
equally weighted average of individual model (cumulative or bin)
probabilities at each target variable value, which is equivalent to the
linear pool method.

Any aggregation function \(C\) may be specified by the user. For
example, a median ensemble may also be created by specifying
\texttt{median} as the aggregation function, or a custom function may be
passed to the \texttt{agg\_fun} argument to create other ensemble types.
Similarly, model weights can be specified to create a weighted ensemble.

\subsubsection{Linear pool}\label{sec-linear-pool}

The \texttt{linear\_pool()} function implements the linear opinion pool
(LOP) method for ensembling predictions. Currently, this function can be
used to combine predictions with output types mean, quantile, CDF, PMF,
and sample. Unlike \texttt{simple\_ensemble()}, this function handles
its computation differently based on the output type. For the CDF, PMF,
and mean output types, the linear pool method is equivalent to calling
\texttt{simple\_ensemble()} with a mean aggregation function (see
Table~\ref{tbl-fns-by-output-type}), since \texttt{simple\_ensemble()}
produces a linear pool prediction (an average of individual model
cumulative or bin probabilities).

For the sample output type, the LOP method collects a stratified draw of
the individual models' predictions and pools them into a single ensemble
distribution. By default, all samples are used to create this ensemble.
Additionally, only equally-weighted linear pools of samples are
supported by the {hubEnsembles} package during this time. Samples may
also be converted to another common output type such as quantiles or bin
probabilities (as the main scientific interest often concerns a summary
of samples), and other ensemble methods may then be utilized on the
alternate output type.

For the quantile output type, implementation of LOP is comparatively
less straightforward. This is because LOP averages cumulative
probabilities at each value of the target variable, but the predictions
are given as quantiles (on the scale of the target variable) for fixed
quantile levels. The value for these quantile predictions will generally
differ between models; hence, we are typically not provided cumulative
probabilities at the same values of the target variable for all
component predictions. This lack of alignment between cumulative
probabilities for the same target variable values impedes computation of
LOP from quantile predictions and is illustrated in panel A of
Figure~\ref{fig-example-quantile-average-and-linear-pool}.

\begin{figure}

\centering{

\includegraphics{hubEnsembles_manuscript_files/figure-pdf/fig-example-quantile-average-and-linear-pool-1.pdf}

}

\caption{\label{fig-example-quantile-average-and-linear-pool}(Panel A)
Example of quantile output type predictions. Solid points show model
output collected for seven fixed quantile levels (\(\theta\) = 0.01,
0.1, 0.3, 0.5, 0.7, 0.9, and 0.99) from two distributions
(\(N(100, 10)\) in purple and \(N(120, 5)\) in green), with the
underlying cumulative distribution functions (CDFs) shown with curves.
The y-axis ticks show each of the fixed quantile levels. The associated
values for each fixed quantile level do not align across distributions
(vertical lines). (Panel B) Quantile average ensemble, which is
calculated by averaging values for each fixed quantile level
(represented by horizontal dashed gray lines). The distributions and
corresponding model outputs from panel A are re-plotted and the black
line shows the resulting quantile average ensemble. Inset shows
corresponding probability density functions (PDFs). (Panel C) Linear
pool ensemble, which is calculated by averaging cumulative probabilities
for each fixed value (represented by vertical dashed gray lines). The
distributions and corresponding model outputs from panel A are
re-plotted. To calculate the linear pool in this case, where model
outputs are not defined for the same values, the model outputs are used
to interpolate the full CDF for each distribution from which quantiles
can be extracted for fixed values (shown with open circles). The black
line shows the resulting linear pool average ensemble. Inset shows
corresponding PDFs.}

\end{figure}%

Given that LOP cannot be directly calculated from quantile predictions,
we must first obtain an estimate of the CDF for each component
distribution from the provided quantiles, combine the CDFs, then
calculate the quantiles using the ensemble's CDF. We perform this
calculation in three main steps, assisted by the {distfromq}
package\textsuperscript{29} for the first two:

\begin{enumerate}
\def\labelenumi{\arabic{enumi}.}
\tightlist
\item
  Interpolate and extrapolate from the provided quantiles for each
  component model to obtain an estimate of the CDF of that particular
  distribution.
\item
  Draw samples from each component model distribution. To reduce Monte
  Carlo variability, we use quasi-random samples corresponding to
  quantiles of the estimated distribution\textsuperscript{30}.
\item
  Pool the samples from all component models and extract the desired
  quantiles.
\end{enumerate}

For step 1, functionality in the {distfromq} package uses a monotonic
cubic spline for interpolation on the interior of the provided
quantiles. The user may choose one of several distributions to perform
extrapolation of the CDF tails. These include normal, lognormal, and
cauchy distributions, with ``normal'' set as the default. A
location-scale parameterization is used, with separate location and
scale parameters chosen in the lower and upper tails so as to match the
two most extreme quantiles. The sampling process described in steps 2
and 3 approximates the linear pool calculation described in
Section~\ref{sec-defs}.

\section{Basic demonstration of functionality}\label{sec-simple-ex}

In this section, we provide a simple example to illustrate the two main
functions in {hubEnsembles}, \texttt{simple\_ensemble()} and
\texttt{linear\_pool()}.

\subsection{Example data: a forecast
hub}\label{example-data-a-forecast-hub}

\begin{longtable}[]{@{}
  >{\raggedright\arraybackslash}p{(\columnwidth - 10\tabcolsep) * \real{0.2169}}
  >{\raggedright\arraybackslash}p{(\columnwidth - 10\tabcolsep) * \real{0.1928}}
  >{\raggedleft\arraybackslash}p{(\columnwidth - 10\tabcolsep) * \real{0.1205}}
  >{\raggedright\arraybackslash}p{(\columnwidth - 10\tabcolsep) * \real{0.1687}}
  >{\raggedright\arraybackslash}p{(\columnwidth - 10\tabcolsep) * \real{0.2048}}
  >{\raggedleft\arraybackslash}p{(\columnwidth - 10\tabcolsep) * \real{0.0964}}@{}}

\caption{\label{tbl-example-model-outputs}Example model output for
forecasts of incident influenza hospitalizations. A subset of example
model output is shown: 1-week ahead quantile forecasts made on
2022-12-17 for Massachusetts from three distinct models; only the median
and 5th, 25th, 75th and 95th quantiles are displayed. The
\texttt{location}, \texttt{reference\_date} and
\texttt{target\_end\_date} columns have been omitted for brevity. This
example data is provided in the {hubExamples} package.}

\tabularnewline

\toprule\noalign{}
\begin{minipage}[b]{\linewidth}\raggedright
\texttt{model\_id}
\end{minipage} & \begin{minipage}[b]{\linewidth}\raggedright
\texttt{target}
\end{minipage} & \begin{minipage}[b]{\linewidth}\raggedleft
\texttt{horizon}
\end{minipage} & \begin{minipage}[b]{\linewidth}\raggedright
\texttt{output\_type}
\end{minipage} & \begin{minipage}[b]{\linewidth}\raggedright
\texttt{output\_type\_id}
\end{minipage} & \begin{minipage}[b]{\linewidth}\raggedleft
\texttt{value}
\end{minipage} \\
\midrule\noalign{}
\endhead
\bottomrule\noalign{}
\endlastfoot
Flusight-baseline & wk inc flu hosp & 1 & quantile & 0.05 & 496 \\
Flusight-baseline & wk inc flu hosp & 1 & quantile & 0.25 & 566 \\
Flusight-baseline & wk inc flu hosp & 1 & quantile & 0.75 & 598 \\
Flusight-baseline & wk inc flu hosp & 1 & quantile & 0.95 & 668 \\
Flusight-baseline & wk inc flu hosp & 1 & median & NA & 582 \\
MOBS-GLEAM\_FLUH & wk inc flu hosp & 1 & quantile & 0.05 & 446 \\
MOBS-GLEAM\_FLUH & wk inc flu hosp & 1 & quantile & 0.25 & 563 \\
MOBS-GLEAM\_FLUH & wk inc flu hosp & 1 & quantile & 0.75 & 803 \\
MOBS-GLEAM\_FLUH & wk inc flu hosp & 1 & quantile & 0.95 & 1097 \\
MOBS-GLEAM\_FLUH & wk inc flu hosp & 1 & median & NA & 664 \\
PSI-DICE & wk inc flu hosp & 1 & quantile & 0.05 & 290 \\
PSI-DICE & wk inc flu hosp & 1 & quantile & 0.25 & 496 \\
PSI-DICE & wk inc flu hosp & 1 & quantile & 0.75 & 712 \\
PSI-DICE & wk inc flu hosp & 1 & quantile & 0.95 & 843 \\
PSI-DICE & wk inc flu hosp & 1 & median & NA & 613 \\

\end{longtable}

We will use an example hub provided by the hubverse to demonstrate the
functionality of the {hubEnsembles} package\textsuperscript{22}. This
hub was generated using modified forecasts from the FluSight forecasting
challenge (discussed in further detail in Section~\ref{sec-case-study}).
The example hub includes both example model output data and target data
(sometimes known as ``truth'' data), which are stored in the
{hubExamples} package as data objects named \texttt{forecast\_outputs}
and \texttt{forecast\_target\_ts}. Note that this model outputs data
contain only a small subset of predictions for select dates, locations,
and output type IDs, far fewer than an actual modeling hub would
typically collect.

The model output data includes quantile, mean and median forecasts of
future incident influenza hospitalizations and PMF forecasts of
hospitalization intensity. Each forecast is made for five task ID
variables, including the location for which the forecast was made
(\texttt{location}), the date on which the forecast was made
(\texttt{reference\_date}), the number of steps ahead
(\texttt{horizon}), the date of the forecast prediction (a combination
of the date the forecast was made and the forecast horizon,
\texttt{target\_end\_date}), and the forecast target (\texttt{target}).
Table~\ref{tbl-example-model-outputs} provides an example set of
quantile forecasts included in this example model output. In
Table~\ref{tbl-example-model-outputs}, we show only the median, the
50\%, and 90\% prediction intervals, although other intervals and mean
forecasts are included in the example model output data.

The {hubExamples} package also provides corresponding target data
(Table~\ref{tbl-example-target-data}) that contains incident influenza
hospitalizations (\texttt{observation}) in a given week (\texttt{date})
for a given location (\texttt{location}). This target data can be used
as calibration data for generating forecasts or for evaluating the
forecasts post hoc. The forecast-specific task ID variables
\texttt{reference\_date} and \texttt{horizon} are not relevant for the
target data.

\begin{longtable}[]{@{}llr@{}}

\caption{\label{tbl-example-target-data}Example target data for incident
influenza hospitalizations. This table includes target data from
2022-11-01 and 2023-02-01. The target data is provided in the
{hubExamples} package.}

\tabularnewline

\toprule\noalign{}
\texttt{date} & \texttt{location} & \texttt{observation} \\
\midrule\noalign{}
\endhead
\bottomrule\noalign{}
\endlastfoot
2022-11-05 & 25 & 31 \\
2022-11-12 & 25 & 43 \\
2022-11-19 & 25 & 79 \\
2022-11-26 & 25 & 221 \\
2022-12-03 & 25 & 446 \\
2022-12-10 & 25 & 578 \\
2022-12-17 & 25 & 694 \\
2022-12-24 & 25 & 769 \\
2022-12-31 & 25 & 733 \\
2023-01-07 & 25 & 466 \\
2023-01-14 & 25 & 238 \\
2023-01-21 & 25 & 122 \\
2023-01-28 & 25 & 71 \\

\end{longtable}

We can plot these forecasts and the target data using the
\texttt{plot\_step\_ahead\_model\_output()} function from {hubVis},
another package for visualizing model outputs from the hubverse suite
(Figure~\ref{fig-plot-ex-mods}). We subset the model output data and the
target data to the location and time horizons we are interested in.

\begin{Shaded}
\begin{Highlighting}[]
\SpecialCharTok{\textgreater{}}\NormalTok{ model\_outputs\_plot }\OtherTok{\textless{}{-}}\NormalTok{ hubExamples}\SpecialCharTok{::}\NormalTok{forecast\_outputs }\SpecialCharTok{|\textgreater{}}
\SpecialCharTok{+}\NormalTok{     hubUtils}\SpecialCharTok{::}\FunctionTok{as\_model\_out\_tbl}\NormalTok{() }\SpecialCharTok{|\textgreater{}}
\SpecialCharTok{+}\NormalTok{     dplyr}\SpecialCharTok{::}\FunctionTok{filter}\NormalTok{(}
\SpecialCharTok{+}\NormalTok{       location }\SpecialCharTok{==} \StringTok{"25"}\NormalTok{,}
\SpecialCharTok{+}\NormalTok{       output\_type }\SpecialCharTok{\%in\%} \FunctionTok{c}\NormalTok{(}\StringTok{"median"}\NormalTok{, }\StringTok{"mean"}\NormalTok{, }\StringTok{"quantile"}\NormalTok{),}
\SpecialCharTok{+}\NormalTok{       reference\_date }\SpecialCharTok{==} \StringTok{"2022{-}12{-}17"}
\SpecialCharTok{+}\NormalTok{     )}
\SpecialCharTok{\textgreater{}}\NormalTok{ target\_data\_plot }\OtherTok{\textless{}{-}}\NormalTok{ hubExamples}\SpecialCharTok{::}\NormalTok{forecast\_target\_ts }\SpecialCharTok{|\textgreater{}}
\SpecialCharTok{+}\NormalTok{     dplyr}\SpecialCharTok{::}\FunctionTok{filter}\NormalTok{(}
\SpecialCharTok{+}\NormalTok{       location }\SpecialCharTok{==} \StringTok{"25"}\NormalTok{,}
\SpecialCharTok{+}\NormalTok{       date }\SpecialCharTok{\textgreater{}=} \StringTok{"2022{-}11{-}01"}\NormalTok{, date }\SpecialCharTok{\textless{}=} \StringTok{"2023{-}02{-}01"}
\SpecialCharTok{+}\NormalTok{     )}
\SpecialCharTok{\textgreater{}} 
\ErrorTok{\textgreater{}}\NormalTok{ hubVis}\SpecialCharTok{::}\FunctionTok{plot\_step\_ahead\_model\_output}\NormalTok{(}
\SpecialCharTok{+}     \AttributeTok{model\_out\_tbl =}\NormalTok{ model\_outputs\_plot,}
\SpecialCharTok{+}     \AttributeTok{target\_data =}\NormalTok{ target\_data\_plot,}
\SpecialCharTok{+}     \AttributeTok{facet =} \StringTok{"model\_id"}\NormalTok{,}
\SpecialCharTok{+}     \AttributeTok{facet\_nrow =} \DecValTok{1}\NormalTok{,}
\SpecialCharTok{+}     \AttributeTok{interactive =} \ConstantTok{FALSE}\NormalTok{,}
\SpecialCharTok{+}     \AttributeTok{intervals =} \FunctionTok{c}\NormalTok{(}\FloatTok{0.5}\NormalTok{, }\FloatTok{0.9}\NormalTok{),}
\SpecialCharTok{+}     \AttributeTok{show\_legend =} \ConstantTok{FALSE}\NormalTok{,}
\SpecialCharTok{+}     \AttributeTok{use\_median\_as\_point =} \ConstantTok{TRUE}\NormalTok{,}
\SpecialCharTok{+}     \AttributeTok{x\_col\_name =} \StringTok{"target\_end\_date"}\NormalTok{, }
\SpecialCharTok{+}     \AttributeTok{x\_target\_col\_name =} \StringTok{"date"}
\SpecialCharTok{+}\NormalTok{   ) }\SpecialCharTok{+}
\SpecialCharTok{+}     \FunctionTok{theme\_bw}\NormalTok{() }\SpecialCharTok{+}
\SpecialCharTok{+}     \FunctionTok{labs}\NormalTok{(}\AttributeTok{y =} \StringTok{"incident hospitalizations"}\NormalTok{)}
\end{Highlighting}
\end{Shaded}

\begin{figure}[H]

\centering{

\includegraphics{hubEnsembles_manuscript_files/figure-pdf/fig-plot-ex-mods-1.pdf}

}

\caption{\label{fig-plot-ex-mods}One example set of quantile forecasts
for weekly incident influenza hospitalizations in Massachusetts from
each of three models (panels). Forecasts are represented by a median
(line), 50\% and 90\% prediction intervals (ribbons). Gray points
represent observed incident hospitalizations.}

\end{figure}%

Next, we examine the PMF target in the example model output data. For
this target, teams forecasted the probability that hospitalization
intensity will be ``low'', ``moderate'', ``high'', or ``very high''.
These hospitalization intensity categories are determined by thresholds
for weekly hospital admissions per 100,000 population. In other words,
``low'' hospitalization intensity in a given week means few incident
influenza hospitalizations per 100,000 population are predicted, whereas
``very high'' hospitalization intensity means many hospitalizations per
100,000 population are predicted. These forecasts are made for the same
task ID variables as the \texttt{quantile} forecasts of incident
hospitalizations, other than the target, which is ``wk flu hosp rate
category'' for these categorical predictions.

\begin{longtable}[]{@{}
  >{\raggedright\arraybackslash}p{(\columnwidth - 10\tabcolsep) * \real{0.1935}}
  >{\raggedright\arraybackslash}p{(\columnwidth - 10\tabcolsep) * \real{0.2796}}
  >{\raggedleft\arraybackslash}p{(\columnwidth - 10\tabcolsep) * \real{0.1075}}
  >{\raggedright\arraybackslash}p{(\columnwidth - 10\tabcolsep) * \real{0.1505}}
  >{\raggedright\arraybackslash}p{(\columnwidth - 10\tabcolsep) * \real{0.1828}}
  >{\raggedleft\arraybackslash}p{(\columnwidth - 10\tabcolsep) * \real{0.0860}}@{}}

\caption{\label{tbl-example-forecasts-pmf}Example PMF model output for
forecasts of incident influenza hospitalization intensity. A subset of
predictions are shown: 1-week ahead PMF forecasts made on 2022-12-17 for
Massachusetts from three distinct models. We round the forecasted
probability (in the \texttt{value} column) to two digits. The
\texttt{location}, \texttt{reference\_date} and
\texttt{target\_end\_date} columns have been omitted for brevity. This
example data is provided in the {hubExamples} package.}

\tabularnewline

\toprule\noalign{}
\begin{minipage}[b]{\linewidth}\raggedright
\texttt{model\_id}
\end{minipage} & \begin{minipage}[b]{\linewidth}\raggedright
\texttt{target}
\end{minipage} & \begin{minipage}[b]{\linewidth}\raggedleft
\texttt{horizon}
\end{minipage} & \begin{minipage}[b]{\linewidth}\raggedright
\texttt{output\_type}
\end{minipage} & \begin{minipage}[b]{\linewidth}\raggedright
\texttt{output\_type\_id}
\end{minipage} & \begin{minipage}[b]{\linewidth}\raggedleft
\texttt{value}
\end{minipage} \\
\midrule\noalign{}
\endhead
\bottomrule\noalign{}
\endlastfoot
Flusight-baseline & wk flu hosp rate category & 1 & pmf & low & 0.00 \\
Flusight-baseline & wk flu hosp rate category & 1 & pmf & moderate &
0.00 \\
Flusight-baseline & wk flu hosp rate category & 1 & pmf & high & 0.07 \\
Flusight-baseline & wk flu hosp rate category & 1 & pmf & very high &
0.92 \\
MOBS-GLEAM\_FLUH & wk flu hosp rate category & 1 & pmf & low & 0.00 \\
MOBS-GLEAM\_FLUH & wk flu hosp rate category & 1 & pmf & moderate &
0.00 \\
MOBS-GLEAM\_FLUH & wk flu hosp rate category & 1 & pmf & high & 0.16 \\
MOBS-GLEAM\_FLUH & wk flu hosp rate category & 1 & pmf & very high &
0.83 \\
PSI-DICE & wk flu hosp rate category & 1 & pmf & low & 0.01 \\
PSI-DICE & wk flu hosp rate category & 1 & pmf & moderate & 0.07 \\
PSI-DICE & wk flu hosp rate category & 1 & pmf & high & 0.22 \\
PSI-DICE & wk flu hosp rate category & 1 & pmf & very high & 0.70 \\

\end{longtable}

We show a representative example of the hospitalization intensity
category forecasts in Table~\ref{tbl-example-forecasts-pmf}. Because
these forecasts are PMF output type, the \texttt{output\_type\_id}
column specifies the bin of hospitalization intensity and the
\texttt{value} column provides the forecasted probability of
hospitalization incidence being in that category. Values sum to 1 across
bins. For the MOBS-GLEAM\_FLUH and PSI-DICE models, incidence is
forecasted to decrease over the horizon (Figure~\ref{fig-plot-ex-mods}),
and correspondingly, there is lower probability of ``high'' and ``very
high'' hospitalization intensity for later horizons
(Figure~\ref{fig-plot-ex-mods-pmf}).

\begin{figure}

\centering{

\includegraphics{hubEnsembles_manuscript_files/figure-pdf/fig-plot-ex-mods-pmf-1.pdf}

}

\caption{\label{fig-plot-ex-mods-pmf}One example PMF forecast of
incident influenza hospitalization intensity is shown for each of three
models (panels). Each cell shows the forecasted probability of a given
hospitalization intensity bin (low, moderate, high, and very high) for
each forecast horizon (0-3 weeks ahead). Darker colors indicate higher
forecasted probability.}

\end{figure}%

\newpage

\subsection{Creating ensembles with
simple\_ensemble}\label{creating-ensembles-with-simple_ensemble}

Using the default options for \texttt{simple\_ensemble()}, we can
generate an equally weighted mean ensemble for each unique combination
of values for the task ID variables, the \texttt{output\_type} and the
\texttt{output\_type\_id}. Recall that this function uses different
ensemble methods for different output types: for the quantile output
type in our example data, the resulting ensemble is a quantile average,
while for the PMF output type, the ensemble is a linear pool.

\begin{Shaded}
\begin{Highlighting}[]
\SpecialCharTok{\textgreater{}}\NormalTok{ mean\_ens }\OtherTok{\textless{}{-}}\NormalTok{ hubExamples}\SpecialCharTok{::}\NormalTok{forecast\_outputs }\SpecialCharTok{|\textgreater{}}
\SpecialCharTok{+}\NormalTok{     dplyr}\SpecialCharTok{::}\FunctionTok{filter}\NormalTok{(output\_type }\SpecialCharTok{!=} \StringTok{"sample"}\NormalTok{) }\SpecialCharTok{|\textgreater{}}
\SpecialCharTok{+}\NormalTok{     hubEnsembles}\SpecialCharTok{::}\FunctionTok{simple\_ensemble}\NormalTok{(}
\SpecialCharTok{+}       \AttributeTok{model\_id =} \StringTok{"simple{-}ensemble{-}mean"}
\SpecialCharTok{+}\NormalTok{     )}
\end{Highlighting}
\end{Shaded}

The resulting model output has the same structure as the original model
output data (Table~\ref{tbl-mean-ensemble}), with columns for model ID,
task ID variables, output type, output type ID, and value. We also use
\texttt{model\_id\ =\ "simple-ensemble-mean"} to change the name of this
ensemble in the resulting model output; if not specified, the default is
``hub-ensemble''.

\newpage

\begin{longtable}[]{@{}
  >{\raggedright\arraybackslash}p{(\columnwidth - 10\tabcolsep) * \real{0.2188}}
  >{\raggedright\arraybackslash}p{(\columnwidth - 10\tabcolsep) * \real{0.2708}}
  >{\raggedleft\arraybackslash}p{(\columnwidth - 10\tabcolsep) * \real{0.1042}}
  >{\raggedright\arraybackslash}p{(\columnwidth - 10\tabcolsep) * \real{0.1458}}
  >{\raggedright\arraybackslash}p{(\columnwidth - 10\tabcolsep) * \real{0.1771}}
  >{\raggedleft\arraybackslash}p{(\columnwidth - 10\tabcolsep) * \real{0.0833}}@{}}

\caption{\label{tbl-mean-ensemble}Mean ensemble model output. The values
in the \texttt{model\_id} column are set by the argument
\texttt{simple\_ensemble(...,\ model\_id)}. Results are generated for
all output types, but only a subset are shown: 1-week ahead forecasts
made on 2022-12-17 for Massachusetts, with only the median, 25th and
75th quantiles for the quantile output type and all bins for the PMF
output type. The \texttt{location}, \texttt{reference\_date} and
\texttt{target\_end\_date} columns have been omitted for brevity, and
the \texttt{value} column is rounded to two digits.}

\tabularnewline

\toprule\noalign{}
\begin{minipage}[b]{\linewidth}\raggedright
\texttt{model\_id}
\end{minipage} & \begin{minipage}[b]{\linewidth}\raggedright
\texttt{target}
\end{minipage} & \begin{minipage}[b]{\linewidth}\raggedleft
\texttt{horizon}
\end{minipage} & \begin{minipage}[b]{\linewidth}\raggedright
\texttt{output\_type}
\end{minipage} & \begin{minipage}[b]{\linewidth}\raggedright
\texttt{output\_type\_id}
\end{minipage} & \begin{minipage}[b]{\linewidth}\raggedleft
\texttt{value}
\end{minipage} \\
\midrule\noalign{}
\endhead
\bottomrule\noalign{}
\endlastfoot
simple-ensemble-mean & wk flu hosp rate category & 1 & pmf & high &
0.15 \\
simple-ensemble-mean & wk flu hosp rate category & 1 & pmf & low &
0.00 \\
simple-ensemble-mean & wk flu hosp rate category & 1 & pmf & moderate &
0.02 \\
simple-ensemble-mean & wk flu hosp rate category & 1 & pmf & very high &
0.82 \\
simple-ensemble-mean & wk inc flu hosp & 1 & median & NA & 619.67 \\
simple-ensemble-mean & wk inc flu hosp & 1 & quantile & 0.25 & 541.67 \\
simple-ensemble-mean & wk inc flu hosp & 1 & quantile & 0.75 & 704.33 \\

\end{longtable}

\subsubsection{Changing the aggregation
function}\label{changing-the-aggregation-function}

We can change the function that is used to aggregate model outputs. For
example, we may want to calculate a median of the component models'
submitted values for each quantile. We do so by specifying
\texttt{agg\_fun\ =\ median}.

\begin{Shaded}
\begin{Highlighting}[]
\SpecialCharTok{\textgreater{}}\NormalTok{ median\_ens }\OtherTok{\textless{}{-}}\NormalTok{ hubExamples}\SpecialCharTok{::}\NormalTok{forecast\_outputs }\SpecialCharTok{|\textgreater{}}
\SpecialCharTok{+}\NormalTok{     dplyr}\SpecialCharTok{::}\FunctionTok{filter}\NormalTok{(output\_type }\SpecialCharTok{!=} \StringTok{"sample"}\NormalTok{) }\SpecialCharTok{|\textgreater{}}
\SpecialCharTok{+}\NormalTok{     hubEnsembles}\SpecialCharTok{::}\FunctionTok{simple\_ensemble}\NormalTok{(}
\SpecialCharTok{+}       \AttributeTok{agg\_fun =}\NormalTok{ median,}
\SpecialCharTok{+}       \AttributeTok{model\_id =} \StringTok{"simple{-}ensemble{-}median"}
\SpecialCharTok{+}\NormalTok{     )}
\end{Highlighting}
\end{Shaded}

Custom functions can also be passed into the \texttt{agg\_fun} argument.
We illustrate this by defining a custom function to compute the ensemble
prediction as a geometric mean of the component model predictions. Any
custom function to be used must have an argument \texttt{x} for the
vector of numeric values to summarize, and if relevant, an argument
\texttt{w} of numeric weights.

\begin{Shaded}
\begin{Highlighting}[]
\SpecialCharTok{\textgreater{}}\NormalTok{ geometric\_mean }\OtherTok{\textless{}{-}} \ControlFlowTok{function}\NormalTok{(x) \{}
\SpecialCharTok{+}\NormalTok{     n }\OtherTok{\textless{}{-}} \FunctionTok{length}\NormalTok{(x)}
\SpecialCharTok{+}     \FunctionTok{return}\NormalTok{(}\FunctionTok{prod}\NormalTok{(x)}\SpecialCharTok{\^{}}\NormalTok{(}\DecValTok{1} \SpecialCharTok{/}\NormalTok{ n))}
\SpecialCharTok{+}\NormalTok{   \}}
\SpecialCharTok{\textgreater{}}\NormalTok{ geometric\_mean\_ens }\OtherTok{\textless{}{-}}\NormalTok{ hubExamples}\SpecialCharTok{::}\NormalTok{forecast\_outputs }\SpecialCharTok{|\textgreater{}}
\SpecialCharTok{+}\NormalTok{     dplyr}\SpecialCharTok{::}\FunctionTok{filter}\NormalTok{(output\_type }\SpecialCharTok{!=} \StringTok{"sample"}\NormalTok{) }\SpecialCharTok{|\textgreater{}}
\SpecialCharTok{+}\NormalTok{     hubEnsembles}\SpecialCharTok{::}\FunctionTok{simple\_ensemble}\NormalTok{(}
\SpecialCharTok{+}       \AttributeTok{agg\_fun =}\NormalTok{ geometric\_mean,}
\SpecialCharTok{+}       \AttributeTok{model\_id =} \StringTok{"simple{-}ensemble{-}geometric"}
\SpecialCharTok{+}\NormalTok{     )}
\end{Highlighting}
\end{Shaded}

As expected, the mean, median, and geometric mean each give us slightly
different resulting ensembles. The median point estimates, 50\%
prediction intervals, and 90\% prediction intervals in
Figure~\ref{fig-plot-ensembles} demonstrate this.

\begin{figure}

\centering{

\includegraphics{hubEnsembles_manuscript_files/figure-pdf/fig-plot-ensembles-1.pdf}

}

\caption{\label{fig-plot-ensembles}Three different ensembles for weekly
incident influenza hospitalizations in Massachusetts. Each ensemble
combines individual predictions from the example hub
(Figure~\ref{fig-plot-ex-mods}) using a different method: arithmetic
mean, geometric mean, or median. All methods correspond to variations of
the quantile average approach. Ensembles are represented by a median
(line), 50\% and 90\% prediction intervals (ribbons). Geometric mean
ensemble and simple mean ensemble generate similar estimates in this
case.}

\end{figure}%

\subsubsection{Weighting model
contributions}\label{weighting-model-contributions}

We can weight the contributions of each model in the ensemble using the
\texttt{weights} argument of \texttt{simple\_ensemble()}. This argument
takes a \texttt{data.frame} that should include a \texttt{model\_id}
column containing each unique model ID and a \texttt{weight} column. In
the following example, we include the baseline model in the ensemble,
but give it less weight than the other forecasts.

\begin{Shaded}
\begin{Highlighting}[]
\SpecialCharTok{\textgreater{}}\NormalTok{ model\_weights }\OtherTok{\textless{}{-}} \FunctionTok{data.frame}\NormalTok{(}
\SpecialCharTok{+}     \AttributeTok{model\_id =} \FunctionTok{c}\NormalTok{(}\StringTok{"MOBS{-}GLEAM\_FLUH"}\NormalTok{, }\StringTok{"PSI{-}DICE"}\NormalTok{, }\StringTok{"simple\_hub{-}baseline"}\NormalTok{),}
\SpecialCharTok{+}     \AttributeTok{weight =} \FunctionTok{c}\NormalTok{(}\FloatTok{0.4}\NormalTok{, }\FloatTok{0.4}\NormalTok{, }\FloatTok{0.2}\NormalTok{)}
\SpecialCharTok{+}\NormalTok{   )}
\SpecialCharTok{\textgreater{}}\NormalTok{ weighted\_mean\_ens }\OtherTok{\textless{}{-}}\NormalTok{ hubExamples}\SpecialCharTok{::}\NormalTok{forecast\_outputs }\SpecialCharTok{|\textgreater{}}
\SpecialCharTok{+}\NormalTok{     dplyr}\SpecialCharTok{::}\FunctionTok{filter}\NormalTok{(output\_type }\SpecialCharTok{!=} \StringTok{"sample"}\NormalTok{) }\SpecialCharTok{|\textgreater{}}
\SpecialCharTok{+}\NormalTok{     hubEnsembles}\SpecialCharTok{::}\FunctionTok{simple\_ensemble}\NormalTok{(}
\SpecialCharTok{+}       \AttributeTok{weights =}\NormalTok{ model\_weights,}
\SpecialCharTok{+}       \AttributeTok{model\_id =} \StringTok{"simple{-}ensemble{-}weighted{-}mean"}
\SpecialCharTok{+}\NormalTok{     )}
\end{Highlighting}
\end{Shaded}

\subsection{Creating ensembles with
linear\_pool}\label{creating-ensembles-with-linear_pool}

We can also generate a linear pool ensemble, or distributional mixture,
using the \texttt{linear\_pool()} function; this function can be applied
to predictions with an \texttt{output\_type} of mean, quantile, CDF, or
PMF. Our example hub includes median output type, so we exclude it from
the calculation.

\begin{Shaded}
\begin{Highlighting}[]
\SpecialCharTok{\textgreater{}}\NormalTok{ linear\_pool\_ens }\OtherTok{\textless{}{-}}\NormalTok{ hubExamples}\SpecialCharTok{::}\NormalTok{forecast\_outputs }\SpecialCharTok{|\textgreater{}}
\SpecialCharTok{+}\NormalTok{     dplyr}\SpecialCharTok{::}\FunctionTok{filter}\NormalTok{(}\SpecialCharTok{!}\NormalTok{output\_type }\SpecialCharTok{\%in\%} \FunctionTok{c}\NormalTok{(}\StringTok{"median"}\NormalTok{, }\StringTok{"sample"}\NormalTok{)) }\SpecialCharTok{|\textgreater{}}
\SpecialCharTok{+}\NormalTok{     hubEnsembles}\SpecialCharTok{::}\FunctionTok{linear\_pool}\NormalTok{(}\AttributeTok{model\_id =} \StringTok{"linear{-}pool"}\NormalTok{)}
\end{Highlighting}
\end{Shaded}

As described above, for \texttt{quantile} model outputs, the
\texttt{linear\_pool} function approximates the full probability
distribution for each component prediction using the value-quantile
pairs provided by that model, and then obtains quasi-random samples from
that distributional estimate. The number of samples drawn from the
distribution of each component model defaults to \texttt{1e4}, but this
can be changed using the \texttt{n\_samples} argument.

In Figure~\ref{fig-plot-ex-quantile-and-linear-pool}, we compare
ensemble results generated by \texttt{simple\_ensemble()} and
\texttt{linear\_pool()} for model outputs of output types PMF and
quantile. As expected, the results from the two functions are equivalent
for the PMF output type: for this output type, the
\texttt{simple\_ensemble()} method averages the predicted probability of
each category across the component models, which is the definition of
the linear pool ensemble method. This is not the case for the quantile
output type, because the \texttt{simple\_ensemble()} is computing a
quantile average.

\begin{figure}

\centering{

\includegraphics{hubEnsembles_manuscript_files/figure-pdf/fig-plot-ex-quantile-and-linear-pool-1.pdf}

}

\caption{\label{fig-plot-ex-quantile-and-linear-pool}Comparison of
results from \texttt{linear\_pool()} (blue) and
\texttt{simple\_ensemble()} (red). (Panel A) Ensemble predictions of
Massachusetts incident influenza hospitalization intensity (classified
as low, moderate, high, or very high), which provide an example of PMF
output type. (Panel B) Ensemble predictions of weekly incident influenza
hospitalizations in Massachusetts, which provide an example of quantile
output type. Note, for quantile output type, \texttt{simple\_ensemble()}
corresponds to a quantile average. Ensembles combine individual models
from the example hub, and are represented by a median (line), 50\% and
90\% prediction intervals (ribbons) (Figure~\ref{fig-plot-ex-mods}).}

\end{figure}%

\section{Example: in-depth analysis of forecast
data}\label{sec-case-study}

To further demonstrate the utility of the {hubEnsembles} package and the
differences between the two ensembling functions, we examine a more
complex example. Unlike the previous section's basic showcase of
functionality, we use this case study to provide a more complete
analysis that compares and evaluates ensemble model performance using
real forecasts collected by a modeling hub, with an overarching goal of
choosing a single best ensembling approach for the application.

Since 2013, the US Centers for Disease Control and Prevention (CDC) has
been soliciting forecasts of seasonal influenza from modeling teams
through a collaborative challenge called FluSight\textsuperscript{31}.
We use a subset of these predictions to create four equally-weighted
ensembles with \texttt{simple\_ensemble()} and \texttt{linear\_pool()}
and compare the resulting ensembles' performance. The ensembling methods
chosen for this case study consist of a quantile (arithmetic) mean, a
quantile median, a linear pool with normal tails, and a linear pool with
lognormal tails. Note that only a select portion of the code is shown in
this manuscript for brevity, although all the functions and scripts used
to generate the case study results can be found in the associated GitHub
repository
(\url{https://github.com/hubverse-org/hubEnsemblesManuscript}). More
specifically, the figures and tables supporting this analysis are
generated reproducibly using data from rds files stored in the
\texttt{analysis/data/raw-data} directory and scripts in the
\texttt{inst} directory of the repository.

\subsection{Data and Methods}\label{data-and-methods}

We begin by querying the component forecasts used to generate the four
ensembles from Zoltar\textsuperscript{32}, a repository designed to
archive forecasts created by the Reich Lab at UMass Amherst. For this
analysis we only consider FluSight predictions in a quantile format from
the 2021-2022 and 2022-2023 seasons. These forecasts were stored in two
data objects, split by season, called
\texttt{flu\_forecasts-zoltar\_21-22.rds} and
\texttt{flu\_forecasts-zoltar\_22-23.rds}, and a subset is shown below
in Table~\ref{tbl-raw-flu-forecasts}.

\begin{Shaded}
\begin{Highlighting}[]
\SpecialCharTok{\textgreater{}}\NormalTok{ flu\_forecasts\_raw\_21\_22 }\OtherTok{\textless{}{-}}\NormalTok{ readr}\SpecialCharTok{::}\FunctionTok{read\_rds}\NormalTok{(}
\SpecialCharTok{+}\NormalTok{     here}\SpecialCharTok{::}\FunctionTok{here}\NormalTok{(}\StringTok{"analysis/data/raw\_data/flu\_forecasts{-}zoltar\_21{-}22.rds"}\NormalTok{)}
\SpecialCharTok{+}\NormalTok{   )}
\SpecialCharTok{\textgreater{}}\NormalTok{ flu\_forecasts\_raw\_22\_23 }\OtherTok{\textless{}{-}}\NormalTok{ readr}\SpecialCharTok{::}\FunctionTok{read\_rds}\NormalTok{(}
\SpecialCharTok{+}\NormalTok{     here}\SpecialCharTok{::}\FunctionTok{here}\NormalTok{(}\StringTok{"analysis/data/raw\_data/flu\_forecasts{-}zoltar\_22{-}23.rds"}\NormalTok{)}
\SpecialCharTok{+}\NormalTok{   )}
\SpecialCharTok{\textgreater{}}\NormalTok{ flu\_forecasts\_raw }\OtherTok{\textless{}{-}} \FunctionTok{rbind}\NormalTok{(flu\_forecasts\_raw\_21\_22, flu\_forecasts\_raw\_22\_23)}
\end{Highlighting}
\end{Shaded}

\begin{longtable}[]{@{}
  >{\raggedright\arraybackslash}p{(\columnwidth - 16\tabcolsep) * \real{0.2095}}
  >{\raggedright\arraybackslash}p{(\columnwidth - 16\tabcolsep) * \real{0.2286}}
  >{\raggedright\arraybackslash}p{(\columnwidth - 16\tabcolsep) * \real{0.0857}}
  >{\raggedleft\arraybackslash}p{(\columnwidth - 16\tabcolsep) * \real{0.0762}}
  >{\raggedright\arraybackslash}p{(\columnwidth - 16\tabcolsep) * \real{0.0571}}
  >{\raggedright\arraybackslash}p{(\columnwidth - 16\tabcolsep) * \real{0.0667}}
  >{\raggedright\arraybackslash}p{(\columnwidth - 16\tabcolsep) * \real{0.0857}}
  >{\raggedleft\arraybackslash}p{(\columnwidth - 16\tabcolsep) * \real{0.1048}}
  >{\raggedright\arraybackslash}p{(\columnwidth - 16\tabcolsep) * \real{0.0857}}@{}}

\caption{\label{tbl-raw-flu-forecasts}An example prediction of weekly
incident influenza hospitalizations pulled directly from Zoltar. The
example forecasts were made on May 15, 2023 for California at the 1 week
ahead horizon. The forecasts were generated during the FluSight
forecasting challenge, then formatted according to Zoltar standards for
storage. The \texttt{timezero}, \texttt{season}, \texttt{unit},
\texttt{param1}, \texttt{param2}, and \texttt{param3} columns have been
omitted for brevity. (The \texttt{season} column has a value of
`2021-2022' or `2022-2023' while the last three `param' columns always
have a value of NA.)}

\tabularnewline

\toprule\noalign{}
\begin{minipage}[b]{\linewidth}\raggedright
\texttt{model}
\end{minipage} & \begin{minipage}[b]{\linewidth}\raggedright
\texttt{target}
\end{minipage} & \begin{minipage}[b]{\linewidth}\raggedright
\texttt{class}
\end{minipage} & \begin{minipage}[b]{\linewidth}\raggedleft
\texttt{value}
\end{minipage} & \begin{minipage}[b]{\linewidth}\raggedright
\texttt{cat}
\end{minipage} & \begin{minipage}[b]{\linewidth}\raggedright
\texttt{prob}
\end{minipage} & \begin{minipage}[b]{\linewidth}\raggedright
\texttt{sample}
\end{minipage} & \begin{minipage}[b]{\linewidth}\raggedleft
\texttt{quantile}
\end{minipage} & \begin{minipage}[b]{\linewidth}\raggedright
\texttt{family}
\end{minipage} \\
\midrule\noalign{}
\endhead
\bottomrule\noalign{}
\endlastfoot
UMass-trends\_ensemble & 1 wk ahead inc flu hosp & quantile & 12 & NA &
NA & NA & 0.025 & NA \\
UMass-trends\_ensemble & 1 wk ahead inc flu hosp & quantile & 17 & NA &
NA & NA & 0.100 & NA \\
UMass-trends\_ensemble & 1 wk ahead inc flu hosp & quantile & 25 & NA &
NA & NA & 0.250 & NA \\
UMass-trends\_ensemble & 1 wk ahead inc flu hosp & quantile & 46 & NA &
NA & NA & 0.750 & NA \\
UMass-trends\_ensemble & 1 wk ahead inc flu hosp & quantile & 56 & NA &
NA & NA & 0.900 & NA \\
UMass-trends\_ensemble & 1 wk ahead inc flu hosp & quantile & 68 & NA &
NA & NA & 0.975 & NA \\

\end{longtable}

Forecasts must conform to hubverse standards to be fed into either of
the ensembling functions, so we first transform the raw forecasts using
the \texttt{as\_model\_out\_tbl()}\footnote{https://hubverse-org.github.io/hubUtils/reference/as\_model\_out\_tbl.html}
function from the {hubUtils} package. Here, we specify the task ID
variables \texttt{forecast\_date} (when the forecast was made),
\texttt{location}, \texttt{horizon}, and \texttt{target}.

\begin{Shaded}
\begin{Highlighting}[]
\SpecialCharTok{\textgreater{}}\NormalTok{ flu\_forecasts\_hubverse }\OtherTok{\textless{}{-}}\NormalTok{ flu\_forecasts\_raw }\SpecialCharTok{|\textgreater{}}
\SpecialCharTok{+}\NormalTok{     dplyr}\SpecialCharTok{::}\FunctionTok{rename}\NormalTok{(}\AttributeTok{forecast\_date =}\NormalTok{ timezero, }\AttributeTok{location =}\NormalTok{ unit) }\SpecialCharTok{|\textgreater{}}
\SpecialCharTok{+}\NormalTok{     tidyr}\SpecialCharTok{::}\FunctionTok{separate}\NormalTok{(target, }\AttributeTok{sep =} \StringTok{" "}\NormalTok{, }\AttributeTok{convert =} \ConstantTok{TRUE}\NormalTok{,}
\SpecialCharTok{+}                     \AttributeTok{into =} \FunctionTok{c}\NormalTok{(}\StringTok{"horizon"}\NormalTok{, }\StringTok{"target"}\NormalTok{), }\AttributeTok{extra =} \StringTok{"merge"}\NormalTok{) }\SpecialCharTok{|\textgreater{}}
\SpecialCharTok{+}\NormalTok{     dplyr}\SpecialCharTok{::}\FunctionTok{mutate}\NormalTok{(}\AttributeTok{target\_end\_date =} 
\SpecialCharTok{+}                     \FunctionTok{ceiling\_date}\NormalTok{(forecast\_date }\SpecialCharTok{+} \FunctionTok{weeks}\NormalTok{(horizon), }\StringTok{"weeks"}\NormalTok{) }\SpecialCharTok{{-}}
\SpecialCharTok{+}                       \FunctionTok{days}\NormalTok{(}\DecValTok{1}\NormalTok{)) }\SpecialCharTok{|\textgreater{}}
\SpecialCharTok{+}     \FunctionTok{as\_model\_out\_tbl}\NormalTok{(}
\SpecialCharTok{+}       \AttributeTok{model\_id\_col =} \StringTok{"model"}\NormalTok{,}
\SpecialCharTok{+}       \AttributeTok{output\_type\_col =} \StringTok{"class"}\NormalTok{,}
\SpecialCharTok{+}       \AttributeTok{output\_type\_id\_col =} \StringTok{"quantile"}\NormalTok{,}
\SpecialCharTok{+}       \AttributeTok{value\_col =} \StringTok{"value"}\NormalTok{,}
\SpecialCharTok{+}       \AttributeTok{sep =} \StringTok{"{-}"}\NormalTok{,}
\SpecialCharTok{+}       \AttributeTok{trim\_to\_task\_ids =} \ConstantTok{FALSE}\NormalTok{,}
\SpecialCharTok{+}       \AttributeTok{hub\_con =} \ConstantTok{NULL}\NormalTok{,}
\SpecialCharTok{+}       \AttributeTok{task\_id\_cols =} 
\SpecialCharTok{+}         \FunctionTok{c}\NormalTok{(}\StringTok{"forecast\_date"}\NormalTok{, }\StringTok{"location"}\NormalTok{, }\StringTok{"horizon"}\NormalTok{, }\StringTok{"target"}\NormalTok{, target\_end\_date),}
\SpecialCharTok{+}       \AttributeTok{remove\_empty =} \ConstantTok{TRUE}
\SpecialCharTok{+}\NormalTok{     )}
\end{Highlighting}
\end{Shaded}

Prior to ensemble calculation (shown later in this section), we filter
out any predictions (defined by a unique combination of task ID
variables) that did not include all 23 quantiles specified by FluSight
(\(\theta \in \{.010, 0.025, .050, .100, ..., .900, .950, .990\}\)). The
FluSight baseline and median ensemble models generated by the FluSight
hub are also excluded from the component forecasts. We chose to remove
the baseline to match the composition of models used to create the
official FluSight ensemble.

With these inclusion criteria, the final data set of component forecasts
consists of predictions from 25 modeling teams and 42 distinct models,
53 forecast dates (one per week), 54 US locations, 4 horizons, 1 target,
and 23 quantiles. In the 2021-2022 season, 25 models made predictions
for 22 weeks spanning from late January 2022 to late June 2022, and in
the 2022-2023 season, there were 31 models making predictions for 31
weeks spanning mid-October 2022 to mid-May 2023. Fourteen of the 42
total models made forecasts for both seasons.

In both seasons, forecasts were made for the same locations (the 50 US
states, Washington DC, Puerto Rico, the Virgin Islands, and the US as a
whole), horizons (1 to 4 weeks ahead), quantiles (the 23 described
above), and target (week ahead incident flu hospitalization). The values
for the forecasts are always non-negative. In
Table~\ref{tbl-case-study-flu-forecasts}, we provide an example of these
predictions, showing select quantiles from a single model, forecast
date, horizon, and location.

\begin{longtable}[]{@{}
  >{\raggedright\arraybackslash}p{(\columnwidth - 10\tabcolsep) * \real{0.2366}}
  >{\raggedright\arraybackslash}p{(\columnwidth - 10\tabcolsep) * \real{0.2366}}
  >{\raggedleft\arraybackslash}p{(\columnwidth - 10\tabcolsep) * \real{0.1075}}
  >{\raggedright\arraybackslash}p{(\columnwidth - 10\tabcolsep) * \real{0.1505}}
  >{\raggedleft\arraybackslash}p{(\columnwidth - 10\tabcolsep) * \real{0.1828}}
  >{\raggedleft\arraybackslash}p{(\columnwidth - 10\tabcolsep) * \real{0.0860}}@{}}

\caption{\label{tbl-case-study-flu-forecasts}An example prediction of
weekly incident influenza hospitalizations. The example model output was
made on May 15, 2023 for California at the 1 week ahead horizon. The
forecast was generated during the FluSight forecasting challenge, then
formatted according to hubverse standards post hoc. The
\texttt{location}, \texttt{forecast\_date}, and \texttt{season} columns
have been omitted for brevity; quantiles representing the endpoints of
the central 50\%, 80\% and 95\% prediction intervals are shown.}

\tabularnewline

\toprule\noalign{}
\begin{minipage}[b]{\linewidth}\raggedright
\texttt{model\_id}
\end{minipage} & \begin{minipage}[b]{\linewidth}\raggedright
\texttt{target}
\end{minipage} & \begin{minipage}[b]{\linewidth}\raggedleft
\texttt{horizon}
\end{minipage} & \begin{minipage}[b]{\linewidth}\raggedright
\texttt{output\_type}
\end{minipage} & \begin{minipage}[b]{\linewidth}\raggedleft
\texttt{output\_type\_id}
\end{minipage} & \begin{minipage}[b]{\linewidth}\raggedleft
\texttt{value}
\end{minipage} \\
\midrule\noalign{}
\endhead
\bottomrule\noalign{}
\endlastfoot
UMass-trends\_ensemble & wk ahead inc flu hosp & 1 & quantile & 0.025 &
12 \\
UMass-trends\_ensemble & wk ahead inc flu hosp & 1 & quantile & 0.100 &
17 \\
UMass-trends\_ensemble & wk ahead inc flu hosp & 1 & quantile & 0.250 &
25 \\
UMass-trends\_ensemble & wk ahead inc flu hosp & 1 & quantile & 0.750 &
46 \\
UMass-trends\_ensemble & wk ahead inc flu hosp & 1 & quantile & 0.900 &
56 \\
UMass-trends\_ensemble & wk ahead inc flu hosp & 1 & quantile & 0.975 &
68 \\

\end{longtable}

Next, we combine the component model outputs to generate predictions
from each ensemble model. We begin by excluding the baseline model from
the set of predictions that will be combined. Then, we create one object
to store the ensemble results generated from each method we are
interested in comparing.

\begin{Shaded}
\begin{Highlighting}[]
\SpecialCharTok{\textgreater{}}\NormalTok{ flu\_forecasts\_component }\OtherTok{\textless{}{-}}\NormalTok{ dplyr}\SpecialCharTok{::}\FunctionTok{filter}\NormalTok{(}
\SpecialCharTok{+}\NormalTok{     flu\_forecasts\_hubverse,}
\SpecialCharTok{+}     \SpecialCharTok{!}\NormalTok{model\_id }\SpecialCharTok{\%in\%} \FunctionTok{c}\NormalTok{(}\StringTok{"Flusight{-}baseline"}\NormalTok{, }\StringTok{"Flusight{-}ensemble"}\NormalTok{)}
\SpecialCharTok{+}\NormalTok{   )}
\SpecialCharTok{\textgreater{}} 
\ErrorTok{\textgreater{}}\NormalTok{ mean\_ensemble }\OtherTok{\textless{}{-}}\NormalTok{ flu\_forecasts\_component }\SpecialCharTok{|\textgreater{}}
\SpecialCharTok{+}\NormalTok{     hubEnsembles}\SpecialCharTok{::}\FunctionTok{simple\_ensemble}\NormalTok{(}
\SpecialCharTok{+}       \AttributeTok{weights =} \ConstantTok{NULL}\NormalTok{,}
\SpecialCharTok{+}       \AttributeTok{agg\_fun =}\NormalTok{ mean,}
\SpecialCharTok{+}       \AttributeTok{model\_id =} \StringTok{"mean{-}ensemble"}
\SpecialCharTok{+}\NormalTok{     )}
\SpecialCharTok{\textgreater{}}\NormalTok{ median\_ensemble }\OtherTok{\textless{}{-}}\NormalTok{ flu\_forecasts\_component }\SpecialCharTok{|\textgreater{}}
\SpecialCharTok{+}\NormalTok{     hubEnsembles}\SpecialCharTok{::}\FunctionTok{simple\_ensemble}\NormalTok{(}
\SpecialCharTok{+}       \AttributeTok{weights =} \ConstantTok{NULL}\NormalTok{,}
\SpecialCharTok{+}       \AttributeTok{agg\_fun =}\NormalTok{ median,}
\SpecialCharTok{+}       \AttributeTok{model\_id =} \StringTok{"median{-}ensemble"}
\SpecialCharTok{+}\NormalTok{     )}
\SpecialCharTok{\textgreater{}}\NormalTok{ lp\_normal }\OtherTok{\textless{}{-}}\NormalTok{ flu\_forecasts\_component }\SpecialCharTok{|\textgreater{}}
\SpecialCharTok{+}\NormalTok{     hubEnsembles}\SpecialCharTok{::}\FunctionTok{linear\_pool}\NormalTok{(}
\SpecialCharTok{+}       \AttributeTok{weights =} \ConstantTok{NULL}\NormalTok{,}
\SpecialCharTok{+}       \AttributeTok{n\_samples =} \FloatTok{1e5}\NormalTok{,}
\SpecialCharTok{+}       \AttributeTok{model\_id =} \StringTok{"lp{-}normal"}\NormalTok{,}
\SpecialCharTok{+}       \AttributeTok{tail\_dist =} \StringTok{"norm"}
\SpecialCharTok{+}\NormalTok{     )}
\SpecialCharTok{\textgreater{}}\NormalTok{ lp\_lognormal }\OtherTok{\textless{}{-}}\NormalTok{ flu\_forecasts\_component }\SpecialCharTok{|\textgreater{}}
\SpecialCharTok{+}\NormalTok{     hubEnsembles}\SpecialCharTok{::}\FunctionTok{linear\_pool}\NormalTok{(}
\SpecialCharTok{+}       \AttributeTok{weights =} \ConstantTok{NULL}\NormalTok{,}
\SpecialCharTok{+}       \AttributeTok{n\_samples =} \FloatTok{1e5}\NormalTok{,}
\SpecialCharTok{+}       \AttributeTok{model\_id =} \StringTok{"lp{-}lognormal"}\NormalTok{,}
\SpecialCharTok{+}       \AttributeTok{tail\_dist =} \StringTok{"lnorm"}
\SpecialCharTok{+}\NormalTok{     ) }
\end{Highlighting}
\end{Shaded}

We evaluate the performance of these ensembles using scoring metrics
that measure the accuracy and calibration of their forecasts. Here, we
choose several common metrics in forecast evaluation, including mean
absolute error (MAE), weighted interval score (WIS)\textsuperscript{33},
50\% prediction interval (PI) coverage, and 95\% PI coverage. MAE
measures the average absolute error of a set of point forecasts; smaller
values of MAE indicate better forecast accuracy. WIS is a generalization
of MAE for probabilistic forecasts and is an alternative to other common
proper scoring rules which cannot be evaluated directly for quantile
forecasts\textsuperscript{33}. WIS is made up of three component
penalties: (1) for over-prediction, (2) for under-prediction, and (3)
for the spread of each interval (where an interval is defined by a
symmetric set of two quantiles). This metric weights these penalties
across all prediction intervals provided. A lower WIS value indicates a
more accurate forecast\textsuperscript{33}. PI coverage provides
information about whether a forecast has accurately characterized its
uncertainty about future observations. The \(50\)\% PI coverage rate
measures the proportion of the time that 50\% prediction intervals at
that nominal level included the observed value; the 95\% PI coverage
rate is defined similarly. Achieving approximately nominal (50\% or
95\%) coverage indicates a well-calibrated forecast.

We also use relative versions of WIS and MAE (rWIS and rMAE,
respectively) to understand how the ensemble performance compares to
that of the FluSight baseline model. These metrics are calculated as
\[\textrm{rWIS} = \frac{\textrm{WIS}_{\textrm{model }m}}{\textrm{WIS}_{\textrm{baseline}}} \hspace{3cm} \textrm{rMAE} = \frac{\textrm{MAE}_{\textrm{model }m}}{\textrm{MAE}_{\textrm{baseline}}},\]
where model \(m\) is any given model being compared against the
baseline. For both of these metrics, a value less than one indicates
better performance compared to the baseline while a value greater than
one indicates worse performance. By definition, the FluSight baseline
itself will always have a value of one for both of these metrics.

Each unique prediction from an ensemble model is scored against target
data using the \texttt{score\_forecasts()}\footnote{https://reichlab.io/covidHubUtils/reference/score\_forecasts.html}
function from the {covidHubUtils} package, as a hubverse package for
scoring and evaluation has not yet been fully implemented. This function
outputs each of the metrics described above. We use median forecasts
taken from the 0.5 quantile for the MAE evaluation.

\subsection{Performance results across
ensembles}\label{performance-results-across-ensembles}

The quantile median ensemble has the best overall performance in terms
of WIS and MAE (and the relative versions of these metrics), and has
coverage rates that were close to the nominal levels
(Table~\ref{tbl-overall-evaluation}). The two linear opinion pools have
very similar performance to each other. These methods have the
second-best performance as measured by WIS and MAE, but they have the
highest 50\% and 95\% coverage rates, with empirical coverage that was
well above the nominal coverage rate. The quantile mean performs the
worst of the ensembles with the highest MAE, which is substantially
different from that of the other ensembles.

\begin{longtable}[]{@{}lrrrrrr@{}}

\caption{\label{tbl-overall-evaluation}Summary of overall model
performance across both seasons, averaged over all locations except the
US national location and sorted by ascending WIS. The quantile median
ensemble has the best value for every metric except 50\% coverage rate,
though metric values are often quite similar among the models.}

\tabularnewline

\toprule\noalign{}
\texttt{model} & \texttt{wis} & \texttt{rwis} & \texttt{mae} &
\texttt{rmae} & \texttt{cov50} & \texttt{cov95} \\
\midrule\noalign{}
\endhead
\bottomrule\noalign{}
\endlastfoot
median-ensemble & 18.158 & 0.794 & 27.360 & 0.933 & 0.597 & 0.922 \\
lp-normal & 19.745 & 0.863 & 27.932 & 0.953 & 0.709 & 0.990 \\
lp-lognormal & 19.747 & 0.863 & 27.933 & 0.953 & 0.708 & 0.990 \\
mean-ensemble & 20.180 & 0.882 & 29.582 & 1.009 & 0.595 & 0.889 \\
Flusight-baseline & 22.876 & 1.000 & 29.315 & 1.000 & 0.604 & 0.881 \\

\end{longtable}

Plots of the models' forecasts can aid our understanding about the
origin of these accuracy differences. For example, the linear opinion
pools consistently have some of the widest prediction intervals, and
consequently the highest coverage rates. The median ensemble, which has
the best WIS, balanced interval width with calibration best overall,
with narrower intervals than the linear pools that still achieved
near-nominal coverage on average across all time points. The quantile
mean's interval widths vary, though it usually has narrower intervals
than the linear pools. However, this model's point forecasts have a
larger error margin compared to the other ensembles, especially at
longer horizons. This pattern is demonstrated in
Figure~\ref{fig-plot-forecasts-hubVis} for the 4-week ahead forecast in
California following the 2022-23 season peak on December 5, 2022. Here,
the quantile mean predicted a continued increase in hospitalizations, at
a steeper slope than the other ensemble methods.

\begin{figure}

\centering{

\includegraphics{hubEnsembles_manuscript_files/figure-pdf/fig-plot-forecasts-hubVis-1.pdf}

}

\caption{\label{fig-plot-forecasts-hubVis}One to four week ahead
forecasts for select dates plotted against target data for California.
The first panel shows all models on the same scale. All other panels
show forecasts for each individual model, with varying y-axis scales,
and their prediction accuracy as compared to observed influenza
hospitalizations.}

\end{figure}%

Averaging across all time points, the median model can be seen to have
the best scores for every metric. It outperforms the mean ensemble by a
similar amount for both MAE and WIS, particularly around local times of
change (see Figure~\ref{fig-mae-vs-forecast-date} and
Figure~\ref{fig-wis-vs-forecast-date}). The median ensemble also has
better coverage rates than the mean ensemble in the tails of the
distribution (95\% intervals, see
Figure~\ref{fig-cov95-vs-forecast-date}) and similar coverage in the
center (50\% intervals). The median model also outperforms the linear
pools for most weeks, with the greatest differences in scores being for
WIS and coverage rates (Figure~\ref{fig-wis-vs-forecast-date} and
Figure~\ref{fig-cov95-vs-forecast-date}). This seems to indicate that
the linear pools' estimates are usually too conservative, with their
wide intervals and higher-than-nominal coverage rates being penalized by
WIS. However, during the 2022-2023 season there are several localized
times when the linear pools showcased better one-week-ahead forecasts
than the median ensemble (Figure~\ref{fig-wis-vs-forecast-date}). These
localized instances are characterized by similar MAE values
(Figure~\ref{fig-wis-vs-forecast-date}) for the two methods and poor
median ensemble coverage rates
(Figure~\ref{fig-cov95-vs-forecast-date}). In these instances, the wide
intervals from the linear pools were useful in capturing the
eventually-observed hospitalizations, usually during times of rapid
change.

\begin{figure}

\centering{

\includegraphics{hubEnsembles_manuscript_files/figure-pdf/fig-mae-vs-forecast-date-1.pdf}

}

\caption{\label{fig-mae-vs-forecast-date}Mean absolute error (MAE)
averaged across all locations. Average target data across all locations
for 2021-2022 (A) and 2022-2023 (B) seasons for reference. For each
season, average MAE is shown for 1-week (C-D) and 4-week ahead (E-F)
forecasts. Results are plotted for each ensemble model (colors) across
the entire season. Lower values indicate better performance.}

\end{figure}%

\begin{figure}

\centering{

\includegraphics{hubEnsembles_manuscript_files/figure-pdf/fig-wis-vs-forecast-date-1.pdf}

}

\caption{\label{fig-wis-vs-forecast-date}Weighted interval score (WIS)
averaged across all locations. Average target data across all locations
for 2021-2022 (A) and 2022-2023 (B) seasons for reference. For each
season, average WIS is shown for 1-week (C-D) and 4-week ahead (E-F)
forecasts. Results are plotted for each ensemble model (colors) across
the entire season. Lower values indicate better performance.}

\end{figure}%

\begin{figure}

\centering{

\includegraphics{hubEnsembles_manuscript_files/figure-pdf/fig-cov95-vs-forecast-date-1.pdf}

}

\caption{\label{fig-cov95-vs-forecast-date}95\% prediction interval (PI)
coverage averaged across all locations. Average target data across all
locations for 2021-2022 (A) and 2022-2023 (B) seasons for reference. For
each season, average coverage is shown for 1-week (C-D) and 4-week ahead
(E-F) forecasts. Results are plotted for each ensemble model (colors)
across the entire season. Ideal coverage of 95\% is shown (black
horizontal line); values closer to 95\% indicate better performance.}

\end{figure}%

In this analysis, all of the ensemble variations outperform the baseline
model; yet, different ensembling methods perform best under different
circumstances. While the quantile median has the best overall results
for WIS, MAE, 50\% PI coverage, and 95\% PI coverage, other models may
perform better from week-to-week for each metric. Around the 2022-2023
season's peak in early December, the remaining four models (including
the baseline) each have instances in which they achieve the lowest WIS,
like the linear pool ensembles for the one week ahead horizon over
several weeks of this period.

The choice of an appropriate ensemble aggregation method may depend on
the forecast target, the goal of forecasting, and the behavior of the
individual models contributing to an ensemble. One case may call for
prioritizing high coverage rates while another may prioritize accurate
point forecasts. The \texttt{simple\_ensemble()} and
\texttt{linear\_pool()} functions and the ability to specify component
model weights and an aggregation function for
\texttt{simple\_ensemble()} allow users to implement a variety of
ensemble methods.

\section{Summary and discussion}\label{sec-conclusions}

Ensembles of independent models are a powerful tool to generate more
accurate and more reliable predictions of future outcomes than a single
model alone. Here, we have demonstrated how to utilize {hubEnsembles}, a
simple and flexible framework to combine individual model predictions
into an ensemble.

The {hubEnsembles} package is situated within the larger hubverse
collection of open-source software and data tools to support
collaborative modeling exercises\textsuperscript{22}. Collaborative hubs
offer many benefits, including serving as a centralized entity to guide
and elicit predictions from multiple independent
models\textsuperscript{23}. Given the increasing popularity of
multi-model ensembles and collaborative hubs, there is a clear need for
generalized data standards and software infrastructure to support these
hubs. By addressing this need, the hubverse suite of tools can reduce
duplicative efforts across existing hubs, support other communities
engaged in collaborative efforts, and enable the adoption of multi-model
approaches in new domains.

When using {hubEnsembles}, it is important to carefully choose an
ensemble method that is well suited for the situation. Although there
may not be a universal ``best'' method, matching the properties of a
given ensemble method with the features of the component models will
likely yield best results\textsuperscript{28}. Our case study on
seasonal influenza forecasts in the US demonstrates this point. The
quantile median ensemble performs best overall for a range of metrics,
including weighted interval score, mean absolute error, and prediction
interval coverage. Yet, the linear pool method, which generates an
ensemble with wider prediction intervals, demonstrates performance
advantages during periods of rapid change, when outlying component
forecasts are likely more important. Notably, all ensemble methods
outperform the baseline model. The performance improvements from
ensemble models motivate the use of a ``hub-based'' approach to
prediction for infectious diseases and in other fields.

Ongoing development of the {hubEnsembles} package and the larger suite
of hubverse tools will continue to support multi-model predictions in
new ways, including for example supporting additional types of
predictions, enabling scoring and evaluation of those predictions, and
allowing for cloud-based data storage. All such infrastructure will
ultimately provide a comprehensive suite of open-source software tools
for leveraging the power of collaborative hubs and multi-model
ensembles.

\section*{Acknowledgements}\label{acknowledgements}
\addcontentsline{toc}{section}{Acknowledgements}

The authors thank all members of the hubverse community; the broader
hubverse software infrastructure made this package possible. L.
Shandross, A. Krystalli, N. G. Reich, and E. L. Ray were supported by
the National Institutes of General Medical Sciences (R35GM119582) and
the US Centers for Disease Control and Prevention (U01IP001122 and
NU38FT000008). E. Howerton was supported by NSF RAPID awards DEB-2126278
and DEB-2220903, as well as the Eberly College of Science Barbara
McClintock Science Achievement Graduate Scholarship in Biology at the
Pennsylvania State University. L. Contamin and H. Hochheiser were
supported by NIGMS grant U24GM132013. The content is solely the
responsibility of the authors and does not necessarily represent the
official views of NIGMS, the National Institutes of Health, or CDC.

\section*{Consortium of Infectious Disease Modeling
Hubs}\label{consortium-of-infectious-disease-modeling-hubs}
\addcontentsline{toc}{section}{Consortium of Infectious Disease Modeling
Hubs}

Consortium of Infectious Disease Modeling Hubs authors include Alvaro J.
Castro Rivadeneira (University of Massachusetts Amherst), Lucie Contamin
(University of Pittsburgh), Sebastian Funk (London School of Hygiene \&
Tropical Medicine), Aaron Gerding (University of Massachusetts Amherst),
Hugo Gruson (data.org), Harry Hochheiser (University of Pittsburgh),
Emily Howerton (The Pennsylvania State University), Melissa Kerr
(University of Massachusetts Amherst), Anna Krystalli (R-RSE SMPC), Sara
L. Loo (Johns Hopkins University), Evan L. Ray (University of
Massachusetts Amherst), Nicholas G. Reich (University of Massachusetts
Amherst), Koji Sato (Johns Hopkins University), Li Shandross (University
of Massachusetts Amherst), Katharine Sherratt (London School of Hygene
and Tropical Medicine), Shaun Truelove (Johns Hopkins University),
Martha Zorn (University of Massachusetts Amherst)

\section*{References}\label{references}
\addcontentsline{toc}{section}{References}

\phantomsection\label{refs}
\begin{CSLReferences}{0}{1}
\bibitem[\citeproctext]{ref-clemen1989}
\CSLLeftMargin{1. }%
\CSLRightInline{Clemen RT. Combining forecasts: A review and annotated
bibliography. \emph{International Journal of Forecasting}.
1989;5(4):559-583.
doi:\href{https://doi.org/10.1016/0169-2070(89)90012-5}{10.1016/0169-2070(89)90012-5}}

\bibitem[\citeproctext]{ref-timmermann2006}
\CSLLeftMargin{2. }%
\CSLRightInline{Timmermann A. Chapter 4 Forecast Combinations. In: Vol
1. Elsevier; 2006:135-196.
doi:\href{https://doi.org/10.1016/S1574-0706(05)01004-9}{10.1016/S1574-0706(05)01004-9}}

\bibitem[\citeproctext]{ref-hibon2005}
\CSLLeftMargin{3. }%
\CSLRightInline{Hibon M, Evgeniou T. To combine or not to combine:
selecting among forecasts and their combinations. \emph{International
Journal of Forecasting}. 2005;21(1):15-24.
doi:\href{https://doi.org/10.1016/j.ijforecast.2004.05.002}{10.1016/j.ijforecast.2004.05.002}}

\bibitem[\citeproctext]{ref-alley2019}
\CSLLeftMargin{4. }%
\CSLRightInline{Alley RB, Emanuel KA, Zhang F. Advances in weather
prediction. \emph{Science}. 2019;363(6425):342-344.
doi:\href{https://doi.org/10.1126/science.aav7274}{10.1126/science.aav7274}}

\bibitem[\citeproctext]{ref-tebaldi2007}
\CSLLeftMargin{5. }%
\CSLRightInline{Tebaldi C, Knutti R. The use of the multi-model ensemble
in probabilistic climate projections. \emph{Philosophical Transactions:
Mathematical, Physical and Engineering Sciences}.
2007;365(1857):2053-2075.
doi:\href{https://doi.org/10.1098/rsta.2007.2076}{10.1098/rsta.2007.2076}}

\bibitem[\citeproctext]{ref-aastveit2018}
\CSLLeftMargin{6. }%
\CSLRightInline{Aastveit KA, Mitchell J, Ravazzolo F, Dijk HK van. The
Evolution of Forecast Density Combinations in Economics. \emph{Tinbergen
Institute Discussion Papers}. Published online 2018.
\url{https://hdl.handle.net/10419/185588}}

\bibitem[\citeproctext]{ref-viboud2018}
\CSLLeftMargin{7. }%
\CSLRightInline{Viboud C, Sun K, Gaffey R, et al. The RAPIDD ebola
forecasting challenge: Synthesis and lessons learnt. \emph{Epidemics}.
2018;22:13-21.
doi:\href{https://doi.org/10.1016/j.epidem.2017.08.002}{10.1016/j.epidem.2017.08.002}}

\bibitem[\citeproctext]{ref-johansson2019}
\CSLLeftMargin{8. }%
\CSLRightInline{Johansson MA, Apfeldorf KM, Dobson S, et al. An open
challenge to advance probabilistic forecasting for dengue epidemics.
\emph{Proceedings of the National Academy of Sciences}.
2019;116(48):24268-24274.
doi:\href{https://doi.org/10.1073/pnas.1909865116}{10.1073/pnas.1909865116}}

\bibitem[\citeproctext]{ref-mcgowan2019}
\CSLLeftMargin{9. }%
\CSLRightInline{McGowan CJ, Biggerstaff M, Johansson M, et al.
Collaborative efforts to forecast seasonal influenza in the United
States, 2015{\textendash}2016. \emph{Scientific Reports}. 2019;9(1):683.
doi:\href{https://doi.org/10.1038/s41598-018-36361-9}{10.1038/s41598-018-36361-9}}

\bibitem[\citeproctext]{ref-reich_accuracy_2019}
\CSLLeftMargin{10. }%
\CSLRightInline{Reich NG, McGowan CJ, Yamana TK, et al. Accuracy of
real-time multi-model ensemble forecasts for seasonal influenza in the
{U}.{S}. \emph{PLOS computational biology}. 2019;15(11):e1007486.
doi:\href{https://doi.org/10.1371/journal.pcbi.1007486}{10.1371/journal.pcbi.1007486}}

\bibitem[\citeproctext]{ref-cramer2022}
\CSLLeftMargin{11. }%
\CSLRightInline{Cramer EY, Ray EL, Lopez VK, et al. Evaluation of
individual and ensemble probabilistic forecasts of COVID-19 mortality in
the united states. \emph{Proceedings of the National Academy of
Sciences}. 2022;119(15):e2113561119.
doi:\href{https://doi.org/10.1073/pnas.2113561119}{10.1073/pnas.2113561119}}

\bibitem[\citeproctext]{ref-paireau_ensemble_2022}
\CSLLeftMargin{12. }%
\CSLRightInline{Paireau J, Andronico A, Hozé N, et al. An ensemble model
based on early predictors to forecast {COVID}-19 health care demand in
{France}. \emph{Proceedings of the National Academy of Sciences}.
2022;119(18):e2103302119.
doi:\href{https://doi.org/10.1073/pnas.2103302119}{10.1073/pnas.2103302119}}

\bibitem[\citeproctext]{ref-ray_comparing_2023}
\CSLLeftMargin{13. }%
\CSLRightInline{Ray EL, Brooks LC, Bien J, et al. Comparing trained and
untrained probabilistic ensemble forecasts of {COVID}-19 cases and
deaths in the {United} {States}. \emph{International Journal of
Forecasting}. 2023;39(3):1366-1383.
doi:\href{https://doi.org/10.1016/j.ijforecast.2022.06.005}{10.1016/j.ijforecast.2022.06.005}}

\bibitem[\citeproctext]{ref-winkler2015}
\CSLLeftMargin{14. }%
\CSLRightInline{Winkler RL. Equal Versus Differential Weighting in
Combining Forecasts. \emph{Risk Analysis}. 2015;35(1):16-18.
doi:\href{https://doi.org/10.1111/risa.12302}{10.1111/risa.12302}}

\bibitem[\citeproctext]{ref-yamana_superensemble_2016}
\CSLLeftMargin{15. }%
\CSLRightInline{Yamana TK, Kandula S, Shaman J. Superensemble forecasts
of dengue outbreaks. \emph{Journal of The Royal Society Interface}.
2016;13(123):20160410.
doi:\href{https://doi.org/10.1098/rsif.2016.0410}{10.1098/rsif.2016.0410}}

\bibitem[\citeproctext]{ref-ray_prediction_2018}
\CSLLeftMargin{16. }%
\CSLRightInline{Ray EL, Reich NG. Prediction of infectious disease
epidemics via weighted density ensembles. \emph{PLOS computational
biology}. 2018;14(2):e1005910.
doi:\href{https://doi.org/10.1371/journal.pcbi.1005910}{10.1371/journal.pcbi.1005910}}

\bibitem[\citeproctext]{ref-colon-gonzalez_probabilistic_2021}
\CSLLeftMargin{17. }%
\CSLRightInline{Colón-González FJ, Bastos LS, Hofmann B, et al.
Probabilistic seasonal dengue forecasting in {Vietnam}: {A} modelling
study using superensembles. \emph{PLOS Medicine}. 2021;18(3):e1003542.
doi:\href{https://doi.org/10.1371/journal.pmed.1003542}{10.1371/journal.pmed.1003542}}

\bibitem[\citeproctext]{ref-pedregosa_scikit-learn_2011}
\CSLLeftMargin{18. }%
\CSLRightInline{Pedregosa F, Varoquaux G, Gramfort A, et al.
Scikit-learn: {Machine} {Learning} in {Python}. \emph{Journal of Machine
Learning Research}. 2011;12(85):2825-2830.
doi:\href{https://doi.org/10.5555/1953048.2078195}{10.5555/1953048.2078195}}

\bibitem[\citeproctext]{ref-weiss2019}
\CSLLeftMargin{19. }%
\CSLRightInline{Weiss Christoph,E, Raviv E, Roetzer G. Forecast
Combinations in R using the ForecastComb Package. \emph{The R Journal}.
2019;10(2):262.
doi:\href{https://doi.org/10.32614/RJ-2018-052}{10.32614/RJ-2018-052}}

\bibitem[\citeproctext]{ref-bosse_stackr_2023}
\CSLLeftMargin{20. }%
\CSLRightInline{Bosse N, Yao Y, Abbott S, Funk S. \emph{Stackr: {Create}
{Mixture} {Models} {From} {Predictive} {Samples}}.; 2023.}

\bibitem[\citeproctext]{ref-couch_stacks_2023}
\CSLLeftMargin{21. }%
\CSLRightInline{Couch S, Kuhn M. \emph{Stacks: Tidy Model Stacking}.;
2023.}

\bibitem[\citeproctext]{ref-hubverse_docs}
\CSLLeftMargin{22. }%
\CSLRightInline{Consortium of Infectious Disease Modeling Hubs. The
hubverse: Open tools for collaborative forecasting. Published online
2024. \url{https://hubverse.io/en/latest/index.html}}

\bibitem[\citeproctext]{ref-reich2022}
\CSLLeftMargin{23. }%
\CSLRightInline{Reich NG, Lessler J, Funk S, et al. Collaborative hubs:
Making the most of predictive epidemic modeling. \emph{American Journal
of Public Health}. 2022;112(6):839-842.
doi:\href{https://doi.org/10.2105/AJPH.2022.306831}{10.2105/AJPH.2022.306831}}

\bibitem[\citeproctext]{ref-borchering_public_2023}
\CSLLeftMargin{24. }%
\CSLRightInline{Borchering RK, Healy JM, Cadwell BL, et al. Public
health impact of the {U}.{S}. {Scenario} {Modeling} {Hub}.
\emph{Epidemics}. 2023;44:100705.
doi:\href{https://doi.org/10.1016/j.epidem.2023.100705}{10.1016/j.epidem.2023.100705}}

\bibitem[\citeproctext]{ref-vincent1912}
\CSLLeftMargin{25. }%
\CSLRightInline{Vincent SB. \emph{The Function of the Vibrissae in the
Behavior of the White Rat.} PhD thesis. University of Chicago; 1912.}

\bibitem[\citeproctext]{ref-stone1961}
\CSLLeftMargin{26. }%
\CSLRightInline{Stone M. The opinion pool. \emph{The Annals of
Mathematical Statistics}. 1961;32(4):1339-1342.}

\bibitem[\citeproctext]{ref-lichtendahl2013}
\CSLLeftMargin{27. }%
\CSLRightInline{Lichtendahl KC, Grushka-Cockayne Y, Winkler RL. Is it
better to average probabilities or quantiles? \emph{Management Science}.
2013;59(7):1594-1611.
doi:\href{https://doi.org/10.1287/mnsc.1120.1667}{10.1287/mnsc.1120.1667}}

\bibitem[\citeproctext]{ref-howerton2023}
\CSLLeftMargin{28. }%
\CSLRightInline{Howerton E, Runge MC, Bogich TL, et al.
Context-dependent representation of within- and between-model
uncertainty: Aggregating probabilistic predictions in infectious disease
epidemiology. \emph{Journal of The Royal Society Interface}.
2023;20(198):20220659.
doi:\href{https://doi.org/10.1098/rsif.2022.0659}{10.1098/rsif.2022.0659}}

\bibitem[\citeproctext]{ref-distfromq}
\CSLLeftMargin{29. }%
\CSLRightInline{Ray EL, Gerding A. \emph{Distfromq: Reconstruct a
Distribution from a Collection of Quantiles}.; 2024.}

\bibitem[\citeproctext]{ref-niederreiter1992quasirandom}
\CSLLeftMargin{30. }%
\CSLRightInline{Niederreiter H. \emph{Random Number Generation and
Quasi-Monte Carlo Methods}. Society for Industrial; Applied Mathematics;
1992.}

\bibitem[\citeproctext]{ref-cdc_flusight}
\CSLLeftMargin{31. }%
\CSLRightInline{CDC. About flu forecasting. Published online 2023.
\url{https://www.cdc.gov/flu/weekly/flusight/how-flu-forecasting.htm}}

\bibitem[\citeproctext]{ref-reich_zoltar_2021}
\CSLLeftMargin{32. }%
\CSLRightInline{Reich NG, Cornell M, Ray EL, House K, Le K. The {Zoltar}
forecast archive, a tool to standardize and store interdisciplinary
prediction research. \emph{Scientific Data}. 2021;8(1):59.
doi:\href{https://doi.org/10.1038/s41597-021-00839-5}{10.1038/s41597-021-00839-5}}

\bibitem[\citeproctext]{ref-bracher_evaluating_2021}
\CSLLeftMargin{33. }%
\CSLRightInline{Bracher J, Ray EL, Gneiting T, Reich NG. Evaluating
epidemic forecasts in an interval format. \emph{PLOS Computational
Biology}. 2021;17(2):e1008618.
doi:\href{https://doi.org/10.1371/journal.pcbi.1008618}{10.1371/journal.pcbi.1008618}}

\end{CSLReferences}



\end{document}
