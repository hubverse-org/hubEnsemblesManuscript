% Options for packages loaded elsewhere
\PassOptionsToPackage{unicode}{hyperref}
\PassOptionsToPackage{hyphens}{url}
\PassOptionsToPackage{dvipsnames,svgnames,x11names}{xcolor}
%
\documentclass[
]{article}

\usepackage{amsmath,amssymb}
\usepackage{iftex}
\ifPDFTeX
  \usepackage[T1]{fontenc}
  \usepackage[utf8]{inputenc}
  \usepackage{textcomp} % provide euro and other symbols
\else % if luatex or xetex
  \usepackage{unicode-math}
  \defaultfontfeatures{Scale=MatchLowercase}
  \defaultfontfeatures[\rmfamily]{Ligatures=TeX,Scale=1}
\fi
\usepackage{lmodern}
\ifPDFTeX\else  
    % xetex/luatex font selection
\fi
% Use upquote if available, for straight quotes in verbatim environments
\IfFileExists{upquote.sty}{\usepackage{upquote}}{}
\IfFileExists{microtype.sty}{% use microtype if available
  \usepackage[]{microtype}
  \UseMicrotypeSet[protrusion]{basicmath} % disable protrusion for tt fonts
}{}
\makeatletter
\@ifundefined{KOMAClassName}{% if non-KOMA class
  \IfFileExists{parskip.sty}{%
    \usepackage{parskip}
  }{% else
    \setlength{\parindent}{0pt}
    \setlength{\parskip}{6pt plus 2pt minus 1pt}}
}{% if KOMA class
  \KOMAoptions{parskip=half}}
\makeatother
\usepackage{xcolor}
\usepackage[lmargin=1 in,rmargin=1 in,tmargin=1 in,bmargin=1
in]{geometry}
\setlength{\emergencystretch}{3em} % prevent overfull lines
\setcounter{secnumdepth}{-\maxdimen} % remove section numbering
% Make \paragraph and \subparagraph free-standing
\ifx\paragraph\undefined\else
  \let\oldparagraph\paragraph
  \renewcommand{\paragraph}[1]{\oldparagraph{#1}\mbox{}}
\fi
\ifx\subparagraph\undefined\else
  \let\oldsubparagraph\subparagraph
  \renewcommand{\subparagraph}[1]{\oldsubparagraph{#1}\mbox{}}
\fi

\usepackage{color}
\usepackage{fancyvrb}
\newcommand{\VerbBar}{|}
\newcommand{\VERB}{\Verb[commandchars=\\\{\}]}
\DefineVerbatimEnvironment{Highlighting}{Verbatim}{commandchars=\\\{\}}
% Add ',fontsize=\small' for more characters per line
\usepackage{framed}
\definecolor{shadecolor}{RGB}{241,243,245}
\newenvironment{Shaded}{\begin{snugshade}}{\end{snugshade}}
\newcommand{\AlertTok}[1]{\textcolor[rgb]{0.68,0.00,0.00}{#1}}
\newcommand{\AnnotationTok}[1]{\textcolor[rgb]{0.37,0.37,0.37}{#1}}
\newcommand{\AttributeTok}[1]{\textcolor[rgb]{0.40,0.45,0.13}{#1}}
\newcommand{\BaseNTok}[1]{\textcolor[rgb]{0.68,0.00,0.00}{#1}}
\newcommand{\BuiltInTok}[1]{\textcolor[rgb]{0.00,0.23,0.31}{#1}}
\newcommand{\CharTok}[1]{\textcolor[rgb]{0.13,0.47,0.30}{#1}}
\newcommand{\CommentTok}[1]{\textcolor[rgb]{0.37,0.37,0.37}{#1}}
\newcommand{\CommentVarTok}[1]{\textcolor[rgb]{0.37,0.37,0.37}{\textit{#1}}}
\newcommand{\ConstantTok}[1]{\textcolor[rgb]{0.56,0.35,0.01}{#1}}
\newcommand{\ControlFlowTok}[1]{\textcolor[rgb]{0.00,0.23,0.31}{#1}}
\newcommand{\DataTypeTok}[1]{\textcolor[rgb]{0.68,0.00,0.00}{#1}}
\newcommand{\DecValTok}[1]{\textcolor[rgb]{0.68,0.00,0.00}{#1}}
\newcommand{\DocumentationTok}[1]{\textcolor[rgb]{0.37,0.37,0.37}{\textit{#1}}}
\newcommand{\ErrorTok}[1]{\textcolor[rgb]{0.68,0.00,0.00}{#1}}
\newcommand{\ExtensionTok}[1]{\textcolor[rgb]{0.00,0.23,0.31}{#1}}
\newcommand{\FloatTok}[1]{\textcolor[rgb]{0.68,0.00,0.00}{#1}}
\newcommand{\FunctionTok}[1]{\textcolor[rgb]{0.28,0.35,0.67}{#1}}
\newcommand{\ImportTok}[1]{\textcolor[rgb]{0.00,0.46,0.62}{#1}}
\newcommand{\InformationTok}[1]{\textcolor[rgb]{0.37,0.37,0.37}{#1}}
\newcommand{\KeywordTok}[1]{\textcolor[rgb]{0.00,0.23,0.31}{#1}}
\newcommand{\NormalTok}[1]{\textcolor[rgb]{0.00,0.23,0.31}{#1}}
\newcommand{\OperatorTok}[1]{\textcolor[rgb]{0.37,0.37,0.37}{#1}}
\newcommand{\OtherTok}[1]{\textcolor[rgb]{0.00,0.23,0.31}{#1}}
\newcommand{\PreprocessorTok}[1]{\textcolor[rgb]{0.68,0.00,0.00}{#1}}
\newcommand{\RegionMarkerTok}[1]{\textcolor[rgb]{0.00,0.23,0.31}{#1}}
\newcommand{\SpecialCharTok}[1]{\textcolor[rgb]{0.37,0.37,0.37}{#1}}
\newcommand{\SpecialStringTok}[1]{\textcolor[rgb]{0.13,0.47,0.30}{#1}}
\newcommand{\StringTok}[1]{\textcolor[rgb]{0.13,0.47,0.30}{#1}}
\newcommand{\VariableTok}[1]{\textcolor[rgb]{0.07,0.07,0.07}{#1}}
\newcommand{\VerbatimStringTok}[1]{\textcolor[rgb]{0.13,0.47,0.30}{#1}}
\newcommand{\WarningTok}[1]{\textcolor[rgb]{0.37,0.37,0.37}{\textit{#1}}}

\providecommand{\tightlist}{%
  \setlength{\itemsep}{0pt}\setlength{\parskip}{0pt}}\usepackage{longtable,booktabs,array}
\usepackage{calc} % for calculating minipage widths
% Correct order of tables after \paragraph or \subparagraph
\usepackage{etoolbox}
\makeatletter
\patchcmd\longtable{\par}{\if@noskipsec\mbox{}\fi\par}{}{}
\makeatother
% Allow footnotes in longtable head/foot
\IfFileExists{footnotehyper.sty}{\usepackage{footnotehyper}}{\usepackage{footnote}}
\makesavenoteenv{longtable}
\usepackage{graphicx}
\makeatletter
\def\maxwidth{\ifdim\Gin@nat@width>\linewidth\linewidth\else\Gin@nat@width\fi}
\def\maxheight{\ifdim\Gin@nat@height>\textheight\textheight\else\Gin@nat@height\fi}
\makeatother
% Scale images if necessary, so that they will not overflow the page
% margins by default, and it is still possible to overwrite the defaults
% using explicit options in \includegraphics[width, height, ...]{}
\setkeys{Gin}{width=\maxwidth,height=\maxheight,keepaspectratio}
% Set default figure placement to htbp
\makeatletter
\def\fps@figure{htbp}
\makeatother
% definitions for citeproc citations
\NewDocumentCommand\citeproctext{}{}
\NewDocumentCommand\citeproc{mm}{%
  \begingroup\def\citeproctext{#2}\cite{#1}\endgroup}
\makeatletter
 % allow citations to break across lines
 \let\@cite@ofmt\@firstofone
 % avoid brackets around text for \cite:
 \def\@biblabel#1{}
 \def\@cite#1#2{{#1\if@tempswa , #2\fi}}
\makeatother
\newlength{\cslhangindent}
\setlength{\cslhangindent}{1.5em}
\newlength{\csllabelwidth}
\setlength{\csllabelwidth}{3em}
\newenvironment{CSLReferences}[2] % #1 hanging-indent, #2 entry-spacing
 {\begin{list}{}{%
  \setlength{\itemindent}{0pt}
  \setlength{\leftmargin}{0pt}
  \setlength{\parsep}{0pt}
  % turn on hanging indent if param 1 is 1
  \ifodd #1
   \setlength{\leftmargin}{\cslhangindent}
   \setlength{\itemindent}{-1\cslhangindent}
  \fi
  % set entry spacing
  \setlength{\itemsep}{#2\baselineskip}}}
 {\end{list}}
\usepackage{calc}
\newcommand{\CSLBlock}[1]{\hfill\break\parbox[t]{\linewidth}{\strut\ignorespaces#1\strut}}
\newcommand{\CSLLeftMargin}[1]{\parbox[t]{\csllabelwidth}{\strut#1\strut}}
\newcommand{\CSLRightInline}[1]{\parbox[t]{\linewidth - \csllabelwidth}{\strut#1\strut}}
\newcommand{\CSLIndent}[1]{\hspace{\cslhangindent}#1}

\usepackage[noblocks]{authblk}
\renewcommand*{\Authsep}{, }
\renewcommand*{\Authand}{, }
\renewcommand*{\Authands}{, }
\renewcommand\Affilfont{\small}
\makeatletter
\@ifpackageloaded{caption}{}{\usepackage{caption}}
\AtBeginDocument{%
\ifdefined\contentsname
  \renewcommand*\contentsname{Table of contents}
\else
  \newcommand\contentsname{Table of contents}
\fi
\ifdefined\listfigurename
  \renewcommand*\listfigurename{List of Figures}
\else
  \newcommand\listfigurename{List of Figures}
\fi
\ifdefined\listtablename
  \renewcommand*\listtablename{List of Tables}
\else
  \newcommand\listtablename{List of Tables}
\fi
\ifdefined\figurename
  \renewcommand*\figurename{Figure}
\else
  \newcommand\figurename{Figure}
\fi
\ifdefined\tablename
  \renewcommand*\tablename{Table}
\else
  \newcommand\tablename{Table}
\fi
}
\@ifpackageloaded{float}{}{\usepackage{float}}
\floatstyle{ruled}
\@ifundefined{c@chapter}{\newfloat{codelisting}{h}{lop}}{\newfloat{codelisting}{h}{lop}[chapter]}
\floatname{codelisting}{Listing}
\newcommand*\listoflistings{\listof{codelisting}{List of Listings}}
\makeatother
\makeatletter
\makeatother
\makeatletter
\@ifpackageloaded{caption}{}{\usepackage{caption}}
\@ifpackageloaded{subcaption}{}{\usepackage{subcaption}}
\makeatother
\ifLuaTeX
  \usepackage{selnolig}  % disable illegal ligatures
\fi
\usepackage{bookmark}

\IfFileExists{xurl.sty}{\usepackage{xurl}}{} % add URL line breaks if available
\urlstyle{same} % disable monospaced font for URLs
\hypersetup{
  pdftitle={hubEnsembles: Ensembling Methods in R},
  pdfauthor={Li Shandross; Emily Howerton; Lucie Contamin; Harry Hochheiser; Anna Krystalli; Consortium of Infectious Disease Modeling Hubs; Nicholas G. Reich; Evan L. Ray},
  pdfkeywords={multiple models; aggregation; forecast; prediction},
  colorlinks=true,
  linkcolor={blue},
  filecolor={Maroon},
  citecolor={Blue},
  urlcolor={Blue},
  pdfcreator={LaTeX via pandoc}}

\title{\texttt{hubEnsembles}: Ensembling Methods in R}


\author[1]{Li Shandross}
\author[2]{Emily Howerton}
\author[3]{Lucie Contamin}
\author[3]{Harry Hochheiser}
\author[4]{Anna Krystalli}
\author[5]{Consortium of Infectious Disease Modeling Hubs}
\author[1]{Nicholas G. Reich}
\author[1]{Evan L. Ray}

\affil[1]{University of Massachusetts Amherst}
\affil[2]{Pennsylvania State University}
\affil[3]{University of Pittsburgh}
\affil[4]{R-RSE SMPC}
\affil[5]{Alvaro J. Castro Rivadeneira, Sebastian Funk, Aaron Gerding,
Hugo Gruson, Melissa Kerr, Sara L. Loo, Koji Sato, Katharine Sherratt,
Shaun Truelove, Martha Zorn}

\begin{document}
\maketitle
\begin{abstract}
Combining predictions from multiple models into an ensemble is a widely
used practice across many fields with demonstrated performance benefits.
The R package \texttt{hubEnsembles} provides a flexible framework for
ensembling various types of predictions, including point estimates and
probabilistic predictions. A range of common methods for generating
ensembles are supported, including weighted averages, quantile averages,
and linear pools. The \texttt{hubEnsembles} package fits within a
broader framework of open-source software and data tools called the
``hubverse'', which facilitates the development and management of
collaborative modelling exercises.
\end{abstract}

Keywords: multiple models; aggregation; forecast; prediction

\section{Introduction}\label{introduction}

Predictions of future outcomes are essential to planning and decision
making, yet generating reliable predictions of the future is
challenging. One method for overcoming this challenge is combining
predictions across multiple, independent models. These combination
methods (also called aggregation or ensembling) have been repeatedly
shown to produce predictions that are more accurate (Clemen 1989;
Timmermann 2006) and more consistent (Hibon and Evgeniou 2005) than
individual models. Because of the clear performance benefits,
multi-model ensembles are commonplace across fields, including weather
(Alley, Emanuel, and Zhang 2019), climate (Tebaldi and Knutti 2007), and
economics (Aastveit et al. 2018). More recently, multi-model ensembles
have been used to improve predictions of infectious disease outbreaks
(Viboud et al. 2018; Johansson et al. 2019; McGowan et al. 2019; Reich
et al. 2019; Cramer et al. 2022).

In the rapidly growing field of outbreak forecasting, there are many
proposed methods for generating ensembles. Generally, these methods
differ in at least one of two ways: (1) the function used to combine or
``average'' predictions, and (2) how predictions are weighted when
performing the combination. No one method is universally ``the best''; a
simple average of predictions works surprisingly well across a range of
settings (McGowan et al. 2019; Paireau et al. 2022; Ray et al. 2023) for
established theoretical reasons (Winkler 2015). However, more complex
approaches have also been shown to have benefits in some settings
(Yamana, Kandula, and Shaman 2016; Ray and Reich 2018; Reich et al.
2019; Colón-González et al. 2021). Here, we present the
\texttt{hubEnsembles} package, which provides a flexible framework for
generating ensemble predictions from multiple models. Complementing
other software for combining predictions from multiple models (e.g.,
(Pedregosa et al. 2011; Weiss, Raviv, and Roetzer 2019; Bosse et al.
2023; Couch and Kuhn 2023)), \texttt{hubEnsembles} supports multiple
types of predictions, including point estimates and different kinds of
probabilistic predictions. Throughout, we will use the term
``prediction'' to refer to any kind of model output that may be combined
including a forecast, a scenario projection, or a parameter estimate.

The \texttt{hubEnsembles} package is part of the ``hubverse'' collection
of open-source software and data tools. The hubverse project facilitates
the development and management of collaborative modelling exercises
(hubverse 2022). The broader hubverse initiative is motivated by the
demonstrated benefits of collaborative hubs (Reich et al. 2022;
Borchering et al. 2023), including performance benefits of multi-model
ensembles and the desire for standardization across such hubs. In this
paper, we focus specifically on the functionality encompassed in
\texttt{hubEnsembles}. We provide an overview of the methods
implemented, including mathematical definitions and properties
(Section~\ref{sec-defs}) as well as implementation details
(Section~\ref{sec-implementation}); we give simple examples to
demonstrate the functionality (Section~\ref{sec-simple-ex}) and a more
complex case study (Section~\ref{sec-flu}) that motivates a discussion
and comparison of the various methods (Section~\ref{sec-conclusions}).

\section{Mathematical definitions and properties of ensemble
methods}\label{sec-defs}

The \texttt{hubEnsembles} package supports both point predictions and
probabilistic predictions of different formats. A point prediction gives
a single estimate of a future outcome while a probabilistic prediction
provides an estimated probability distribution over a set of future
outcomes. We use \(N\) to denote the total number of individual
predictions that the ensemble will combine. For example, these
predictions will often be produced by different statistical or
mathematical models, and \(N\) is the total number of models that have
provided predictions. Individual predictions will be indexed by the
subscript \(i\). Optionally, the package allows for calculating
ensembles that use a weight \(w_i\) for each prediction; we define the
set of model-specific weights as
\(\pmb{w} = \{w_i | i \in 1, ..., N\}\). Informally, predictions with a
larger weight have a greater influence on the ensemble prediction,
though the details of this depend on the ensemble method (described
further below).

For a set of \(N\) point predictions,
\(\pmb{p} = \{p_i|i \in 1, ..., N\}\), each from a distinct model \(i\),
the \texttt{hubEnsembles} package can compute an ensemble of these
predictions

\[
p_E = C(\pmb{p}, \pmb{w}) 
\]

using any function \(C\) and a any set of model-specific weights
\(\pmb{w}\). For example, an arithmetic average of predictions yields
\(p_E = \sum_{i=1}^Np_iw_i\), where the weights are non-negative and sum
to 1. If \(w_i = 1/N\) for all \(i\), all predictions will be equally
weighted. This framework can also support more complex functions for
aggregation, such as a (weighted) median or geometric mean.

For probabilistic predictions, there are two commonly used classes of
methods to average or ensemble multiple predictions: quantile averaging
(also called a Vincent average (Vincent 1912)) and probability averaging
(also called a distributional mixture or linear opinion pool (Stone
1961)) (Lichtendahl, Grushka-Cockayne, and Winkler 2013). To define
these two classes of methods, let \(F(x)\) be a cumulative density
function (CDF) defined over values \(x\) of the target variable for the
prediction, and \(F^{-1}(\theta)\) be the corresponding quantile
function defined over quantile levels \(\theta \in [0, 1]\). Throughout
this article, we may refer to \(x\) as either a `value of the target
variable' or a `quantile' depending on the context, and similarly we may
refer to \(\theta\) as either a `quantile level' or a `(cumulative)
probability'. Additionally, we will use \(f(x)\) to denote a probability
mass function (PMF) for a prediction of a discrete variable or a
discretization (such as binned values) of a continuous variable.

The quantile average combines a set of quantile functions,
\(\mathcal{Q} = \{F_i^{-1}(\theta)| i \in 1,...,N \}\), with a given set
of weights, \(\pmb{w}\), as \[
F^{-1}_Q(\theta) = C_Q(\mathcal{Q}, \pmb{w}) = \sum_{i = 1}^Nw_iF^{-1}_i(\theta).
\]This computes the average value of predictions across different models
for each fixed quantile level \(\theta\). It is also possible to use
other combination functions, such as a weighted median, to combine
quantile predictions.

The probability average or linear pool is calculated by averaging
probabilities across predictions for a fixed value of the target
variable, \(x\). In other words, for a set of CDFs,
\(\mathcal{F} = \{F_i(x)| i \in 1,...,N \}\) and weights, \(\pmb{w}\),
the linear pool is calculated as

\[
F_{LOP}(x) = C_{LOP}(\mathcal{F}, \pmb{w}) = \sum_{i = 1}^Nw_iF_i(x). 
\]

For a set of PMFs, \(\{f_i|i \in 1, ..., N\}\), the linear pool can be
equivalently calculated: \(f_{LOP}(x) = \sum_{i = 1}^N w_i f_i(x)\).

The different averaging methods for probabilistic predictions yield
different properties of the resulting ensemble distribution. For
example, the variance of the linear pool is
\(\sigma^2_{LOP} = \sum_{i=1}^Nw_i\sigma_i^2 + \sum_{i=1}^Nw_i(\mu_i-\mu_{LOP})^2\),
where \(\mu_i\) is the mean and \(\sigma^2_i\) is the variance of
individual prediction \(i\), and although there is no closed-form
variance for the quantile average, the variance of the quantile average
will always be less than or equal to that of the linear pool
(Lichtendahl, Grushka-Cockayne, and Winkler 2013). Both methods generate
distributions with the same mean,
\(\mu_Q = \mu_{LOP} = \sum_{i=1}^Nw_i\mu_i\), which is the mean of
individual model means (Lichtendahl, Grushka-Cockayne, and Winkler
2013). The linear pool method preserves variation between individual
models, whereas the quantile average cancels away this variation under
the assumption it constitutes sampling error (Howerton et al. 2023).

\section{Model implementation details}\label{sec-implementation}

To understand how these methods are implemented in
\texttt{hubEnsembles}, we first must define the conventions employed by
the hubverse and its packages for representing and working with model
predictions. We begin with a short overview of concepts and conventions
needed to utilize the \texttt{hubEnsembles} package, then explain the
implementation of the two ensembling functions provided by the package,
\texttt{simple\_ensemble} and \texttt{linear\_pool}.

\subsection{Hubverse terminology and
conventions}\label{hubverse-terminology-and-conventions}

A central concept in the hubverse effort is ``model output''. Model
output is a specially formatted tabular representation of predictions.
Each row represents a single, unique prediction with each column
providing information about what is being predicted, its scope, and its
value. Per hubverse convention, each column serves one of three
purposes: denote which model has produced the prediction (called the
``model ID''), provide details about what is being predicted (called the
``task IDs''), or specify how the prediction is represented (called the
``model output representation'') (hubverse 2022).

Predictions are assumed to be generated by distinct models, typically
developed and run by a modeling team of one or more individuals. Each
model should have a unique identifier that is stored in the
\texttt{model\_id} column. Then, the details of the outcome being
predicted can be stored in a series of task ID columns. These task ID
columns may also include additional information, such as any conditions
or assumptions that were used to generate the predictions (hubverse
2022). For example, short-term forecasts of incident influenza
hospitalizations in the US at different locations and amounts of time in
the future might represent this information using a \texttt{target}
column with the value ``wk ahead inc flu hosp'', a \texttt{location}
column identifying the location being predicted, a
\texttt{reference\_date} column with the ``starting point'' of the
forecasts, and a \texttt{horizon} column with the number of steps ahead
that the forecast is predicting relative to the \texttt{reference\_date}
(Table~\ref{tbl-flu-forecasts}). All these variables make up the task ID
columns.

\begin{longtable}[]{@{}
  >{\raggedright\arraybackslash}p{(\columnwidth - 12\tabcolsep) * \real{0.2200}}
  >{\raggedright\arraybackslash}p{(\columnwidth - 12\tabcolsep) * \real{0.2200}}
  >{\raggedright\arraybackslash}p{(\columnwidth - 12\tabcolsep) * \real{0.1500}}
  >{\raggedleft\arraybackslash}p{(\columnwidth - 12\tabcolsep) * \real{0.0800}}
  >{\raggedright\arraybackslash}p{(\columnwidth - 12\tabcolsep) * \real{0.1200}}
  >{\raggedleft\arraybackslash}p{(\columnwidth - 12\tabcolsep) * \real{0.1500}}
  >{\raggedleft\arraybackslash}p{(\columnwidth - 12\tabcolsep) * \real{0.0600}}@{}}

\caption{\label{tbl-flu-forecasts}Example of forecasts for weekly
incident flu hospitalizations, formatted according to hubverse
standards. Quantile forecasts for the median and 50\%, 80\%, and 98\%
prediction intervals are shown from a single model. The
\texttt{location} column has been omitted for brevity; all forecasts in
this example are for the US.}

\tabularnewline

\toprule\noalign{}
\begin{minipage}[b]{\linewidth}\raggedright
model\_id
\end{minipage} & \begin{minipage}[b]{\linewidth}\raggedright
target
\end{minipage} & \begin{minipage}[b]{\linewidth}\raggedright
reference\_date
\end{minipage} & \begin{minipage}[b]{\linewidth}\raggedleft
horizon
\end{minipage} & \begin{minipage}[b]{\linewidth}\raggedright
output\_type
\end{minipage} & \begin{minipage}[b]{\linewidth}\raggedleft
output\_type\_id
\end{minipage} & \begin{minipage}[b]{\linewidth}\raggedleft
value
\end{minipage} \\
\midrule\noalign{}
\endhead
\bottomrule\noalign{}
\endlastfoot
UMass-trends\_ensemble & wk ahead inc flu hosp & 2023-05-15 & 1 &
quantile & 0.01 & 766 \\
UMass-trends\_ensemble & wk ahead inc flu hosp & 2023-05-15 & 1 &
quantile & 0.10 & 850 \\
UMass-trends\_ensemble & wk ahead inc flu hosp & 2023-05-15 & 1 &
quantile & 0.25 & 913 \\
UMass-trends\_ensemble & wk ahead inc flu hosp & 2023-05-15 & 1 &
quantile & 0.50 & 1010 \\
UMass-trends\_ensemble & wk ahead inc flu hosp & 2023-05-15 & 1 &
quantile & 0.75 & 1077 \\
UMass-trends\_ensemble & wk ahead inc flu hosp & 2023-05-15 & 1 &
quantile & 0.90 & 1153 \\
UMass-trends\_ensemble & wk ahead inc flu hosp & 2023-05-15 & 1 &
quantile & 0.99 & 1281 \\

\end{longtable}

Alternatively, longer-term scenario projections may require different
task ID columns. For example, projections of cumulative COVID-19 deaths
in the US at different locations, amounts of time in the future, and
under different assumed conditions may use a \texttt{target} column of
``cum death'', a \texttt{location} column specifying the location being
predicted, an \texttt{origin\_date} column specifying the date on which
the projections were made, a \texttt{horizon} column describing the
number of steps ahead that the projection is predicting relative to the
\texttt{origin\_date}, and a \texttt{scenario\_id} column denoting the
future conditions that were modeled and are projected to result in the
specified number of cumulative deaths
(Table~\ref{tbl-example-scenarios}). Different modeling efforts may use
different sets of task ID columns and values to specify their prediction
goals. Additional examples of task ID variables are available on the
hubverse documentation website (hubverse 2022).

\begin{longtable}[]{@{}
  >{\raggedright\arraybackslash}p{(\columnwidth - 14\tabcolsep) * \real{0.1684}}
  >{\raggedright\arraybackslash}p{(\columnwidth - 14\tabcolsep) * \real{0.1053}}
  >{\raggedright\arraybackslash}p{(\columnwidth - 14\tabcolsep) * \real{0.1263}}
  >{\raggedleft\arraybackslash}p{(\columnwidth - 14\tabcolsep) * \real{0.0842}}
  >{\raggedright\arraybackslash}p{(\columnwidth - 14\tabcolsep) * \real{0.1368}}
  >{\raggedright\arraybackslash}p{(\columnwidth - 14\tabcolsep) * \real{0.1263}}
  >{\raggedleft\arraybackslash}p{(\columnwidth - 14\tabcolsep) * \real{0.1579}}
  >{\raggedleft\arraybackslash}p{(\columnwidth - 14\tabcolsep) * \real{0.0947}}@{}}

\caption{\label{tbl-example-scenarios}Example of scenario projections
for cumulative COVID-19 deaths, formatted according to hubverse
standards. Quantile predictions for the median and 50\% prediction
intervals from a single model are shown for four distinct scenarios. The
\texttt{location} column has been omitted for brevity; all forecasts in
this example are for the US. This example is a subset of the
\texttt{example-complex-scenario-hub} data provided by the hubverse
(hubverse 2022).}

\tabularnewline

\toprule\noalign{}
\begin{minipage}[b]{\linewidth}\raggedright
model\_id
\end{minipage} & \begin{minipage}[b]{\linewidth}\raggedright
target
\end{minipage} & \begin{minipage}[b]{\linewidth}\raggedright
origin\_date
\end{minipage} & \begin{minipage}[b]{\linewidth}\raggedleft
horizon
\end{minipage} & \begin{minipage}[b]{\linewidth}\raggedright
scenario\_id
\end{minipage} & \begin{minipage}[b]{\linewidth}\raggedright
output\_type
\end{minipage} & \begin{minipage}[b]{\linewidth}\raggedleft
output\_type\_id
\end{minipage} & \begin{minipage}[b]{\linewidth}\raggedleft
value
\end{minipage} \\
\midrule\noalign{}
\endhead
\bottomrule\noalign{}
\endlastfoot
HUBuni-simexamp & cum death & 2021-03-07 & 26 & A-2021-03-05 & quantile
& 0.25 & 571104.0 \\
HUBuni-simexamp & cum death & 2021-03-07 & 26 & A-2021-03-05 & quantile
& 0.50 & 574478.0 \\
HUBuni-simexamp & cum death & 2021-03-07 & 26 & A-2021-03-05 & quantile
& 0.75 & 578958.0 \\
HUBuni-simexamp & cum death & 2021-03-07 & 26 & B-2021-03-05 & quantile
& 0.25 & 579960.2 \\
HUBuni-simexamp & cum death & 2021-03-07 & 26 & B-2021-03-05 & quantile
& 0.50 & 584210.0 \\
HUBuni-simexamp & cum death & 2021-03-07 & 26 & B-2021-03-05 & quantile
& 0.75 & 588643.5 \\
HUBuni-simexamp & cum death & 2021-03-07 & 26 & C-2021-03-05 & quantile
& 0.25 & 630991.8 \\
HUBuni-simexamp & cum death & 2021-03-07 & 26 & C-2021-03-05 & quantile
& 0.50 & 640293.0 \\
HUBuni-simexamp & cum death & 2021-03-07 & 26 & C-2021-03-05 & quantile
& 0.75 & 650513.2 \\
HUBuni-simexamp & cum death & 2021-03-07 & 26 & D-2021-03-05 & quantile
& 0.25 & 675736.0 \\
HUBuni-simexamp & cum death & 2021-03-07 & 26 & D-2021-03-05 & quantile
& 0.50 & 688469.0 \\
HUBuni-simexamp & cum death & 2021-03-07 & 26 & D-2021-03-05 & quantile
& 0.75 & 700861.5 \\

\end{longtable}

The third purpose of a model output column is to specify the models
prediction and details about how it is represented. This ``model output
representation'' includes the predicted values along with metadata that
specifies how the predictions are conveyed and always consists of the
same three columns: (1) \texttt{output\_type}, (2)
\texttt{output\_type\_id}, and (3) \texttt{value}. The
\texttt{output\_type} column defines how the prediction is represented
and may be one of \texttt{"mean"} or \texttt{"median"} (point
prediction), \texttt{"quantile"}, \texttt{"cdf"}, \texttt{"pmf"}
(probabilistic prediction), or \texttt{"sample"} (although this output
type is not yet supported by the \texttt{hubEnsembles} package). The
\texttt{output\_type\_id} provides additional identifying information
for a prediction and is specific to the particular \texttt{output\_type}
(see Table~\ref{tbl-flu-forecasts}). For quantile predictions, the
\texttt{output\_type\_id} is a numeric value between 0 and 1 specifying
the cumulative probability associated with the quantile prediction. In
the notation we defined above, the \texttt{output\_type\_id} corresponds
to \(\theta\) and the \texttt{value} is the quantile prediction
\(F^{-1}(\theta)\). For CDF or PMF predictions, the
\texttt{output\_type\_id} is the target variable value \(x\) at which
the cumulative distribution function or probability mass function for
the predictive distribution should be evaluated, and the \texttt{value}
column contains the predicted \(F(x)\) or \(f(x)\), respectively.
Requirements for the values of the \texttt{output\_type\_id} and
\texttt{value} columns associated with each valid output type are
summarized in Table~\ref{tbl-model-output-rep}.

This representation of predictive model output is codified by the
\texttt{model\_out\_tbl} S3 class in the \texttt{hubUtils} package, one
of the foundational hubverse packages. Although this S3 class is
required for all \texttt{hubEnsembles} functions, model predictions in
other formats can easily be transformed using the
\texttt{as\_model\_out\_tbl()} function from \texttt{hubUtils}. An
example of this transformation is provided in Section~\ref{sec-flu}.

\begin{longtable}[]{@{}
  >{\raggedright\arraybackslash}p{(\columnwidth - 4\tabcolsep) * \real{0.2083}}
  >{\raggedright\arraybackslash}p{(\columnwidth - 4\tabcolsep) * \real{0.3333}}
  >{\raggedright\arraybackslash}p{(\columnwidth - 4\tabcolsep) * \real{0.4583}}@{}}
\caption{A table summarizing how the model output representation columns
are used for predictions of different output types. Adapted from
(hubverse 2022)}\label{tbl-model-output-rep}\tabularnewline
\toprule\noalign{}
\begin{minipage}[b]{\linewidth}\raggedright
\texttt{output\_type}
\end{minipage} & \begin{minipage}[b]{\linewidth}\raggedright
\texttt{output\_type\_id}
\end{minipage} & \begin{minipage}[b]{\linewidth}\raggedright
\texttt{value}
\end{minipage} \\
\midrule\noalign{}
\endfirsthead
\toprule\noalign{}
\begin{minipage}[b]{\linewidth}\raggedright
\texttt{output\_type}
\end{minipage} & \begin{minipage}[b]{\linewidth}\raggedright
\texttt{output\_type\_id}
\end{minipage} & \begin{minipage}[b]{\linewidth}\raggedright
\texttt{value}
\end{minipage} \\
\midrule\noalign{}
\endhead
\bottomrule\noalign{}
\endlastfoot
\texttt{mean} & NA (not used for mean predictions) & Numeric: The mean
of the predictive distribution \\
\texttt{median} & NA (not used for median predictions) & Numeric: The
median of the predictive distribution \\
\texttt{quantile} & Numeric between 0.0 and 1.0: A quantile level &
Numeric: The quantile of the predictive distribution at the quantile
level specified by the \texttt{output\_type\_id} \\
\texttt{cdf} & Numeric within the support of the outcome variable: a
possible value of the target variable & Numeric between 0.0 and 1.0: The
cumulative probability of the predictive distribution at the value of
the outcome variable specified by the \texttt{output\_type\_id} \\
\texttt{pmf} & String naming a possible category of a discrete outcome
variable & Numeric between 0.0 and 1.0: The probability mass of the
predictive distribution when evaluated at a specified level of a
categorical outcome variable \\
\texttt{sample} & Positive integer sample index & Numeric: A sample from
the predictive distribution \\
\end{longtable}

\subsection{\texorpdfstring{Ensemble functions in
\texttt{hubEnsembles}}{Ensemble functions in hubEnsembles}}\label{ensemble-functions-in-hubensembles}

The \texttt{hubEnsembles} package includes two functions that perform
ensemble calculations: \texttt{simple\_ensemble()}, which applies some
function to each model prediction, and \texttt{linear\_pool()}, which
computes an ensemble using the linear opinion pool method. In the
following sections, we outline the implementation details for each
function and how these implementations correspond to the statistical
ensembling methods described in Section~\ref{sec-defs}. A short
description of the calculation performed by each function is summarized
by output type in Table~\ref{tbl-fns-by-output-type}.

\subsubsection{Simple ensemble}\label{simple-ensemble}

The \texttt{simple\_ensemble} function directly computes an ensemble
from component model outputs by combining them via some function (\(C\))
within each unique combination of task ID variables, output types, and
output type IDs. This function can be used to summarize predictions of
output types mean, median, quantile, CDF, and PMF. The mechanics of the
ensemble calculations are the same for each of the output types, though
the resulting statistical ensembling method differs for different output
types (Table~\ref{tbl-fns-by-output-type}).

By default, \texttt{simple\_ensemble} uses the mean for the aggregation
function \(C\) and equal weights for all models. For point predictions
with a mean or median output type, the resulting ensemble prediction is
an equally weighted average of the individual models' predictions. For
probabilistic predictions in a quantile format, by default
\texttt{simple\_ensemble} calculates an equally weighted average of
individual model target variable values at each quantile level, which is
equivalent to a quantile average. For model outputs in a CDF or PMF
format, by default \texttt{simple\_ensemble} computes an equally
weighted average of individual model (cumulative or bin) probabilities
at each target variable value, which is equivalent to the linear pool
method.

Any aggregation function \(C\) may be specified by the user. For
example, a median ensemble may also be created by specifying ``median''
as the aggregation function, or a custom function may be passed to the
\texttt{agg\_fun} argument to create other ensemble types. Similarly,
model weights can be specified to create a weighted ensemble.

\begin{longtable}[]{@{}
  >{\raggedright\arraybackslash}p{(\columnwidth - 4\tabcolsep) * \real{0.1831}}
  >{\raggedright\arraybackslash}p{(\columnwidth - 4\tabcolsep) * \real{0.3380}}
  >{\raggedright\arraybackslash}p{(\columnwidth - 4\tabcolsep) * \real{0.4789}}@{}}
\caption{Summary of ensemble function calculations for each output type.
The ensemble function (columns) determines the operation that is
performed, and in the case of probabilistic output types
(\texttt{quantile}, \texttt{cdf}, \texttt{pmf}), this also determines
what ensemble distribution is generated (quantile average,
\(F_{Q}^{-1}(\theta)\), or linear pool, \(F_{LOP}(x)\)). The resulting
ensemble will be returned in the same output type as the inputs. Thus,
the output type (rows) determines how the resulting ensemble
distribution is summarized (as a quantile function, \(F^{-1}(\theta)\),
cumulative distribution function, \(F(x)\), or probability mass function
\(f(x)\)). Estimating individual model cumulative probabilities is
required to compute a \texttt{linear\_pool()} for predictions of
\texttt{quantile} output type; see Section~\ref{sec-linear-pool} for
details. In the case of \texttt{simple\_ensemble()}, we report the
calculations for the default case where \texttt{agg\_fun\ =\ "mean"};
however, if another aggregation function is chosen (e.g.,
\texttt{agg\_fun\ =\ "median"}), that calculation would be performed
instead. For example,
\texttt{simple\_ensemble(...,\ agg\_fun\ =\ "median")} applied to
predictions of \texttt{mean} output type would return the median of
individual model means.}\label{tbl-fns-by-output-type}\tabularnewline
\toprule\noalign{}
\begin{minipage}[b]{\linewidth}\raggedright
\texttt{output\_type}
\end{minipage} & \begin{minipage}[b]{\linewidth}\raggedright
\texttt{simple\_ensemble(...,\ agg\_fun\ =\ "mean")}
\end{minipage} & \begin{minipage}[b]{\linewidth}\raggedright
\texttt{linear\_pool()}
\end{minipage} \\
\midrule\noalign{}
\endfirsthead
\toprule\noalign{}
\begin{minipage}[b]{\linewidth}\raggedright
\texttt{output\_type}
\end{minipage} & \begin{minipage}[b]{\linewidth}\raggedright
\texttt{simple\_ensemble(...,\ agg\_fun\ =\ "mean")}
\end{minipage} & \begin{minipage}[b]{\linewidth}\raggedright
\texttt{linear\_pool()}
\end{minipage} \\
\midrule\noalign{}
\endhead
\bottomrule\noalign{}
\endlastfoot
\texttt{mean} & mean of individual model means & mean of individual
model means \\
\texttt{median} & mean of individual model medians & NA \\
\texttt{quantile} & mean of individual model target variable values at
each quantile level, \(F^{-1}_Q(\theta)\) & quantile of the distribution
obtained by computing the mean of estimated individual model cumulative
probabilities at each target variable value, \(F^{-1}_{LOP}(x)\) \\
\texttt{cdf} & mean of individual model cumulative probabilities at each
target variable value, \(F_{LOP}(x)\) & mean of individual model
cumulative probabilities at each target variable value,
\(F_{LOP}(x)\) \\
\texttt{pmf} & mean of individual model bin probabilities at each target
variable value, \(f_{LOP}(x)\) & mean of individual model bin
probabilities at each target variable value, \(f_{LOP}(x)\) \\
\end{longtable}

\subsubsection{Linear pool}\label{sec-linear-pool}

The \texttt{linear\_pool} function implements the linear opinion pool
method for ensembling predictions. This function can be used to combine
predictions with output types mean, quantile, CDF, and PMF. Unlike
\texttt{simple\_ensemble}, this function handles its computation
differently based on the output type. For the CDF, PMF, and mean output
types, the linear pool method is equivalent to calling
\texttt{simple\_ensemble} with a mean aggregation function (see
Table~\ref{tbl-fns-by-output-type}), since \texttt{simple\_ensemble}
produces a linear pool prediction (an average of individual model
cumulative or bin probabilities).

However, implementation of LOP is less straightforward for the quantile
output type. This is because LOP averages cumulative probabilities at
each value of the target variable, but the predictions are quantiles (on
the scale of the target variable) for fixed quantile levels. The value
for these quantile predictions will generally differ between models, and
as a result we are typically not provided cumulative probabilities at
the same values of the target variable for all component predictions.
This lack of alignment between cumulative probabilities for the same
target variable values impedes computation of LOP from quantile
predictions and is illustrated in panel A of
Figure~\ref{fig-example-quantile-average-and-linear-pool}.

\begin{figure}

\centering{

\includegraphics{hubEnsembles_manuscript_files/figure-pdf/fig-example-quantile-average-and-linear-pool-1.pdf}

}

\caption{\label{fig-example-quantile-average-and-linear-pool}(Panel A)
Example of quantile output type predictions. In this example, points
show model output collected for seven fixed quantile levels (\(\theta\)
= 0.01, 0.1, 0.3, 0.5, 0.7, 0.9, and 0.99) from two distributions
(\(N(100, 10)\) in purple and \(N(120, 5)\) in green), with the
underlying cumulative distribution functions (CDFs) shown with curves.
The associated values for each fixed quantile level do not align across
distributions (vertical lines). (Panel B) Quantile average ensemble,
which is calculated by averaging values for each fixed quantile level
(represented by horizontal dashed gray lines). The distributions and
corresponding model outputs from panel A are re-plotted and the black
line shows the resulting quantile average ensemble. Inset shows
corresponding probability density functions (PDFs). (Panel C) Linear
pool ensemble, which is calculated by averaging cumulative probabilities
for each fixed value (represented by vertical dashed gray lines). The
distributions and corresponding model outputs from panel A are
re-plotted. To calculate the linear pool in this case, where model
outputs are not defined for the same values, the model outputs are used
to interpolate the full CDF for each distribution from which quantiles
can be extracted for fixed values (shown with open circles). The black
line shows theresulting linear pool average ensemble. Inset shows
corresponding PDFs.}

\end{figure}%

Given that LOP cannot be directly calculated from quantile predictions,
we must first obtain an estimate of the CDF for each component
distribution using the provided quantiles, combine the CDFs, then
calculate the quantiles from the ensemble's CDF. We perform this
calculation in three main steps, assisted by the \texttt{distfromq}
package (Ray and Gerding 2024) for the first two:

\begin{enumerate}
\def\labelenumi{\arabic{enumi}.}
\tightlist
\item
  Interpolate and extrapolate from the provided quantiles for each
  component model to obtain an estimate of the CDF of that particular
  distribution.
\item
  Draw samples from each component model distribution. To reduce Monte
  Carlo variability, we use quasi-random samples corresponding to
  quantiles of the estimated distribution (Niederreiter 1992).
\item
  Pool the samples from all component models and extract the desired
  quantiles.
\end{enumerate}

For step 1, functionality in the \texttt{distfromq} package uses a
monotonic cubic spline for interpolation on the interior of the provided
quantiles. The user may choose one of several distributions to perform
extrapolation of the CDF tails. These include normal, lognormal, and
cauchy distributions, with ``normal'' set as the default. A
location-scale parameterization is used, with separate location and
scale parameters chosen in the lower and upper tails so as to match the
two most extreme quantiles. The sampling process described in steps 2
and 3 approximates the linear pool calculation described in
Section~\ref{sec-defs}.

\section{Demonstration of functionality}\label{sec-simple-ex}

In this section, we provide a simple example to illustrate the two main
functions in \texttt{hubEnsembles}, \texttt{simple\_ensemble()} and
\texttt{linear\_pool()}.

\subsection{Example data: a forecast
hub}\label{example-data-a-forecast-hub}

We will use an example hub provided by the hubverse to demonstrate the
functionality of the \texttt{hubEnsembles} package (hubverse 2022). The
example hub includes both example model output data and target data
(sometimes known as ``truth'' data), which are included in the
\texttt{hubEnsembles} package as data objects named
\texttt{example\_model\_output} and \texttt{example\_target\_data}.

The model output data includes \texttt{quantile}, \texttt{mean} and
\texttt{median} forecasts of future incident influenza hospitalizations
and \texttt{pmf} forecasts of hospitalization intensity. Each forecast
is made for five task ID variables, including the location for which the
forecast was made (\texttt{location}), the date on which the forecast
was made (\texttt{reference\_date}), the number of steps ahead
(\texttt{horizon}), the date of the forecast prediction (a combination
of the date the forecast was made and the forecast horizon,
\texttt{target\_end\_date}), and the forecast target (\texttt{target}).
Table~\ref{tbl-example-forecasts} provides an example set of quantile
forecasts included in this example model output. In
Table~\ref{tbl-example-forecasts}, we show only the median, the 50\%,
and 95\% prediction intervals, although other intervals and
\texttt{mean} forecasts are included in the example model output data.

\begin{Shaded}
\begin{Highlighting}[]
\NormalTok{hubEnsembles}\SpecialCharTok{::}\NormalTok{example\_model\_output }\SpecialCharTok{|\textgreater{}}
\NormalTok{  dplyr}\SpecialCharTok{::}\FunctionTok{filter}\NormalTok{(}
\NormalTok{    output\_type }\SpecialCharTok{\%in\%} \FunctionTok{c}\NormalTok{(}\StringTok{"quantile"}\NormalTok{, }\StringTok{"median"}\NormalTok{),}
\NormalTok{    output\_type\_id }\SpecialCharTok{\%in\%} \FunctionTok{c}\NormalTok{(}\FloatTok{0.025}\NormalTok{, }\FloatTok{0.25}\NormalTok{, }\FloatTok{0.75}\NormalTok{, }\FloatTok{0.75}\NormalTok{, }\FloatTok{0.975}\NormalTok{, }\ConstantTok{NA}\NormalTok{),}
\NormalTok{    reference\_date }\SpecialCharTok{==} \StringTok{"2022{-}12{-}17"}\NormalTok{,}
\NormalTok{    location }\SpecialCharTok{==} \StringTok{"US"}\NormalTok{,}
\NormalTok{    horizon }\SpecialCharTok{==} \DecValTok{1}
\NormalTok{  ) }\SpecialCharTok{|\textgreater{}}
\NormalTok{  dplyr}\SpecialCharTok{::}\FunctionTok{select}\NormalTok{(}\SpecialCharTok{{-}}\NormalTok{location, }\SpecialCharTok{{-}}\NormalTok{target\_end\_date) }\SpecialCharTok{|\textgreater{}}
\NormalTok{  knitr}\SpecialCharTok{::}\FunctionTok{kable}\NormalTok{()}
\end{Highlighting}
\end{Shaded}

\begin{longtable}[]{@{}
  >{\raggedright\arraybackslash}p{(\columnwidth - 12\tabcolsep) * \real{0.2000}}
  >{\raggedright\arraybackslash}p{(\columnwidth - 12\tabcolsep) * \real{0.1667}}
  >{\raggedleft\arraybackslash}p{(\columnwidth - 12\tabcolsep) * \real{0.0889}}
  >{\raggedright\arraybackslash}p{(\columnwidth - 12\tabcolsep) * \real{0.1778}}
  >{\raggedright\arraybackslash}p{(\columnwidth - 12\tabcolsep) * \real{0.1333}}
  >{\raggedright\arraybackslash}p{(\columnwidth - 12\tabcolsep) * \real{0.1667}}
  >{\raggedleft\arraybackslash}p{(\columnwidth - 12\tabcolsep) * \real{0.0667}}@{}}

\caption{\label{tbl-example-forecasts}Example model output for forecasts
of incident influenza hospitalizations. A subset of example model output
is shown: 1-week ahead quantile forecasts made on 2022-12-17 for the US
from three distinct models; only the median, 50\% prediction intervals,
and 95\% prediction intervals are displayed. The \texttt{location} and
\texttt{target\_end\_date} columns have been omitted for brevity. This
example data is provided in the \texttt{hubEnsembles} package and is a
subset of the \texttt{example-complex-forecast-hub} data provided by the
hubverse (hubverse 2022).}

\tabularnewline

\toprule\noalign{}
\begin{minipage}[b]{\linewidth}\raggedright
model\_id
\end{minipage} & \begin{minipage}[b]{\linewidth}\raggedright
reference\_date
\end{minipage} & \begin{minipage}[b]{\linewidth}\raggedleft
horizon
\end{minipage} & \begin{minipage}[b]{\linewidth}\raggedright
target
\end{minipage} & \begin{minipage}[b]{\linewidth}\raggedright
output\_type
\end{minipage} & \begin{minipage}[b]{\linewidth}\raggedright
output\_type\_id
\end{minipage} & \begin{minipage}[b]{\linewidth}\raggedleft
value
\end{minipage} \\
\midrule\noalign{}
\endhead
\bottomrule\noalign{}
\endlastfoot
Flusight-baseline & 2022-12-17 & 1 & wk inc flu hosp & quantile & 0.025
& 18886 \\
Flusight-baseline & 2022-12-17 & 1 & wk inc flu hosp & quantile & 0.25 &
23534 \\
Flusight-baseline & 2022-12-17 & 1 & wk inc flu hosp & quantile & 0.75 &
24369 \\
Flusight-baseline & 2022-12-17 & 1 & wk inc flu hosp & quantile & 0.975
& 28980 \\
Flusight-baseline & 2022-12-17 & 1 & wk inc flu hosp & median & NA &
23951 \\
MOBS-GLEAM\_FLUH & 2022-12-17 & 1 & wk inc flu hosp & quantile & 0.025 &
14791 \\
MOBS-GLEAM\_FLUH & 2022-12-17 & 1 & wk inc flu hosp & quantile & 0.25 &
20676 \\
MOBS-GLEAM\_FLUH & 2022-12-17 & 1 & wk inc flu hosp & quantile & 0.75 &
30801 \\
MOBS-GLEAM\_FLUH & 2022-12-17 & 1 & wk inc flu hosp & quantile & 0.975 &
42528 \\
MOBS-GLEAM\_FLUH & 2022-12-17 & 1 & wk inc flu hosp & median & NA &
25235 \\
PSI-DICE & 2022-12-17 & 1 & wk inc flu hosp & quantile & 0.025 &
14027 \\
PSI-DICE & 2022-12-17 & 1 & wk inc flu hosp & quantile & 0.25 & 16770 \\
PSI-DICE & 2022-12-17 & 1 & wk inc flu hosp & quantile & 0.75 & 23984 \\
PSI-DICE & 2022-12-17 & 1 & wk inc flu hosp & quantile & 0.975 &
27899 \\
PSI-DICE & 2022-12-17 & 1 & wk inc flu hosp & median & NA & 19666 \\

\end{longtable}

We also have corresponding target data included in the
\texttt{hubEnsembles} package (Table~\ref{tbl-example-target-data}). The
example target data provide observed incident influenza hospitalizations
(\texttt{value}) in a given week (\texttt{time\_idx}) and for a given
location (\texttt{location}). This target data could be used as
calibration data for generating forecasts or for evaluating these
forecasts post hoc. The forecast-specific task ID variables
\texttt{reference\_date} and \texttt{horizon} are not relevant for the
target data.

\begin{Shaded}
\begin{Highlighting}[]
\NormalTok{hubEnsembles}\SpecialCharTok{::}\NormalTok{example\_target\_data }\SpecialCharTok{|\textgreater{}}
\NormalTok{  dplyr}\SpecialCharTok{::}\FunctionTok{filter}\NormalTok{(}
\NormalTok{    location }\SpecialCharTok{==} \StringTok{"US"}\NormalTok{,}
\NormalTok{    time\_idx }\SpecialCharTok{\textgreater{}=} \StringTok{"2022{-}11{-}01"}\NormalTok{,}
\NormalTok{    time\_idx }\SpecialCharTok{\textless{}=} \StringTok{"2023{-}02{-}01"}
\NormalTok{  ) }\SpecialCharTok{|\textgreater{}}
\NormalTok{  knitr}\SpecialCharTok{::}\FunctionTok{kable}\NormalTok{()}
\end{Highlighting}
\end{Shaded}

\begin{longtable}[]{@{}llr@{}}

\caption{\label{tbl-example-target-data}Example target data for incident
influenza hospitalizations. This table includes target data from
2022-11-01 and 2023-02-01. This target data is provided in the
\texttt{hubEnsembles} package and is a subset of the
\texttt{example-complex-forecast-hub} target data provided by the
hubverse (hubverse 2022).}

\tabularnewline

\toprule\noalign{}
time\_idx & location & value \\
\midrule\noalign{}
\endhead
\bottomrule\noalign{}
\endlastfoot
2022-11-05 & US & 6571 \\
2022-11-12 & US & 8848 \\
2022-11-19 & US & 11427 \\
2022-11-26 & US & 19846 \\
2022-12-03 & US & 26333 \\
2022-12-10 & US & 23851 \\
2022-12-17 & US & 21435 \\
2022-12-24 & US & 19286 \\
2022-12-31 & US & 19369 \\
2023-01-07 & US & 12928 \\
2023-01-14 & US & 6692 \\
2023-01-21 & US & 4182 \\
2023-01-28 & US & 2838 \\

\end{longtable}

We can plot these forecasts and the target data using the
\texttt{plot\_step\_ahead\_model\_output()} function from
\texttt{hubVis}, another package for visualizing model outputs from the
hubverse suite (Figure~\ref{fig-plot-ex-mods}). We subset the model
output data and the target data to the location and time horizons we are
interested in.

\begin{Shaded}
\begin{Highlighting}[]
\NormalTok{model\_outputs\_plot }\OtherTok{\textless{}{-}}\NormalTok{ hubEnsembles}\SpecialCharTok{::}\NormalTok{example\_model\_output }\SpecialCharTok{|\textgreater{}}
\NormalTok{  hubUtils}\SpecialCharTok{::}\FunctionTok{as\_model\_out\_tbl}\NormalTok{() }\SpecialCharTok{|\textgreater{}}
\NormalTok{  dplyr}\SpecialCharTok{::}\FunctionTok{filter}\NormalTok{(}
\NormalTok{    location }\SpecialCharTok{==} \StringTok{"US"}\NormalTok{,}
\NormalTok{    output\_type }\SpecialCharTok{\%in\%} \FunctionTok{c}\NormalTok{(}\StringTok{"median"}\NormalTok{, }\StringTok{"mean"}\NormalTok{, }\StringTok{"quantile"}\NormalTok{),}
\NormalTok{    reference\_date }\SpecialCharTok{==} \StringTok{"2022{-}12{-}17"}
\NormalTok{  )}

\NormalTok{target\_data\_plot }\OtherTok{\textless{}{-}}\NormalTok{ hubEnsembles}\SpecialCharTok{::}\NormalTok{example\_target\_data }\SpecialCharTok{|\textgreater{}}
\NormalTok{  dplyr}\SpecialCharTok{::}\FunctionTok{filter}\NormalTok{(}
\NormalTok{    location }\SpecialCharTok{==} \StringTok{"US"}\NormalTok{,}
\NormalTok{    time\_idx }\SpecialCharTok{\textgreater{}=} \StringTok{"2022{-}11{-}01"}\NormalTok{,}
\NormalTok{    time\_idx }\SpecialCharTok{\textless{}=} \StringTok{"2023{-}02{-}01"}
\NormalTok{  )}

\NormalTok{hubVis}\SpecialCharTok{::}\FunctionTok{plot\_step\_ahead\_model\_output}\NormalTok{(}
  \AttributeTok{model\_output\_data =}\NormalTok{ model\_outputs\_plot,}
  \AttributeTok{truth\_data =}\NormalTok{ target\_data\_plot,}
  \AttributeTok{facet =} \StringTok{"model\_id"}\NormalTok{,}
  \AttributeTok{facet\_nrow =} \DecValTok{1}\NormalTok{,}
  \AttributeTok{interactive =} \ConstantTok{FALSE}\NormalTok{,}
  \AttributeTok{intervals =} \FunctionTok{c}\NormalTok{(}\FloatTok{0.5}\NormalTok{, }\FloatTok{0.95}\NormalTok{),}
  \AttributeTok{show\_legend =} \ConstantTok{FALSE}\NormalTok{,}
  \AttributeTok{use\_median\_as\_point =} \ConstantTok{TRUE}\NormalTok{,}
  \AttributeTok{x\_col\_name =} \StringTok{"target\_end\_date"}
\NormalTok{) }\SpecialCharTok{+}
  \FunctionTok{theme\_bw}\NormalTok{() }\SpecialCharTok{+}
  \FunctionTok{labs}\NormalTok{(}\AttributeTok{y =} \StringTok{"US incident hospitalizations"}\NormalTok{)}
\end{Highlighting}
\end{Shaded}

\begin{figure}[H]

\centering{

\includegraphics{hubEnsembles_manuscript_files/figure-pdf/fig-plot-ex-mods-1.pdf}

}

\caption{\label{fig-plot-ex-mods}One example quantile forecast of weekly
US incident influenza hospitalizations is shown for each of three models
(panels). Forecasts are represented by a median (line), 50\% prediction
interval (Q25-Q75), and 95\% prediction interval (Q2.5-Q97.5). Gray
points represent observed incident hospitalizations.}

\end{figure}%

Next, we examine the \texttt{pmf} target in the example model output
data. For this target, teams forecasted the probability that
hospitalization intensity will be ``low'', ``moderate'', ``high'', or
``very high''. These hospitalization intensity categories are determined
by thresholds for weekly hospital admissions per 100,000 population. In
other words, ``low'' hospitalization intensity in a given week means few
incident influenza hospitalizations per 100,000 population are
predicted, whereas ``very high'' hospitalization intensity means many
hospitalizations per 100,000 population are predicted. These forecasts
are made for the same task ID variables as the \texttt{quantile}
forecasts of incident hospitalizations.

We show a representative example of the hospitalization intensity
category forecasts in Table~\ref{tbl-example-forecasts-pmf}. Because
these forecasts are \texttt{pmf} output type, the
\texttt{output\_type\_id} column specifies the bin of hospitalization
intensity and the \texttt{value} column provides the forecasted
probability of hospitalization incidence being in that category. Values
sum to 1 across bins. For the MOBS-GLEAM\_FLUH and PSI-DICE models,
incidence is forecasted to decrease over the horizon
(Figure~\ref{fig-plot-ex-mods}), and correspondingly, there is lower
probability of ``high'' and ``very high'' hospitalization intensity for
later horizons (Figure~\ref{fig-plot-ex-mods-pmf}).

\begin{Shaded}
\begin{Highlighting}[]
\NormalTok{hubEnsembles}\SpecialCharTok{::}\NormalTok{example\_model\_output }\SpecialCharTok{|\textgreater{}}
\NormalTok{  dplyr}\SpecialCharTok{::}\FunctionTok{filter}\NormalTok{(}
\NormalTok{    output\_type }\SpecialCharTok{\%in\%} \FunctionTok{c}\NormalTok{(}\StringTok{"pmf"}\NormalTok{),}
\NormalTok{    reference\_date }\SpecialCharTok{==} \StringTok{"2022{-}12{-}17"}\NormalTok{,}
\NormalTok{    location }\SpecialCharTok{==} \StringTok{"US"}\NormalTok{,}
\NormalTok{    horizon }\SpecialCharTok{==} \DecValTok{1}
\NormalTok{  ) }\SpecialCharTok{|\textgreater{}}
\NormalTok{  dplyr}\SpecialCharTok{::}\FunctionTok{mutate}\NormalTok{(}\AttributeTok{value =} \FunctionTok{round}\NormalTok{(value, }\DecValTok{2}\NormalTok{)) }\SpecialCharTok{|\textgreater{}}
\NormalTok{  dplyr}\SpecialCharTok{::}\FunctionTok{select}\NormalTok{(}\SpecialCharTok{{-}}\NormalTok{location, }\SpecialCharTok{{-}}\NormalTok{target\_end\_date) }\SpecialCharTok{|\textgreater{}}
\NormalTok{  knitr}\SpecialCharTok{::}\FunctionTok{kable}\NormalTok{()}
\end{Highlighting}
\end{Shaded}

\begin{longtable}[]{@{}
  >{\raggedright\arraybackslash}p{(\columnwidth - 12\tabcolsep) * \real{0.1800}}
  >{\raggedright\arraybackslash}p{(\columnwidth - 12\tabcolsep) * \real{0.1500}}
  >{\raggedleft\arraybackslash}p{(\columnwidth - 12\tabcolsep) * \real{0.0800}}
  >{\raggedright\arraybackslash}p{(\columnwidth - 12\tabcolsep) * \real{0.2600}}
  >{\raggedright\arraybackslash}p{(\columnwidth - 12\tabcolsep) * \real{0.1200}}
  >{\raggedright\arraybackslash}p{(\columnwidth - 12\tabcolsep) * \real{0.1500}}
  >{\raggedleft\arraybackslash}p{(\columnwidth - 12\tabcolsep) * \real{0.0600}}@{}}

\caption{\label{tbl-example-forecasts-pmf}Example \texttt{pmf} model
output for forecasts of incident influenza hospitalization intensity. A
subset of example model output is shown: 1-week ahead pmf forecasts made
on 2022-12-17 for the US from three distinct models. We round the
forecasted probability (in the \texttt{value} column) to two digits. The
\texttt{location} and \texttt{target\_end\_date} columns have been
omitted for brevity. This example data is provided in the
\texttt{hubEnsembles} package and is a subset of the
\texttt{example-complex-forecast-hub} data provided by the hubverse
(hubverse 2022).}

\tabularnewline

\toprule\noalign{}
\begin{minipage}[b]{\linewidth}\raggedright
model\_id
\end{minipage} & \begin{minipage}[b]{\linewidth}\raggedright
reference\_date
\end{minipage} & \begin{minipage}[b]{\linewidth}\raggedleft
horizon
\end{minipage} & \begin{minipage}[b]{\linewidth}\raggedright
target
\end{minipage} & \begin{minipage}[b]{\linewidth}\raggedright
output\_type
\end{minipage} & \begin{minipage}[b]{\linewidth}\raggedright
output\_type\_id
\end{minipage} & \begin{minipage}[b]{\linewidth}\raggedleft
value
\end{minipage} \\
\midrule\noalign{}
\endhead
\bottomrule\noalign{}
\endlastfoot
Flusight-baseline & 2022-12-17 & 1 & wk flu hosp rate category & pmf &
low & 0.00 \\
Flusight-baseline & 2022-12-17 & 1 & wk flu hosp rate category & pmf &
moderate & 0.01 \\
Flusight-baseline & 2022-12-17 & 1 & wk flu hosp rate category & pmf &
high & 0.86 \\
Flusight-baseline & 2022-12-17 & 1 & wk flu hosp rate category & pmf &
very high & 0.13 \\
MOBS-GLEAM\_FLUH & 2022-12-17 & 1 & wk flu hosp rate category & pmf &
low & 0.00 \\
MOBS-GLEAM\_FLUH & 2022-12-17 & 1 & wk flu hosp rate category & pmf &
moderate & 0.07 \\
MOBS-GLEAM\_FLUH & 2022-12-17 & 1 & wk flu hosp rate category & pmf &
high & 0.41 \\
MOBS-GLEAM\_FLUH & 2022-12-17 & 1 & wk flu hosp rate category & pmf &
very high & 0.52 \\
PSI-DICE & 2022-12-17 & 1 & wk flu hosp rate category & pmf & low &
0.00 \\
PSI-DICE & 2022-12-17 & 1 & wk flu hosp rate category & pmf & moderate &
0.23 \\
PSI-DICE & 2022-12-17 & 1 & wk flu hosp rate category & pmf & high &
0.60 \\
PSI-DICE & 2022-12-17 & 1 & wk flu hosp rate category & pmf & very high
& 0.17 \\

\end{longtable}

\begin{figure}

\centering{

\includegraphics{hubEnsembles_manuscript_files/figure-pdf/fig-plot-ex-mods-pmf-1.pdf}

}

\caption{\label{fig-plot-ex-mods-pmf}One example \texttt{pmf} forecast
of incident influenza hospitalization intensity is shown for each of
three models (panels). Each bar shows forecasts of horizon 0-3 weeks,
and the darkness of the bar shows the hospitalization intensity bins
(low, moderate, high, and very high).}

\end{figure}%

\subsection{\texorpdfstring{Creating ensembles with
\texttt{simple\_ensemble}}{Creating ensembles with simple\_ensemble}}\label{creating-ensembles-with-simple_ensemble}

Using the default options for \texttt{simple\_ensemble()}, we can
generate an equally weighted mean ensemble for each unique combination
of values for the task ID variables, the \texttt{output\_type} and the
\texttt{output\_type\_id}. Recall that this means different ensemble
methods will be used for different output types: for the
\texttt{quantile} output type in our example data, the resulting
ensemble is a quantile average, while for the \texttt{pmf} output type,
the ensemble is a linear pool.

\begin{Shaded}
\begin{Highlighting}[]
\NormalTok{mean\_ens }\OtherTok{\textless{}{-}}\NormalTok{ hubEnsembles}\SpecialCharTok{::}\FunctionTok{simple\_ensemble}\NormalTok{(hubEnsembles}\SpecialCharTok{::}\NormalTok{example\_model\_output,}
  \AttributeTok{model\_id =} \StringTok{"simple{-}ensemble{-}mean"}
\NormalTok{)}
\end{Highlighting}
\end{Shaded}

The resulting model output has the same structure as the original model
output data (Table~\ref{tbl-mean-ensemble}), with columns for model ID,
task ID variables, output type, output type ID, and value. We also use
\texttt{model\_id\ =\ "simple-ensemble-mean} to change the name of this
ensemble in the resulting model output; if not specified, the default
will be ``hub-ensemble''.

\begin{Shaded}
\begin{Highlighting}[]
\NormalTok{mean\_ens }\SpecialCharTok{|\textgreater{}}
\NormalTok{  dplyr}\SpecialCharTok{::}\FunctionTok{filter}\NormalTok{(}
\NormalTok{    output\_type }\SpecialCharTok{\%in\%} \FunctionTok{c}\NormalTok{(}\StringTok{"quantile"}\NormalTok{, }\StringTok{"median"}\NormalTok{, }\StringTok{"pmf"}\NormalTok{),}
\NormalTok{    output\_type\_id }\SpecialCharTok{\%in\%} \FunctionTok{c}\NormalTok{(}
      \FloatTok{0.025}\NormalTok{, }\FloatTok{0.25}\NormalTok{, }\FloatTok{0.75}\NormalTok{, }\FloatTok{0.975}\NormalTok{, }\ConstantTok{NA}\NormalTok{,}
      \StringTok{"low"}\NormalTok{, }\StringTok{"moderate"}\NormalTok{, }\StringTok{"high"}\NormalTok{, }\StringTok{"very high"}
\NormalTok{    ),}
\NormalTok{    reference\_date }\SpecialCharTok{==} \StringTok{"2022{-}12{-}17"}\NormalTok{,}
\NormalTok{    location }\SpecialCharTok{==} \StringTok{"US"}\NormalTok{,}
\NormalTok{    horizon }\SpecialCharTok{==} \DecValTok{1}
\NormalTok{  ) }\SpecialCharTok{|\textgreater{}}
\NormalTok{  dplyr}\SpecialCharTok{::}\FunctionTok{mutate}\NormalTok{(}\AttributeTok{value =} \FunctionTok{round}\NormalTok{(value, }\DecValTok{2}\NormalTok{)) }\SpecialCharTok{|\textgreater{}}
\NormalTok{  dplyr}\SpecialCharTok{::}\FunctionTok{select}\NormalTok{(}\SpecialCharTok{{-}}\NormalTok{location, }\SpecialCharTok{{-}}\NormalTok{target\_end\_date) }\SpecialCharTok{|\textgreater{}}
\NormalTok{  knitr}\SpecialCharTok{::}\FunctionTok{kable}\NormalTok{()}
\end{Highlighting}
\end{Shaded}

\begin{longtable}[]{@{}
  >{\raggedright\arraybackslash}p{(\columnwidth - 12\tabcolsep) * \real{0.1981}}
  >{\raggedright\arraybackslash}p{(\columnwidth - 12\tabcolsep) * \real{0.1415}}
  >{\raggedleft\arraybackslash}p{(\columnwidth - 12\tabcolsep) * \real{0.0755}}
  >{\raggedright\arraybackslash}p{(\columnwidth - 12\tabcolsep) * \real{0.2453}}
  >{\raggedright\arraybackslash}p{(\columnwidth - 12\tabcolsep) * \real{0.1132}}
  >{\raggedright\arraybackslash}p{(\columnwidth - 12\tabcolsep) * \real{0.1415}}
  >{\raggedleft\arraybackslash}p{(\columnwidth - 12\tabcolsep) * \real{0.0849}}@{}}

\caption{\label{tbl-mean-ensemble}Mean ensemble model output. The values
in the model\_id column are determined by
\texttt{simple\_ensemble(...,\ model\_id\ =\ )} argument). A subset of
ensemble model output is shown: 1-week ahead pmf forecasts made on
2022-12-17 for the US. Results are generated for all output types. Here,
we show only the median, 50\% prediction intervals, and 95\% prediction
intervals for the quantile output type and all bins for the pmf output
type. The \texttt{location} and \texttt{target\_end\_date} columns have
been omitted for brevity, and the \texttt{value} column is rounded to
two digits.}

\tabularnewline

\toprule\noalign{}
\begin{minipage}[b]{\linewidth}\raggedright
model\_id
\end{minipage} & \begin{minipage}[b]{\linewidth}\raggedright
reference\_date
\end{minipage} & \begin{minipage}[b]{\linewidth}\raggedleft
horizon
\end{minipage} & \begin{minipage}[b]{\linewidth}\raggedright
target
\end{minipage} & \begin{minipage}[b]{\linewidth}\raggedright
output\_type
\end{minipage} & \begin{minipage}[b]{\linewidth}\raggedright
output\_type\_id
\end{minipage} & \begin{minipage}[b]{\linewidth}\raggedleft
value
\end{minipage} \\
\midrule\noalign{}
\endhead
\bottomrule\noalign{}
\endlastfoot
simple-ensemble-mean & 2022-12-17 & 1 & wk flu hosp rate category & pmf
& high & 0.62 \\
simple-ensemble-mean & 2022-12-17 & 1 & wk flu hosp rate category & pmf
& low & 0.00 \\
simple-ensemble-mean & 2022-12-17 & 1 & wk flu hosp rate category & pmf
& moderate & 0.10 \\
simple-ensemble-mean & 2022-12-17 & 1 & wk flu hosp rate category & pmf
& very high & 0.27 \\
simple-ensemble-mean & 2022-12-17 & 1 & wk inc flu hosp & median & NA &
22950.67 \\
simple-ensemble-mean & 2022-12-17 & 1 & wk inc flu hosp & quantile &
0.025 & 15901.33 \\
simple-ensemble-mean & 2022-12-17 & 1 & wk inc flu hosp & quantile &
0.25 & 20326.67 \\
simple-ensemble-mean & 2022-12-17 & 1 & wk inc flu hosp & quantile &
0.75 & 26384.67 \\
simple-ensemble-mean & 2022-12-17 & 1 & wk inc flu hosp & quantile &
0.975 & 33135.67 \\

\end{longtable}

\subsubsection{Changing the aggregation
function}\label{changing-the-aggregation-function}

We can change the function that is used to aggregate model outputs. For
example, we may want to calculate a median of the component models'
submitted values for each quantile. We do so by specifying
\texttt{agg\_fun\ =\ median}.

\begin{Shaded}
\begin{Highlighting}[]
\NormalTok{median\_ens }\OtherTok{\textless{}{-}}\NormalTok{ hubEnsembles}\SpecialCharTok{::}\FunctionTok{simple\_ensemble}\NormalTok{(hubEnsembles}\SpecialCharTok{::}\NormalTok{example\_model\_output,}
  \AttributeTok{agg\_fun =}\NormalTok{ median,}
  \AttributeTok{model\_id =} \StringTok{"simple{-}ensemble{-}median"}
\NormalTok{)}
\end{Highlighting}
\end{Shaded}

Custom functions can also be passed into the \texttt{agg\_fun} argument.
We illustrate this by defining a custom function to compute the ensemble
prediction as a geometric mean of the component model predictions. Any
custom function to be used must have an argument \texttt{x} for the
vector of numeric values to summarize, and if relevant, an argument
\texttt{w} of numeric weights.

\begin{Shaded}
\begin{Highlighting}[]
\NormalTok{geometric\_mean }\OtherTok{\textless{}{-}} \ControlFlowTok{function}\NormalTok{(x) \{}
\NormalTok{  n }\OtherTok{\textless{}{-}} \FunctionTok{length}\NormalTok{(x)}
  \FunctionTok{return}\NormalTok{(}\FunctionTok{prod}\NormalTok{(x)}\SpecialCharTok{\^{}}\NormalTok{(}\DecValTok{1} \SpecialCharTok{/}\NormalTok{ n))}
\NormalTok{\}}

\NormalTok{geometric\_mean\_ens }\OtherTok{\textless{}{-}}
\NormalTok{  hubEnsembles}\SpecialCharTok{::}\FunctionTok{simple\_ensemble}\NormalTok{(hubEnsembles}\SpecialCharTok{::}\NormalTok{example\_model\_output,}
    \AttributeTok{agg\_fun =}\NormalTok{ geometric\_mean,}
    \AttributeTok{model\_id =} \StringTok{"simple{-}ensemble{-}geometric"}
\NormalTok{  )}
\end{Highlighting}
\end{Shaded}

As expected, the mean, median, and geometric mean each give us slightly
different resulting ensembles. The median point estimates, 50\%
prediction intervals, and 95\% prediction intervals in
Figure~\ref{fig-plot-ensembles} demonstrate this.

\begin{Shaded}
\begin{Highlighting}[]
\NormalTok{model\_output\_plot }\OtherTok{\textless{}{-}}\NormalTok{ dplyr}\SpecialCharTok{::}\FunctionTok{bind\_rows}\NormalTok{(}
\NormalTok{  mean\_ens, median\_ens,}
\NormalTok{  geometric\_mean\_ens}
\NormalTok{) }\SpecialCharTok{|\textgreater{}}
\NormalTok{  dplyr}\SpecialCharTok{::}\FunctionTok{filter}\NormalTok{(}
\NormalTok{    location }\SpecialCharTok{==} \StringTok{"US"}\NormalTok{,}
\NormalTok{    output\_type }\SpecialCharTok{\%in\%} \FunctionTok{c}\NormalTok{(}\StringTok{"median"}\NormalTok{, }\StringTok{"mean"}\NormalTok{, }\StringTok{"quantile"}\NormalTok{),}
\NormalTok{    reference\_date }\SpecialCharTok{==} \StringTok{"2022{-}12{-}17"}
\NormalTok{  ) }\SpecialCharTok{|\textgreater{}}
\NormalTok{  dplyr}\SpecialCharTok{::}\FunctionTok{mutate}\NormalTok{(}\AttributeTok{target\_date =}\NormalTok{ reference\_date }\SpecialCharTok{+}\NormalTok{ horizon)}

\NormalTok{target\_data\_plot }\OtherTok{\textless{}{-}}\NormalTok{ hubEnsembles}\SpecialCharTok{::}\NormalTok{example\_target\_data }\SpecialCharTok{|\textgreater{}}
\NormalTok{  dplyr}\SpecialCharTok{::}\FunctionTok{filter}\NormalTok{(}
\NormalTok{    location }\SpecialCharTok{==} \StringTok{"US"}\NormalTok{, time\_idx }\SpecialCharTok{\textgreater{}=} \StringTok{"2022{-}11{-}01"}\NormalTok{,}
\NormalTok{    time\_idx }\SpecialCharTok{\textless{}=} \StringTok{"2023{-}03{-}01"}
\NormalTok{  )}

\NormalTok{hubVis}\SpecialCharTok{::}\FunctionTok{plot\_step\_ahead\_model\_output}\NormalTok{(}
  \AttributeTok{model\_output\_data =}\NormalTok{ model\_output\_plot,}
  \AttributeTok{truth\_data =}\NormalTok{ target\_data\_plot,}
  \AttributeTok{use\_median\_as\_point =} \ConstantTok{TRUE}\NormalTok{,}
  \AttributeTok{interactive =} \ConstantTok{FALSE}\NormalTok{,}
  \AttributeTok{intervals =} \FunctionTok{c}\NormalTok{(}\FloatTok{0.5}\NormalTok{, }\FloatTok{0.95}\NormalTok{),}
  \AttributeTok{show\_legend =} \ConstantTok{TRUE}\NormalTok{,}
  \AttributeTok{x\_col\_name =} \StringTok{"target\_end\_date"}
\NormalTok{) }\SpecialCharTok{+}
  \FunctionTok{theme\_bw}\NormalTok{() }\SpecialCharTok{+}
  \FunctionTok{labs}\NormalTok{(}\AttributeTok{y =} \StringTok{"US incident hospitalizations"}\NormalTok{)}
\end{Highlighting}
\end{Shaded}

\begin{figure}[H]

\centering{

\includegraphics{hubEnsembles_manuscript_files/figure-pdf/fig-plot-ensembles-1.pdf}

}

\caption{\label{fig-plot-ensembles}Three different ensembles for weekly
US incident influenza hospitalizations. Each ensemble combines
individual predictions from the example hub
(Figure~\ref{fig-plot-ex-mods}) using a different method: arithmetic
mean, geometric mean, or median. All methods correspond to variations of
the quantile average approach.}

\end{figure}%

\subsubsection{Weighting model
contributions}\label{weighting-model-contributions}

We can weight the contributions of each model in the ensemble using the
\texttt{weights} argument of \texttt{simple\_ensemble}. This arguement
takes a \texttt{data.frame} that should include a \texttt{model\_id}
column containing each unique model ID and a \texttt{weight} column. In
the following example, we include the baseline model in the ensemble,
but give it less weight than the other forecasts.

\begin{Shaded}
\begin{Highlighting}[]
\NormalTok{model\_weights }\OtherTok{\textless{}{-}} \FunctionTok{data.frame}\NormalTok{(}
  \AttributeTok{model\_id =} \FunctionTok{c}\NormalTok{(}
    \StringTok{"MOBS{-}GLEAM\_FLUH"}\NormalTok{, }\StringTok{"PSI{-}DICE"}\NormalTok{,}
    \StringTok{"simple\_hub{-}baseline"}
\NormalTok{  ),}
  \AttributeTok{weight =} \FunctionTok{c}\NormalTok{(}\FloatTok{0.4}\NormalTok{, }\FloatTok{0.4}\NormalTok{, }\FloatTok{0.2}\NormalTok{)}
\NormalTok{)}

\NormalTok{weighted\_mean\_ens }\OtherTok{\textless{}{-}}
\NormalTok{  hubEnsembles}\SpecialCharTok{::}\FunctionTok{simple\_ensemble}\NormalTok{(}
\NormalTok{    hubEnsembles}\SpecialCharTok{::}\NormalTok{example\_model\_output,}
    \AttributeTok{weights =}\NormalTok{ model\_weights,}
    \AttributeTok{model\_id =} \StringTok{"simple{-}ensemble{-}weighted{-}mean"}
\NormalTok{  )}
\end{Highlighting}
\end{Shaded}

\subsection{\texorpdfstring{Creating ensembles with
\texttt{linear\_pool}}{Creating ensembles with linear\_pool}}\label{creating-ensembles-with-linear_pool}

We can also generate a linear pool ensemble, or distributional mixture,
using the \texttt{linear\_pool()} function; this function can be applied
to predictions with an \texttt{output\_type} of \texttt{mean},
\texttt{quantile}, \texttt{cdf}, or \texttt{pmf}. Our example hub
includes \texttt{median} output type, so we exclude it from the
calculation.

\begin{Shaded}
\begin{Highlighting}[]
\NormalTok{linear\_pool\_ens }\OtherTok{\textless{}{-}}
\NormalTok{  hubEnsembles}\SpecialCharTok{::}\FunctionTok{linear\_pool}\NormalTok{(}
\NormalTok{    dplyr}\SpecialCharTok{::}\FunctionTok{filter}\NormalTok{(}
\NormalTok{      hubEnsembles}\SpecialCharTok{::}\NormalTok{example\_model\_output,}
\NormalTok{      output\_type }\SpecialCharTok{!=} \StringTok{"median"}
\NormalTok{    ),}
    \AttributeTok{model\_id =} \StringTok{"linear{-}pool"}
\NormalTok{  )}
\end{Highlighting}
\end{Shaded}

As described above, for \texttt{quantile} model outputs, the
\texttt{linear\_pool} function approximates the full probability
distribution for each component prediction using the value-quantile
pairs provided by that model, and then obtains quasi-random samples from
that distributional estimate. The number of samples drawn from the
distribution of each component model defaults to \texttt{1e4}, but this
can be changed using the \texttt{n\_samples} argument.

In Figure~\ref{fig-plot-ex-quantile-and-linear-pool}, we compare
ensemble results generated by \texttt{simple\_ensemble} and
\texttt{linear\_pool} for model outputs of output types PMF and
quantile. As expected, the results from the two functions are equivalent
for the PMF output type: for this output type, the
\texttt{simple\_ensemble} method averages the predicted probability of
each category across the component models, which is the definition of
the linear pool ensemble method. This is not the case for the quantile
output type, because the \texttt{simple\_ensemble} is computing a
quantile average.

\begin{Shaded}
\begin{Highlighting}[]
\NormalTok{p1 }\OtherTok{\textless{}{-}}\NormalTok{ dplyr}\SpecialCharTok{::}\FunctionTok{bind\_rows}\NormalTok{(}
\NormalTok{  mean\_ens,}
\NormalTok{  linear\_pool\_ens}
\NormalTok{) }\SpecialCharTok{|\textgreater{}}
\NormalTok{  dplyr}\SpecialCharTok{::}\FunctionTok{filter}\NormalTok{(}
\NormalTok{    output\_type }\SpecialCharTok{==} \StringTok{"pmf"}\NormalTok{, reference\_date }\SpecialCharTok{==} \StringTok{"2022{-}12{-}17"}\NormalTok{,}
\NormalTok{    location }\SpecialCharTok{==} \StringTok{"US"}
\NormalTok{  ) }\SpecialCharTok{|\textgreater{}}
\NormalTok{  dplyr}\SpecialCharTok{::}\FunctionTok{mutate}\NormalTok{(}\AttributeTok{output\_type\_id =} \FunctionTok{gsub}\NormalTok{(}\StringTok{"\_"}\NormalTok{, }\StringTok{" "}\NormalTok{, output\_type\_id)) }\SpecialCharTok{|\textgreater{}}
\NormalTok{  dplyr}\SpecialCharTok{::}\FunctionTok{mutate}\NormalTok{(}\AttributeTok{output\_type\_id =} \FunctionTok{factor}\NormalTok{(output\_type\_id,}
    \AttributeTok{levels =} \FunctionTok{c}\NormalTok{(}
      \StringTok{"low"}\NormalTok{, }\StringTok{"moderate"}\NormalTok{, }\StringTok{"high"}\NormalTok{,}
      \StringTok{"very high"}
\NormalTok{    )}
\NormalTok{  )) }\SpecialCharTok{|\textgreater{}}
  \FunctionTok{ggplot}\NormalTok{(}\FunctionTok{aes}\NormalTok{(}\AttributeTok{x =}\NormalTok{ output\_type\_id, }\AttributeTok{y =}\NormalTok{ value, }\AttributeTok{fill =}\NormalTok{ model\_id)) }\SpecialCharTok{+}
  \FunctionTok{geom\_bar}\NormalTok{(}\AttributeTok{stat =} \StringTok{"identity"}\NormalTok{, }\AttributeTok{position =} \StringTok{"dodge"}\NormalTok{) }\SpecialCharTok{+}
  \FunctionTok{facet\_wrap}\NormalTok{(}\FunctionTok{vars}\NormalTok{(horizon), }\AttributeTok{labeller =}\NormalTok{ label\_both) }\SpecialCharTok{+}
  \FunctionTok{labs}\NormalTok{(}\AttributeTok{x =} \StringTok{"US incident hospitalization intensity"}\NormalTok{, }\AttributeTok{y =} \StringTok{"probability"}\NormalTok{) }\SpecialCharTok{+}
  \FunctionTok{scale\_fill\_brewer}\NormalTok{(}\AttributeTok{palette =} \StringTok{"Set1"}\NormalTok{) }\SpecialCharTok{+}
  \FunctionTok{theme\_bw}\NormalTok{() }\SpecialCharTok{+}
  \FunctionTok{theme}\NormalTok{(}
    \AttributeTok{legend.position =} \StringTok{"bottom"}\NormalTok{, }\AttributeTok{legend.title =} \FunctionTok{element\_blank}\NormalTok{(),}
    \AttributeTok{strip.background =} \FunctionTok{element\_blank}\NormalTok{(), }\AttributeTok{strip.placement =} \StringTok{"outside"}\NormalTok{,}
    \AttributeTok{panel.grid.major.x =} \FunctionTok{element\_blank}\NormalTok{()}
\NormalTok{  )}

\NormalTok{model\_output\_plot }\OtherTok{\textless{}{-}}\NormalTok{ linear\_pool\_ens }\SpecialCharTok{|\textgreater{}}
\NormalTok{  dplyr}\SpecialCharTok{::}\FunctionTok{filter}\NormalTok{(output\_type\_id }\SpecialCharTok{==} \FloatTok{0.5}\NormalTok{) }\SpecialCharTok{|\textgreater{}}
\NormalTok{  dplyr}\SpecialCharTok{::}\FunctionTok{mutate}\NormalTok{(}\AttributeTok{output\_type =} \StringTok{"median"}\NormalTok{, }\AttributeTok{output\_type\_id =} \ConstantTok{NA}\NormalTok{)}
\NormalTok{model\_output\_plot }\OtherTok{\textless{}{-}}\NormalTok{ dplyr}\SpecialCharTok{::}\FunctionTok{bind\_rows}\NormalTok{(linear\_pool\_ens, model\_output\_plot)}
\NormalTok{model\_output\_plot }\OtherTok{\textless{}{-}}\NormalTok{ dplyr}\SpecialCharTok{::}\FunctionTok{bind\_rows}\NormalTok{(mean\_ens, model\_output\_plot) }\SpecialCharTok{|\textgreater{}}
\NormalTok{  dplyr}\SpecialCharTok{::}\FunctionTok{filter}\NormalTok{(}
\NormalTok{    location }\SpecialCharTok{==} \StringTok{"US"}\NormalTok{,}
\NormalTok{    output\_type }\SpecialCharTok{\%in\%} \FunctionTok{c}\NormalTok{(}\StringTok{"median"}\NormalTok{, }\StringTok{"mean"}\NormalTok{, }\StringTok{"quantile"}\NormalTok{),}
\NormalTok{    reference\_date }\SpecialCharTok{==} \StringTok{"2022{-}12{-}17"}
\NormalTok{  ) }\SpecialCharTok{|\textgreater{}}
\NormalTok{  dplyr}\SpecialCharTok{::}\FunctionTok{mutate}\NormalTok{(}\AttributeTok{target\_date =}\NormalTok{ reference\_date }\SpecialCharTok{+}\NormalTok{ horizon)}
\NormalTok{target\_data\_plot }\OtherTok{\textless{}{-}}\NormalTok{ hubEnsembles}\SpecialCharTok{::}\NormalTok{example\_target\_data }\SpecialCharTok{|\textgreater{}}
\NormalTok{  dplyr}\SpecialCharTok{::}\FunctionTok{filter}\NormalTok{(}
\NormalTok{    location }\SpecialCharTok{==} \StringTok{"US"}\NormalTok{, time\_idx }\SpecialCharTok{\textgreater{}=} \StringTok{"2022{-}11{-}01"}\NormalTok{,}
\NormalTok{    time\_idx }\SpecialCharTok{\textless{}=} \StringTok{"2023{-}03{-}01"}
\NormalTok{  )}

\NormalTok{p2 }\OtherTok{\textless{}{-}}
\NormalTok{  hubVis}\SpecialCharTok{::}\FunctionTok{plot\_step\_ahead\_model\_output}\NormalTok{(}
    \AttributeTok{model\_output\_data =}\NormalTok{ model\_output\_plot,}
    \AttributeTok{truth\_data =}\NormalTok{ target\_data\_plot,}
    \AttributeTok{use\_median\_as\_point =} \ConstantTok{TRUE}\NormalTok{,}
    \AttributeTok{interactive =} \ConstantTok{FALSE}\NormalTok{,}
    \AttributeTok{intervals =} \FunctionTok{c}\NormalTok{(}\FloatTok{0.5}\NormalTok{, }\FloatTok{0.95}\NormalTok{),}
    \AttributeTok{pal\_color =} \StringTok{"Set1"}\NormalTok{,}
    \AttributeTok{show\_legend =} \ConstantTok{TRUE}\NormalTok{,}
    \AttributeTok{x\_col\_name =} \StringTok{"target\_end\_date"}
\NormalTok{  ) }\SpecialCharTok{+}
  \FunctionTok{theme\_bw}\NormalTok{() }\SpecialCharTok{+}
  \FunctionTok{labs}\NormalTok{(}\AttributeTok{y =} \StringTok{"US incident hospitalizations"}\NormalTok{)}

\NormalTok{l }\OtherTok{\textless{}{-}}\NormalTok{ cowplot}\SpecialCharTok{::}\FunctionTok{get\_legend}\NormalTok{(p1)}

\NormalTok{cowplot}\SpecialCharTok{::}\FunctionTok{plot\_grid}\NormalTok{(}
\NormalTok{  cowplot}\SpecialCharTok{::}\FunctionTok{plot\_grid}\NormalTok{(}
\NormalTok{    p1 }\SpecialCharTok{+}
      \FunctionTok{labs}\NormalTok{(}
        \AttributeTok{subtitle =}
          \StringTok{"example PMF output type"}
\NormalTok{      ) }\SpecialCharTok{+}
      \FunctionTok{theme}\NormalTok{(}\AttributeTok{legend.position =} \StringTok{"none"}\NormalTok{),}
\NormalTok{    p2 }\SpecialCharTok{+}
      \FunctionTok{labs}\NormalTok{(}
        \AttributeTok{subtitle =}
          \StringTok{"example quantile output type"}
\NormalTok{      ) }\SpecialCharTok{+}
      \FunctionTok{theme}\NormalTok{(}\AttributeTok{legend.position =} \StringTok{"none"}\NormalTok{),}
    \AttributeTok{labels =}\NormalTok{ LETTERS[}\DecValTok{1}\SpecialCharTok{:}\DecValTok{2}\NormalTok{]}
\NormalTok{  ), l,}
  \AttributeTok{ncol =} \DecValTok{1}\NormalTok{,}
  \AttributeTok{rel\_heights =} \FunctionTok{c}\NormalTok{(}\FloatTok{0.95}\NormalTok{, }\FloatTok{0.05}\NormalTok{)}
\NormalTok{)}
\end{Highlighting}
\end{Shaded}

\begin{figure}[H]

\centering{

\includegraphics{hubEnsembles_manuscript_files/figure-pdf/fig-plot-ex-quantile-and-linear-pool-1.pdf}

}

\caption{\label{fig-plot-ex-quantile-and-linear-pool}Comparison of
results from \texttt{simple\_ensemble} (blue) and \texttt{linear\ pool}
(red). (Panel A) Ensemble predictions of US incident influenza
hospitalization intensity (classified as low, moderate, high, or very
high), which provide an example of PMF output type. (Panel B) Ensemble
predictions of weekly US incident influenza hospitalizations, which
provide an example of quantile output type. Note, for quantile output
type, \texttt{simple\_ensemble} corresponds to a quantile average.
Ensembles combine individual models from the example hub
(Figure~\ref{fig-plot-ex-mods}).}

\end{figure}%

\section{Case study: Weekly incident flu
hospitalizations}\label{sec-flu}

To demonstrate the utility of the \texttt{hubEnsembles} package and the
differences between the two ensembling functions, we examine a second
set of weekly US influenza hospitalizations forecasts which were
generated in real time.

Since 2013, the US Centers for Disease Control and Prevention (CDC) has
been soliciting forecasts of seasonal influenza from modeling teams
through a collaborative challenge called FluSight (CDC 2023). We use a
subset of these predictions to create four equally-weighted ensembles
with \texttt{simple\_ensemble()} and \texttt{linear\_pool()} and compare
the resulting ensembles' performance. The ensembling methods chosen for
this case study consist of a quantile (arithmetic) mean, a quantile
median, a linear pool with normal tails, and a linear pool with
lognormal tails. Note that only a select portion of the code is shown in
this manuscript for brevity, although all the functions and scripts used
to generate the case study results can be found in the associated GitHub
repository hubEnsemblesManuscript
(\url{https://github.com/Infectious-Disease-Modeling-Hubs/hubEnsemblesManuscript}).

\subsection{Data and Methods}\label{data-and-methods}

We begin by querying the component forecasts used to generate the four
ensembles from zoltar, a repository designed to archive forecasts
created by the Reich Lab at UMass Amherst. Here, we only consider
Flusight predictions in a quantile format from the 2021-2022 and
2022-2023 seasons. These forecasts were stored in two data objects,
split by season, called \texttt{flu\_forecasts-raw\_21-22.rds} and
\texttt{flu\_forecasts-raw\_22-23.rds}. Since zoltar has its own
formatting conventions, the raw forecasts must be transformed to fit
hubverse standards before being fed into either of the ensembling
functions. To do so, we use the \texttt{as\_model\_out\_tbl()} function
from the \texttt{hubUtils} package. Here, we specify the task ID
variables \texttt{forecast\ date} (when the forecast was made),
\texttt{location}, \texttt{horizon}, and \texttt{target}.

\begin{Shaded}
\begin{Highlighting}[]
\NormalTok{flu\_forecasts\_raw\_21\_22 }\OtherTok{\textless{}{-}}\NormalTok{ readr}\SpecialCharTok{::}\FunctionTok{read\_rds}\NormalTok{(}
\NormalTok{  here}\SpecialCharTok{::}\FunctionTok{here}\NormalTok{(}\StringTok{"analysis/data/raw\_data/flu\_forecasts{-}raw\_21{-}22.rds"}\NormalTok{)}
\NormalTok{)}
\NormalTok{flu\_forecasts\_raw\_22\_23 }\OtherTok{\textless{}{-}}\NormalTok{ readr}\SpecialCharTok{::}\FunctionTok{read\_rds}\NormalTok{(}
\NormalTok{  here}\SpecialCharTok{::}\FunctionTok{here}\NormalTok{(}\StringTok{"analysis/data/raw\_data/flu\_forecasts{-}raw\_22{-}23.rds"}\NormalTok{)}
\NormalTok{)}
\NormalTok{flu\_forecasts\_raw }\OtherTok{\textless{}{-}} \FunctionTok{rbind}\NormalTok{(flu\_forecasts\_raw\_21\_22, flu\_forecasts\_raw\_22\_23)}

\NormalTok{flu\_forecasts\_raw\_21\_22 }\OtherTok{\textless{}{-}}\NormalTok{ readr}\SpecialCharTok{::}\FunctionTok{read\_rds}\NormalTok{(}
\NormalTok{  here}\SpecialCharTok{::}\FunctionTok{here}\NormalTok{(}\StringTok{"analysis/data/raw\_data/flu\_forecasts{-}raw\_21{-}22.rds"}\NormalTok{)}
\NormalTok{)}

\NormalTok{flu\_forecasts\_hubverse }\OtherTok{\textless{}{-}}\NormalTok{ flu\_forecasts\_raw }\SpecialCharTok{|\textgreater{}}
\NormalTok{  dplyr}\SpecialCharTok{::}\FunctionTok{rename}\NormalTok{(}\AttributeTok{forecast\_date =}\NormalTok{ timezero, }\AttributeTok{location =}\NormalTok{ unit) }\SpecialCharTok{|\textgreater{}}
\NormalTok{  tidyr}\SpecialCharTok{::}\FunctionTok{separate}\NormalTok{(target,}
    \AttributeTok{sep =} \StringTok{" "}\NormalTok{, }\AttributeTok{convert =} \ConstantTok{TRUE}\NormalTok{,}
    \AttributeTok{into =} \FunctionTok{c}\NormalTok{(}\StringTok{"horizon"}\NormalTok{, }\StringTok{"target"}\NormalTok{), }\AttributeTok{extra =} \StringTok{"merge"}
\NormalTok{  ) }\SpecialCharTok{|\textgreater{}}
  \FunctionTok{as\_model\_out\_tbl}\NormalTok{(}
    \AttributeTok{model\_id\_col =} \StringTok{"model"}\NormalTok{,}
    \AttributeTok{output\_type\_col =} \StringTok{"class"}\NormalTok{,}
    \AttributeTok{output\_type\_id\_col =} \StringTok{"quantile"}\NormalTok{,}
    \AttributeTok{value\_col =} \StringTok{"value"}\NormalTok{,}
    \AttributeTok{sep =} \StringTok{"{-}"}\NormalTok{,}
    \AttributeTok{trim\_to\_task\_ids =} \ConstantTok{FALSE}\NormalTok{,}
    \AttributeTok{hub\_con =} \ConstantTok{NULL}\NormalTok{,}
    \AttributeTok{task\_id\_cols =} \FunctionTok{c}\NormalTok{(}
      \StringTok{"forecast\_date"}\NormalTok{, }\StringTok{"location"}\NormalTok{, }\StringTok{"horizon"}\NormalTok{,}
      \StringTok{"target"}
\NormalTok{    ),}
    \AttributeTok{remove\_empty =} \ConstantTok{TRUE}
\NormalTok{  )}
\end{Highlighting}
\end{Shaded}

We excluded component forecasts from the ensemble calculations if any
prediction (defined by a unique combination of task ID variables) did
not include all 23 quantiles specified by FluSight
(\(\theta \in \{.010, 0.025, .050, .100, ..., .900, .950, .990\}\)). We
also excluded the FluSight baseline and median ensemble models that were
previously generated. In practice, this means that all ensembles
generated here include the real-time forecasts made by outside teams,
matching the true fluctuation in the number of models used to generate
the FluSight median ensemble (since number of submitted models varied
from week to week).

With these inclusion criteria, the final data set of component forecasts
consists of predictions from 25 modeling teams and 42 distinct models,
53 forecast dates (one per week), 54 US locations, 4 horizons, 1 target,
and 23 quantiles. In the 2021-2022 season, 23 models made predictions
for 22 weeks spanning from late January 2022 to late June 2022, and in
the 2022-2023 season, there were 18 models making predictions for 31
weeks spanning mid-October 2022 to mid-May 2023. In both seasons,
forecasts were made for the same locations (the 50 states, Washington
DC, Puerto Rico, the Virgin Islands, and the US as a whole), horizons (1
to 4 weeks ahead), quantiles (the 23 described above), and target (week
ahead incident flu hospitalization). The values for the forecasts are
always non-negative. In Table~\ref{tbl-case-study-flu-forecasts}, we
provide an example of these predictions, showing select quantiles from a
single model, forecast date, horizon, and location.

\begin{Shaded}
\begin{Highlighting}[]
\NormalTok{readr}\SpecialCharTok{::}\FunctionTok{read\_rds}\NormalTok{(}
\NormalTok{  here}\SpecialCharTok{::}\FunctionTok{here}\NormalTok{(}\StringTok{"analysis/data/raw\_data/flu\_forecasts{-}small.rds"}\NormalTok{)}
\NormalTok{) }\SpecialCharTok{|\textgreater{}}
\NormalTok{  dplyr}\SpecialCharTok{::}\FunctionTok{filter}\NormalTok{(}
\NormalTok{    model\_id }\SpecialCharTok{==} \StringTok{"UMass{-}trends\_ensemble"}\NormalTok{,}
\NormalTok{    forecast\_date }\SpecialCharTok{==} \StringTok{"2023{-}05{-}15"}\NormalTok{,}
\NormalTok{    location }\SpecialCharTok{==} \StringTok{"06"}\NormalTok{,}
\NormalTok{    horizon }\SpecialCharTok{==} \DecValTok{1}\NormalTok{,}
\NormalTok{    output\_type\_id }\SpecialCharTok{\%in\%} \FunctionTok{c}\NormalTok{(}\FloatTok{0.025}\NormalTok{, }\FloatTok{0.1}\NormalTok{, }\FloatTok{0.25}\NormalTok{, }\FloatTok{0.75}\NormalTok{, }\FloatTok{0.9}\NormalTok{, }\FloatTok{0.975}\NormalTok{)}
\NormalTok{  ) }\SpecialCharTok{|\textgreater{}}
\NormalTok{  dplyr}\SpecialCharTok{::}\FunctionTok{select}\NormalTok{(}\SpecialCharTok{{-}}\NormalTok{location, }\SpecialCharTok{{-}}\NormalTok{season) }\SpecialCharTok{|\textgreater{}}
\NormalTok{  knitr}\SpecialCharTok{::}\FunctionTok{kable}\NormalTok{()}
\end{Highlighting}
\end{Shaded}

\begin{longtable}[]{@{}
  >{\raggedright\arraybackslash}p{(\columnwidth - 12\tabcolsep) * \real{0.2222}}
  >{\raggedright\arraybackslash}p{(\columnwidth - 12\tabcolsep) * \real{0.1414}}
  >{\raggedleft\arraybackslash}p{(\columnwidth - 12\tabcolsep) * \real{0.0808}}
  >{\raggedright\arraybackslash}p{(\columnwidth - 12\tabcolsep) * \real{0.2222}}
  >{\raggedright\arraybackslash}p{(\columnwidth - 12\tabcolsep) * \real{0.1212}}
  >{\raggedleft\arraybackslash}p{(\columnwidth - 12\tabcolsep) * \real{0.1515}}
  >{\raggedleft\arraybackslash}p{(\columnwidth - 12\tabcolsep) * \real{0.0606}}@{}}

\caption{\label{tbl-case-study-flu-forecasts}An example prediction of
weekly incident influenza hospitalizations. This exmaple forecast was
made on May 15, 2023 for California at the 1 week ahead horizon. The
forecast was generated during theFluSight forecasting challenge, and
formatted according to hubverse standards post hoc. The
\texttt{location} and \texttt{season} columns have been omitted for
brevity.}

\tabularnewline

\toprule\noalign{}
\begin{minipage}[b]{\linewidth}\raggedright
model\_id
\end{minipage} & \begin{minipage}[b]{\linewidth}\raggedright
forecast\_date
\end{minipage} & \begin{minipage}[b]{\linewidth}\raggedleft
horizon
\end{minipage} & \begin{minipage}[b]{\linewidth}\raggedright
target
\end{minipage} & \begin{minipage}[b]{\linewidth}\raggedright
output\_type
\end{minipage} & \begin{minipage}[b]{\linewidth}\raggedleft
output\_type\_id
\end{minipage} & \begin{minipage}[b]{\linewidth}\raggedleft
value
\end{minipage} \\
\midrule\noalign{}
\endhead
\bottomrule\noalign{}
\endlastfoot
UMass-trends\_ensemble & 2023-05-15 & 1 & wk ahead inc flu hosp &
quantile & 0.025 & 12 \\
UMass-trends\_ensemble & 2023-05-15 & 1 & wk ahead inc flu hosp &
quantile & 0.100 & 17 \\
UMass-trends\_ensemble & 2023-05-15 & 1 & wk ahead inc flu hosp &
quantile & 0.250 & 25 \\
UMass-trends\_ensemble & 2023-05-15 & 1 & wk ahead inc flu hosp &
quantile & 0.750 & 46 \\
UMass-trends\_ensemble & 2023-05-15 & 1 & wk ahead inc flu hosp &
quantile & 0.900 & 56 \\
UMass-trends\_ensemble & 2023-05-15 & 1 & wk ahead inc flu hosp &
quantile & 0.975 & 68 \\

\end{longtable}

Next, we combine the component model outputs using the following code to
generate predictions from each ensemble model. The resulting ensemble
forecasts will have the same task ID variables, model output
specifications, and general data set features (albeit for 4 total models
instead of 42).

\begin{Shaded}
\begin{Highlighting}[]
\NormalTok{flu\_forecasts\_hubverse }\OtherTok{\textless{}{-}}\NormalTok{ dplyr}\SpecialCharTok{::}\FunctionTok{filter}\NormalTok{(}
\NormalTok{  flu\_forecasts\_hubverse,}
\NormalTok{  model\_id }\SpecialCharTok{!=} \StringTok{"Flusight{-}baseline"}
\NormalTok{)}

\NormalTok{mean\_ensemble }\OtherTok{\textless{}{-}}\NormalTok{ hubEnsembles}\SpecialCharTok{::}\FunctionTok{simple\_ensemble}\NormalTok{(flu\_forecasts\_hubverse,}
  \AttributeTok{weights =} \ConstantTok{NULL}\NormalTok{,}
  \AttributeTok{agg\_fun =} \StringTok{"mean"}\NormalTok{,}
  \AttributeTok{model\_id =} \StringTok{"mean{-}ensemble"}
\NormalTok{)}

\NormalTok{median\_ensemble }\OtherTok{\textless{}{-}}\NormalTok{ hubEnsembles}\SpecialCharTok{::}\FunctionTok{simple\_ensemble}\NormalTok{(flu\_forecasts\_hubverse,}
  \AttributeTok{weights =} \ConstantTok{NULL}\NormalTok{,}
  \AttributeTok{agg\_fun =} \StringTok{"median"}\NormalTok{,}
  \AttributeTok{model\_id =} \StringTok{"median{-}ensemble"}
\NormalTok{)}

\NormalTok{lp\_normal }\OtherTok{\textless{}{-}}\NormalTok{ hubEnsembles}\SpecialCharTok{::}\FunctionTok{linear\_pool}\NormalTok{(flu\_forecasts\_hubverse,}
  \AttributeTok{weights =} \ConstantTok{NULL}\NormalTok{,}
  \AttributeTok{n\_samples =} \FloatTok{1e5}\NormalTok{, }\AttributeTok{model\_id =} \StringTok{"lp{-}normal"}\NormalTok{,}
  \AttributeTok{tail\_dist =} \StringTok{"norm"}
\NormalTok{)}

\NormalTok{lp\_lognormal }\OtherTok{\textless{}{-}}\NormalTok{ hubEnsembles}\SpecialCharTok{::}\FunctionTok{linear\_pool}\NormalTok{(flu\_forecasts\_hubverse,}
  \AttributeTok{weights =} \ConstantTok{NULL}\NormalTok{,}
  \AttributeTok{n\_samples =} \FloatTok{1e5}\NormalTok{,}
  \AttributeTok{model\_id =} \StringTok{"lp{-}lognormal"}\NormalTok{,}
  \AttributeTok{tail\_dist =} \StringTok{"lnorm"}
\NormalTok{)}
\end{Highlighting}
\end{Shaded}

After computing the various ensembles, we evaluate the performance using
various scoring metrics. The goal of a forecast is to most accurately
predict the future observation, and there are various metrics available
to do evaluate a forecast with respect to this goal. Here, we use
several common metrics in forecast evaluation, including mean absolute
error (MAE), weighted interval score (WIS) (Bracher et al. 2021), 50\%
prediction interval (PI) coverage, and 95\% PI coverage. Of these
metrics, MAE evaluates how well a point forecast matches the
corresponding observed value from the target data, and WIS and PI
coverage evaluate how well a probabilistic forecast captures that same
observed value, prioritizing probabilistic forecasts that are more
certain. We briefly summarize each metric below.

MAE measures the average absolute error of a set of forecasts against
the true value; smaller values of MAE indicate better forecast accuracy.
WIS is a generalization of MAE for probabilistic forecasts and is an
alternative to other common proper scoring rules which cannot be
evaluated directly for quantile forecasts (Bracher et al. 2021). WIS is
made up of three component penalties: (1) for over-prediction, and (2)
for under-prediction, which together measure the accuracy of the
forecast, and (3) for the spread of each interval (where an interval is
defined by a symmetric set of two quantiles). This metric weights these
penalties across all prediction intervals provided. A lower WIS value
indicates a more accurate forecast (Bracher et al. 2021). Coverage is
calculated for a particular prediction interval and provides information
about whether a forecast has accurately characterized its uncertainty
about future observations. The \((1-\alpha)*100\)\% PI coverage measures
the proportion of the time that a given prediction interval at that
nominal level includes the observed value. Achieving approximately
nominal (\((1-\alpha)*100\)\%) coverage indicates a well-calibrated
forecast.

We also use relative versions of WIS and MAE (rWIS and rMAE,
respectively) to understand how the ensemble performance compares to
that of the FluSight baseline model. These metrics are calculated as
\[\textrm{rWIS} = \frac{\textrm{WIS}_{\textrm{model }m}}{\textrm{WIS}_{\textrm{baseline}}} \hspace{3cm} \textrm{rMAE} = \frac{\textrm{MAE}_{\textrm{model }m}}{\textrm{MAE}_{\textrm{baseline}}},\]
where model \(m\) is any given model being compared against the
baseline. For both of these metrics, a value less than one indicates
better performance compared to the baseline while a value greater than
one indicates worse performance. By definition, the FluSight baseline
itself will always have a value of one for both of these metrics.

Here, we score each unique prediction from an ensemble model, i.e.,
combination of task ID variables, against target data with the
\texttt{score\_forecasts()} function from the \texttt{covidHubUtils}
package. This function outputs each of the metrics described above. We
use median forecasts taken from the 0.5 quantile for the MAE evaluation.
We calculate these metrics for the four ensembles (as well as the
Flusight baseline) over all the forecasts using the
\texttt{evaluate\_flu\_scores()} function stored in the
\texttt{evaluation\_functions.R} script. We run the ensemble and scoring
code only one time and save the results in data objects split by a model
and season, which then can be loaded separately for analysis plotting
and scoring.

\subsection{Performance results across
ensembles}\label{performance-results-across-ensembles}

The quantile median ensemble had the best overall performance in terms
of WIS and MAE (and the relative versions of these metrics) with
above-nominal coverage rates (Table~\ref{tbl-overall-evaluation}). The
two linear opinion pools had very similar performance to each other.
These methods had the second-best performance as measured by WIS and
MAE, but they had the highest 50\% and 95\% coverage rates, with
empirical coverage that was well above the nominal coverage rate. The
quantile mean performed the worst of the ensembles with the highest MAE,
which was substantially different from that of the other ensembles.

\begin{longtable}[]{@{}
  >{\raggedright\arraybackslash}p{(\columnwidth - 12\tabcolsep) * \real{0.2469}}
  >{\raggedright\arraybackslash}p{(\columnwidth - 12\tabcolsep) * \real{0.1358}}
  >{\raggedright\arraybackslash}p{(\columnwidth - 12\tabcolsep) * \real{0.1235}}
  >{\raggedright\arraybackslash}p{(\columnwidth - 12\tabcolsep) * \real{0.1235}}
  >{\raggedright\arraybackslash}p{(\columnwidth - 12\tabcolsep) * \real{0.1235}}
  >{\raggedright\arraybackslash}p{(\columnwidth - 12\tabcolsep) * \real{0.1235}}
  >{\raggedright\arraybackslash}p{(\columnwidth - 12\tabcolsep) * \real{0.1235}}@{}}

\caption{\label{tbl-overall-evaluation}Summary of overall model
performance across both seasons, averaged over all locations except the
US national location. The best values for each metric is bolded, though
the metric values are often quite similar among the models.}

\tabularnewline

\toprule\noalign{}
\begin{minipage}[b]{\linewidth}\raggedright
model
\end{minipage} & \begin{minipage}[b]{\linewidth}\raggedright
wis
\end{minipage} & \begin{minipage}[b]{\linewidth}\raggedright
rwis
\end{minipage} & \begin{minipage}[b]{\linewidth}\raggedright
mae
\end{minipage} & \begin{minipage}[b]{\linewidth}\raggedright
rmae
\end{minipage} & \begin{minipage}[b]{\linewidth}\raggedright
cov50
\end{minipage} & \begin{minipage}[b]{\linewidth}\raggedright
cov95
\end{minipage} \\
\midrule\noalign{}
\endhead
\bottomrule\noalign{}
\endlastfoot
\textbf{median-ensemble} & \textbf{18.158} & \textbf{0.794} &
\textbf{27.36} & \textbf{0.933} & \emph{0.597} & \textbf{0.922} \\
lp-normal & 19.745 & 0.863 & 27.932 & 0.953 & 0.709 & 0.99 \\
lp-lognormal & 19.747 & 0.863 & 27.933 & 0.953 & 0.708 & 0.99 \\
mean-ensemble & 20.18 & 0.882 & 29.582 & 1.009 & \textbf{0.595} &
0.889 \\
Flusight-baseline & 22.876 & 1 & 29.315 & 1 & 0.604 & 0.881 \\

\end{longtable}

Plots of the models' forecasts can aid our understanding about the
origin of these accuracy differences. For example, the linear opinion
pools consistently had some of the widest prediction intervals, and
consequently the highest coverage rates. The median ensemble, which had
the best WIS, balanced interval width with calibration best overall,
with narrower intervals than the linear pools that still achieved
near-nominal coverage on average across all time points. The quantile
mean's interval widths could vary, though it usually had narrower
intervals than the linear pools. However, this model's point forecasts
demonstrated a larger error margin compared to the other ensembles,
especially at longer horizons. This pattern is demonstrated in
Figure~\ref{fig-plot-forecasts-hubVis} for the 4-week ahead forecast in
California following the 2022-23 season peak on December 5, 2022. Here
the quantile mean predicted a continued increase in hospitalizations, at
a steeper slope than the other ensemble methods.

\begin{Shaded}
\begin{Highlighting}[]
\NormalTok{model\_names }\OtherTok{\textless{}{-}} \FunctionTok{c}\NormalTok{(}
  \StringTok{"Flusight{-}baseline"}\NormalTok{, }\StringTok{"lp{-}lognormal"}\NormalTok{, }\StringTok{"lp{-}normal"}\NormalTok{,}
  \StringTok{"mean{-}ensemble"}\NormalTok{, }\StringTok{"median{-}ensemble"}
\NormalTok{)}

\NormalTok{flu\_dates\_21\_22 }\OtherTok{\textless{}{-}} \FunctionTok{as.Date}\NormalTok{(}\StringTok{"2022{-}01{-}24"}\NormalTok{) }\SpecialCharTok{+} \FunctionTok{weeks}\NormalTok{(}\DecValTok{0}\SpecialCharTok{:}\DecValTok{21}\NormalTok{)}
\NormalTok{flu\_dates\_22\_23 }\OtherTok{\textless{}{-}} \FunctionTok{as.Date}\NormalTok{(}\StringTok{"2022{-}10{-}17"}\NormalTok{) }\SpecialCharTok{+} \FunctionTok{weeks}\NormalTok{(}\DecValTok{0}\SpecialCharTok{:}\DecValTok{30}\NormalTok{)}
\NormalTok{flu\_dates\_off\_season }\OtherTok{\textless{}{-}} \FunctionTok{as.Date}\NormalTok{(}\StringTok{"2022{-}06{-}27"}\NormalTok{) }\SpecialCharTok{+} \FunctionTok{weeks}\NormalTok{(}\DecValTok{0}\SpecialCharTok{:}\DecValTok{15}\NormalTok{)}
\NormalTok{all\_flu\_dates }\OtherTok{\textless{}{-}} \FunctionTok{c}\NormalTok{(flu\_dates\_21\_22, flu\_dates\_22\_23)}

\NormalTok{select\_dates }\OtherTok{\textless{}{-}} \FunctionTok{c}\NormalTok{(all\_flu\_dates[}\FunctionTok{seq}\NormalTok{(}\DecValTok{1}\NormalTok{, }\DecValTok{69}\NormalTok{, }\DecValTok{4}\NormalTok{)], flu\_dates\_21\_22[}\DecValTok{22}\NormalTok{] }\SpecialCharTok{+}
\NormalTok{  lubridate}\SpecialCharTok{::}\FunctionTok{weeks}\NormalTok{(}\DecValTok{1}\SpecialCharTok{:}\DecValTok{16}\NormalTok{))}
\NormalTok{forecasts\_ca }\OtherTok{\textless{}{-}}\NormalTok{ flu\_forecasts\_ensembles }\SpecialCharTok{|\textgreater{}}
  \FunctionTok{rbind}\NormalTok{(}\FunctionTok{expand.grid}\NormalTok{(}
    \AttributeTok{model\_id =}\NormalTok{ model\_names[}\DecValTok{1}\SpecialCharTok{:}\DecValTok{5}\NormalTok{],}
    \AttributeTok{forecast\_date =}\NormalTok{ flu\_dates\_21\_22[}\DecValTok{22}\NormalTok{] }\SpecialCharTok{+}
\NormalTok{      lubridate}\SpecialCharTok{::}\FunctionTok{weeks}\NormalTok{(}\DecValTok{1}\SpecialCharTok{:}\DecValTok{16}\NormalTok{),}
    \AttributeTok{location =} \FunctionTok{unique}\NormalTok{(flu\_truth\_all}\SpecialCharTok{$}\NormalTok{location),}
    \AttributeTok{horizon =} \DecValTok{1}\SpecialCharTok{:}\DecValTok{4}\NormalTok{,}
    \AttributeTok{target =} \StringTok{"wk ahead inc flu hosp"}\NormalTok{,}
    \AttributeTok{output\_type =} \StringTok{"quantile"}\NormalTok{,}
    \AttributeTok{output\_type\_id =} \FunctionTok{c}\NormalTok{(}
      \FloatTok{0.01}\NormalTok{, }\FloatTok{0.025}\NormalTok{, }\FunctionTok{seq}\NormalTok{(}\FloatTok{0.05}\NormalTok{, }\FloatTok{0.95}\NormalTok{, }\FloatTok{0.5}\NormalTok{),}
      \FloatTok{0.975}\NormalTok{, }\FloatTok{0.99}
\NormalTok{    ),}
    \AttributeTok{value =} \ConstantTok{NA}
\NormalTok{  ) }\SpecialCharTok{|\textgreater{}}
\NormalTok{    dplyr}\SpecialCharTok{::}\FunctionTok{mutate}\NormalTok{(}\AttributeTok{target\_end\_date =}\NormalTok{ forecast\_date }\SpecialCharTok{+}
\NormalTok{      lubridate}\SpecialCharTok{::}\FunctionTok{weeks}\NormalTok{(horizon), }\AttributeTok{.before =}\NormalTok{ target)) }\SpecialCharTok{|\textgreater{}}
\NormalTok{  dplyr}\SpecialCharTok{::}\FunctionTok{filter}\NormalTok{(location }\SpecialCharTok{==} \StringTok{"06"}\NormalTok{, forecast\_date }\SpecialCharTok{\%in\%}\NormalTok{ select\_dates) }\SpecialCharTok{|\textgreater{}}
\NormalTok{  dplyr}\SpecialCharTok{::}\FunctionTok{group\_by}\NormalTok{(forecast\_date) }\SpecialCharTok{|\textgreater{}}
  \FunctionTok{as\_model\_out\_tbl}\NormalTok{()}

\NormalTok{truth\_ca }\OtherTok{\textless{}{-}}\NormalTok{ flu\_truth\_all }\SpecialCharTok{|\textgreater{}}
\NormalTok{  dplyr}\SpecialCharTok{::}\FunctionTok{filter}\NormalTok{(location }\SpecialCharTok{==} \StringTok{"06"}\NormalTok{) }\SpecialCharTok{|\textgreater{}}
  \FunctionTok{rbind}\NormalTok{(}\FunctionTok{expand.grid}\NormalTok{(}
    \AttributeTok{model =} \StringTok{"flu{-}truth"}\NormalTok{,}
    \AttributeTok{target\_variable =} \StringTok{"inc flu hosp"}\NormalTok{,}
    \AttributeTok{target\_end\_date =}\NormalTok{ flu\_dates\_21\_22[}\DecValTok{22}\NormalTok{] }\SpecialCharTok{+}
\NormalTok{      lubridate}\SpecialCharTok{::}\FunctionTok{weeks}\NormalTok{(}\DecValTok{1}\SpecialCharTok{:}\DecValTok{16}\NormalTok{),}
    \AttributeTok{location =} \FunctionTok{unique}\NormalTok{(flu\_truth\_all}\SpecialCharTok{$}\NormalTok{location),}
    \AttributeTok{value =} \ConstantTok{NA}
\NormalTok{  ))}

\NormalTok{ca\_plot\_ensembles }\OtherTok{\textless{}{-}}
  \FunctionTok{plot\_step\_ahead\_model\_output}\NormalTok{(}
\NormalTok{    forecasts\_ca }\SpecialCharTok{|\textgreater{}} \FunctionTok{filter}\NormalTok{(model\_id }\SpecialCharTok{!=} \StringTok{"Flusight{-}baseline"}\NormalTok{),}
\NormalTok{    truth\_ca,}
    \AttributeTok{use\_median\_as\_point =} \ConstantTok{TRUE}\NormalTok{,}
    \AttributeTok{show\_plot =} \ConstantTok{FALSE}\NormalTok{,}
    \AttributeTok{x\_col\_name =} \StringTok{"target\_end\_date"}\NormalTok{,}
    \AttributeTok{x\_truth\_col\_name =} \StringTok{"target\_end\_date"}\NormalTok{,}
    \AttributeTok{show\_legend =} \ConstantTok{FALSE}\NormalTok{,}
    \AttributeTok{facet =} \StringTok{"model\_id"}\NormalTok{,}
    \AttributeTok{facet\_scales =} \StringTok{"free\_y"}\NormalTok{,}
    \AttributeTok{facet\_nrow =} \DecValTok{3}\NormalTok{,}
    \AttributeTok{interactive =} \ConstantTok{FALSE}\NormalTok{,}
    \AttributeTok{pal\_color =} \StringTok{"Set2"}\NormalTok{,}
    \AttributeTok{fill\_transparency =} \FloatTok{0.45}\NormalTok{,}
    \AttributeTok{intervals =} \FunctionTok{c}\NormalTok{(}\FloatTok{0.5}\NormalTok{, }\FloatTok{0.95}\NormalTok{),}
    \AttributeTok{title =} \ConstantTok{NULL}\NormalTok{,}
\NormalTok{  )}

\NormalTok{ca\_plot\_ensembles }\OtherTok{\textless{}{-}}\NormalTok{ ca\_plot\_ensembles }\SpecialCharTok{+}
  \FunctionTok{scale\_x\_date}\NormalTok{(}
    \AttributeTok{name =} \ConstantTok{NULL}\NormalTok{, }\AttributeTok{limits =} \FunctionTok{c}\NormalTok{(}
      \FunctionTok{as.Date}\NormalTok{(}\StringTok{"2022{-}01{-}01"}\NormalTok{),}
      \FunctionTok{as.Date}\NormalTok{(}\StringTok{"2023{-}06{-}08"}\NormalTok{)}
\NormalTok{    ),}
    \AttributeTok{date\_breaks =} \StringTok{"4 months"}\NormalTok{, }\AttributeTok{date\_labels =} \StringTok{"\%b \textquotesingle{}\%y"}
\NormalTok{  ) }\SpecialCharTok{+}
  \FunctionTok{scale\_color\_manual}\NormalTok{(}
    \AttributeTok{breaks =}\NormalTok{ model\_names[}\DecValTok{2}\SpecialCharTok{:}\DecValTok{5}\NormalTok{],}
    \AttributeTok{values =}\NormalTok{ RColorBrewer}\SpecialCharTok{::}\FunctionTok{brewer.pal}\NormalTok{(}\DecValTok{5}\NormalTok{, }\StringTok{"Set2"}\NormalTok{)[}\DecValTok{2}\SpecialCharTok{:}\DecValTok{5}\NormalTok{]}
\NormalTok{  ) }\SpecialCharTok{+}
  \FunctionTok{scale\_fill\_manual}\NormalTok{(}
    \AttributeTok{breaks =}\NormalTok{ model\_names[}\DecValTok{2}\SpecialCharTok{:}\DecValTok{5}\NormalTok{],}
    \AttributeTok{values =}\NormalTok{ RColorBrewer}\SpecialCharTok{::}\FunctionTok{brewer.pal}\NormalTok{(}\DecValTok{5}\NormalTok{, }\StringTok{"Set2"}\NormalTok{)[}\DecValTok{2}\SpecialCharTok{:}\DecValTok{5}\NormalTok{]}
\NormalTok{  ) }\SpecialCharTok{+}
  \FunctionTok{theme}\NormalTok{(}
    \AttributeTok{axis.ticks.length.x =} \FunctionTok{unit}\NormalTok{(}\FloatTok{0.5}\NormalTok{, }\StringTok{"cm"}\NormalTok{),}
    \AttributeTok{axis.text.x =} \FunctionTok{element\_text}\NormalTok{(}\AttributeTok{vjust =} \DecValTok{7}\NormalTok{, }\AttributeTok{hjust =} \SpecialCharTok{{-}}\FloatTok{0.2}\NormalTok{),}
    \AttributeTok{axis.title.y =} \FunctionTok{element\_blank}\NormalTok{(),}
    \AttributeTok{legend.position =} \StringTok{"none"}
\NormalTok{  )}

\NormalTok{ca\_plot\_baseline }\OtherTok{\textless{}{-}}
  \FunctionTok{plot\_step\_ahead\_model\_output}\NormalTok{(}
\NormalTok{    forecasts\_ca }\SpecialCharTok{|\textgreater{}} \FunctionTok{filter}\NormalTok{(model\_id }\SpecialCharTok{==} \StringTok{"Flusight{-}baseline"}\NormalTok{),}
\NormalTok{    truth\_ca,}
    \AttributeTok{use\_median\_as\_point =} \ConstantTok{TRUE}\NormalTok{,}
    \AttributeTok{show\_plot =} \ConstantTok{TRUE}\NormalTok{,}
    \AttributeTok{x\_col\_name =} \StringTok{"target\_end\_date"}\NormalTok{,}
    \AttributeTok{x\_truth\_col\_name =} \StringTok{"target\_end\_date"}\NormalTok{,}
    \AttributeTok{show\_legend =} \ConstantTok{FALSE}\NormalTok{,}
    \AttributeTok{facet =} \StringTok{"model\_id"}\NormalTok{,}
    \AttributeTok{facet\_scales =} \StringTok{"free\_y"}\NormalTok{,}
    \AttributeTok{facet\_nrow =} \DecValTok{1}\NormalTok{,}
    \AttributeTok{interactive =} \ConstantTok{FALSE}\NormalTok{,}
    \AttributeTok{fill\_transparency =} \FloatTok{0.45}\NormalTok{,}
    \AttributeTok{intervals =} \FunctionTok{c}\NormalTok{(}\FloatTok{0.5}\NormalTok{, }\FloatTok{0.95}\NormalTok{),}
    \AttributeTok{title =} \ConstantTok{NULL}
\NormalTok{  ) }\SpecialCharTok{+}
  \FunctionTok{scale\_x\_date}\NormalTok{(}
    \AttributeTok{name =} \ConstantTok{NULL}\NormalTok{, }\AttributeTok{limits =} \FunctionTok{c}\NormalTok{(}
      \FunctionTok{as.Date}\NormalTok{(}\StringTok{"2022{-}01{-}01"}\NormalTok{),}
      \FunctionTok{as.Date}\NormalTok{(}\StringTok{"2023{-}06{-}08"}\NormalTok{)}
\NormalTok{    ),}
    \AttributeTok{date\_breaks =} \StringTok{"4 months"}\NormalTok{, }\AttributeTok{date\_labels =} \StringTok{"\%b \textquotesingle{}\%y"}
\NormalTok{  ) }\SpecialCharTok{+}
  \FunctionTok{theme\_bw}\NormalTok{() }\SpecialCharTok{+}
  \FunctionTok{theme}\NormalTok{(}
    \AttributeTok{axis.title.x =} \FunctionTok{element\_blank}\NormalTok{(),}
    \AttributeTok{axis.ticks.length.x =} \FunctionTok{unit}\NormalTok{(}\DecValTok{0}\NormalTok{, }\StringTok{"cm"}\NormalTok{),}
    \AttributeTok{axis.text.x =} \FunctionTok{element\_blank}\NormalTok{(), }\CommentTok{\# element\_text(vjust = 5, hjust = {-}0.2),}
    \AttributeTok{axis.title.y =} \FunctionTok{element\_blank}\NormalTok{(),}
    \AttributeTok{legend.position =} \StringTok{"none"}
\NormalTok{  )}

\NormalTok{ca\_plot\_all }\OtherTok{\textless{}{-}}
  \FunctionTok{plot\_step\_ahead\_model\_output}\NormalTok{(}
\NormalTok{    forecasts\_ca }\SpecialCharTok{|\textgreater{}} \FunctionTok{mutate}\NormalTok{(}\AttributeTok{facet\_name =} \StringTok{"All models, same scale"}\NormalTok{),}
\NormalTok{    truth\_ca,}
    \AttributeTok{use\_median\_as\_point =} \ConstantTok{TRUE}\NormalTok{,}
    \AttributeTok{show\_plot =} \ConstantTok{TRUE}\NormalTok{,}
    \AttributeTok{x\_col\_name =} \StringTok{"target\_end\_date"}\NormalTok{,}
    \AttributeTok{x\_truth\_col\_name =} \StringTok{"target\_end\_date"}\NormalTok{,}
    \AttributeTok{show\_legend =} \ConstantTok{FALSE}\NormalTok{,}
    \AttributeTok{facet =} \StringTok{"facet\_name"}\NormalTok{,}
    \AttributeTok{interactive =} \ConstantTok{FALSE}\NormalTok{,}
    \AttributeTok{fill\_transparency =} \FloatTok{0.45}\NormalTok{,}
    \AttributeTok{intervals =} \FunctionTok{c}\NormalTok{(}\FloatTok{0.5}\NormalTok{, }\FloatTok{0.95}\NormalTok{),}
    \AttributeTok{title =} \ConstantTok{NULL}
\NormalTok{  ) }\SpecialCharTok{+}
  \FunctionTok{scale\_x\_date}\NormalTok{(}
    \AttributeTok{name =} \ConstantTok{NULL}\NormalTok{, }\AttributeTok{limits =} \FunctionTok{c}\NormalTok{(}
      \FunctionTok{as.Date}\NormalTok{(}\StringTok{"2022{-}01{-}01"}\NormalTok{),}
      \FunctionTok{as.Date}\NormalTok{(}\StringTok{"2023{-}06{-}08"}\NormalTok{)}
\NormalTok{    ),}
    \AttributeTok{date\_breaks =} \StringTok{"4 months"}\NormalTok{, }\AttributeTok{date\_labels =} \StringTok{"\%b \textquotesingle{}\%y"}
\NormalTok{  ) }\SpecialCharTok{+}
  \FunctionTok{theme\_bw}\NormalTok{() }\SpecialCharTok{+}
  \FunctionTok{theme}\NormalTok{(}
    \AttributeTok{axis.title.x =} \FunctionTok{element\_blank}\NormalTok{(),}
    \AttributeTok{axis.text.x =} \FunctionTok{element\_blank}\NormalTok{(),}
    \AttributeTok{axis.ticks.length.x =} \FunctionTok{unit}\NormalTok{(}\DecValTok{0}\NormalTok{, }\StringTok{"cm"}\NormalTok{),}
    \AttributeTok{axis.title.y =} \FunctionTok{element\_blank}\NormalTok{(),}
    \AttributeTok{legend.position =} \StringTok{"none"}
\NormalTok{  )}

\NormalTok{((ca\_plot\_all }\SpecialCharTok{|}\NormalTok{ ca\_plot\_baseline) }\SpecialCharTok{/}\NormalTok{ ca\_plot\_ensembles) }\SpecialCharTok{+}
  \FunctionTok{theme\_bw}\NormalTok{() }\SpecialCharTok{+}
  \FunctionTok{plot\_layout}\NormalTok{(}\AttributeTok{guides =} \StringTok{"collect"}\NormalTok{, }\AttributeTok{heights =} \FunctionTok{c}\NormalTok{(}\DecValTok{1}\NormalTok{, }\DecValTok{2}\NormalTok{)) }\SpecialCharTok{+}
  \FunctionTok{plot\_annotation}\NormalTok{(}\AttributeTok{title =} \FunctionTok{paste0}\NormalTok{(}
    \StringTok{"Weekly Incident "}\NormalTok{,}
    \StringTok{"Hospitalizations for Influenza "}\NormalTok{,}
    \StringTok{"in California"}
\NormalTok{  ))}
\end{Highlighting}
\end{Shaded}

\begin{figure}[H]

\centering{

\includegraphics{hubEnsembles_manuscript_files/figure-pdf/fig-plot-forecasts-hubVis-1.pdf}

}

\caption{\label{fig-plot-forecasts-hubVis}TO BE FIXED: One to four week
ahead forecasts for select dates plotted against target data for
California. The first panel shows all models on the same scale. All
other panels show forecasts for each individual model, with varying
y-axis scales show prediction accuracy as compared to observed influenza
hospitalizations.}

\end{figure}%

We can use additional functions from the
\texttt{evaluation\_functions.R} script to examine model performance
with greater granularity. For example, we may use the following lines of
code to generate a table that shows scores from week to week.

\begin{Shaded}
\begin{Highlighting}[]
\NormalTok{flu\_date\_horizon\_season\_states }\OtherTok{\textless{}{-}}\NormalTok{ flu\_scores\_all }\SpecialCharTok{|\textgreater{}}
  \FunctionTok{evaluate\_flu\_scores}\NormalTok{(}
    \AttributeTok{grouping\_variables =} \FunctionTok{c}\NormalTok{(}
      \StringTok{"horizon"}\NormalTok{, }\StringTok{"forecast\_date"}\NormalTok{,}
      \StringTok{"season"}
\NormalTok{    ),}
    \AttributeTok{baseline\_name =} \StringTok{"Flusight{-}baseline"}\NormalTok{, }\AttributeTok{us\_only =} \ConstantTok{FALSE}
\NormalTok{  )}
\end{Highlighting}
\end{Shaded}

However, reading and interpreting such a table becomes unwieldy with 53
weeks of forecasts. Instead, the above summarized scores can be
separated by metric, filtered by horizon, and plotted against truth
data, An example to generate such a plot using the
\texttt{plot\_evaluated\_scores\_forecast\_date()} is shown in the code
chunk below. (In practice, we may also split the predictions by season
for improve readability, but this step is not strictly necessary.)

\begin{Shaded}
\begin{Highlighting}[]
\NormalTok{flu\_date\_horizon\_season\_states }\SpecialCharTok{|\textgreater{}}
\NormalTok{  dplyr}\SpecialCharTok{::}\FunctionTok{filter}\NormalTok{(season }\SpecialCharTok{==} \StringTok{"2021{-}2022"}\NormalTok{) }\SpecialCharTok{|\textgreater{}}
  \FunctionTok{plot\_evaluated\_scores\_forecast\_date}\NormalTok{(model\_names, model\_colors,}
    \AttributeTok{h =} \DecValTok{1}\NormalTok{,}
    \AttributeTok{y\_var =} \StringTok{"mae"}\NormalTok{,}
    \AttributeTok{main =} \StringTok{"MAE 2021{-}2022, 1 week ahead"}\NormalTok{,}
    \AttributeTok{truth\_data =}\NormalTok{ flu\_truth\_all,}
    \AttributeTok{truth\_scaling =} \FloatTok{0.1}
\NormalTok{  )}
\end{Highlighting}
\end{Shaded}

We then may combine several of these plots to obtain a complete picture
by plotting every metric and the 1 and 4 week ahead horizons. From these
combined plots we can see that the ensemble models tend to have similar
MAE values during the entire time period, with slight divergence in MAE
values for certain weeks at the four week ahead horizon
(Figure~\ref{fig-mae-vs-forecast-date}). However, the models show
greater differences for the other two metrics, WIS and coverage,
particularly during times of rapid change in the observed incident
hospitalizations (Figure~\ref{fig-wis-vs-forecast-date} and
Figure~\ref{fig-cov95-vs-forecast-date}). The linear pools have a lower
WIS than the median ensemble at the one week ahead forecast horizon for
over a third of forecast dates (11 weeks) during the 2022-2023 season:
from October 17, 2022 to December 12, 2022; January 2, 2023; and January
9, 2023 (Figure~\ref{fig-wis-vs-forecast-date}). These dates span the
rapid rise and fall of incident flu hospitalizations surrounding the
season's peak, with the largest differences in WIS occurring on November
28, December 5, December 12, January 2, and January 9. Additionally, the
PI coverage rates for the linear pools were at least as large as the
coverage rates of the other models throughout the entire period of
analysis at both the 1 and 4 week ahead forecast horizons (see
Figure~\ref{fig-cov95-vs-forecast-date}).

\begin{figure}

\centering{

\includegraphics{hubEnsembles_manuscript_files/figure-pdf/fig-mae-vs-forecast-date-1.pdf}

}

\caption{\label{fig-mae-vs-forecast-date}Mean absolute error (MAE)
averaged across all locations. Average MAE is shown for each season
(columns) and for 1-week and 4-week ahead forecasts (rows). Results are
plotted for each ensemble model (colored points) across the entire
season. Average target data across all locations is plotted in black.}

\end{figure}%

\begin{figure}

\centering{

\includegraphics{hubEnsembles_manuscript_files/figure-pdf/fig-wis-vs-forecast-date-1.pdf}

}

\caption{\label{fig-wis-vs-forecast-date}Weighted interval score (WIS)
averaged across all locations. Average WIS is shown for each season
(columns) and for 1-week and 4-week ahead forecasts (rows). Results are
plotted for each ensemble model (colored points) across the entire
season. Average target data across all locations is plotted in black.}

\end{figure}%

\begin{figure}

\centering{

\includegraphics{hubEnsembles_manuscript_files/figure-pdf/fig-cov95-vs-forecast-date-1.pdf}

}

\caption{\label{fig-cov95-vs-forecast-date}95\% prediction interval (PI)
coverage averaged across all locations. Average coverage is shown for
each season (columns) and for 1-week and 4-week ahead forecasts (rows).
Results are plotted for each ensemble model (colored points) across the
entire season. Average target data across all locations is plotted in
black.}

\end{figure}%

In this analysis all of the ensemble variations outperformed the
baseline model; yet, different ensembling methods performed best under
different circumstances. While the quantile median had the best overall
results for WIS, MAE, 50\% PI coverage, and 95\% PI coverage, other
models may perform better from week-to-week for each metric. Around the
2022-2023 season's peak in early December, the remaining four models
(including the baseline) each had instances in which they achieved the
lowest WIS, like the linear pool ensembles for the one week ahead
horizon over several weeks of this period.

The choice of an appropriate ensemble aggregation method may depend on
the forecast target, the goal of forecasting, and the behavior of the
individual models contributing to an ensemble. One case may call for
prioritizing above-nominal coverage rates while another may prioritize
accurate point forecasts. The \texttt{simple\_ensemble} and
\texttt{linear\_pool} functions and the ability to specify component
model weights and an aggregation function for \texttt{simple\_ensemble}
allow users to implement a variety of ensemble methods.

\section{Conclusion}\label{sec-conclusions}

Ensembles of independent models are a powerful tool to generate more
accurate and more reliable forecasts of future outcomes than a single
model alone. Here, we have demonstrated how to utilize
\texttt{hubEnsembles}, a simple and flexible framework to combine
individual model forecasts and create ensemble predictions. When using
\texttt{hubEnsembles}, it is important to carefully choose an ensemble
method that is well suited for the situation. Although there may not be
a universal ``best'' method, matching the properties of a given ensemble
method with the features of the component models will likely yield best
results. For example, we showed for forecasts of seasonal influenza in
the US, the quantile median ensemble performed best overall, but the
linear pool method had advantages during periods of rapid change, when
outlying component forecasts were likely more important. Notably, all
ensemble methods outperformed the baseline model. These performance
improvements from ensemble models motivate the use of a ``hub-based''
approach to prediction for infectious diseases and in other fields.
Fitting within the larger suite of ``hubverse'' tools that support such
efforts, the \texttt{hubEnsembles} package provides important software
infrastructure for leveraging the power of multi-model ensembles.

\section{Acknowledgements}\label{acknowledgements}

The authors thank all members of the hubverse community; the broader
hubverse software infrastructure made this package possible. E. Howerton
was supported by the Eberly College of Science Barbara McClintock
Science Achievement Graduate Scholarship in Biology at the Pennsylvania
State University. PLEASE ADD FUNDING HERE.

\section*{References}\label{references}
\addcontentsline{toc}{section}{References}

\phantomsection\label{refs}
\begin{CSLReferences}{1}{0}
\bibitem[\citeproctext]{ref-aastveit2018}
Aastveit, Knut Are, James Mitchell, Francesco Ravazzolo, and Herman K.
van Dijk. 2018. {``The Evolution of Forecast Density Combinations in
Economics.''} Amsterdam; Rotterdam.
\url{https://www.econstor.eu/handle/10419/185588}.

\bibitem[\citeproctext]{ref-alley2019}
Alley, Richard B., Kerry A. Emanuel, and Fuqing Zhang. 2019. {``Advances
in Weather Prediction.''} \emph{Science} 363 (6425): 342--44.
\url{https://doi.org/10.1126/science.aav7274}.

\bibitem[\citeproctext]{ref-borchering_public_2023}
Borchering, Rebecca K., Jessica M. Healy, Betsy L. Cadwell, Michael A.
Johansson, Rachel B. Slayton, Megan Wallace, and Matthew Biggerstaff.
2023. {``Public Health Impact of the {U}.{S}. {Scenario} {Modeling}
{Hub}.''} \emph{Epidemics} 44 (September): 100705.
\url{https://doi.org/10.1016/j.epidem.2023.100705}.

\bibitem[\citeproctext]{ref-bosse_stackr_2023}
Bosse, Nikos, Yuling Yao, Sam Abbott, and Sebastian Funk. 2023.
\emph{Stackr: {Create} {Mixture} {Models} {From} {Predictive}
{Samples}}. \url{http://epiforecasts.io/stackr/}.

\bibitem[\citeproctext]{ref-bracher_evaluating_2021}
Bracher, Johannes, Evan L. Ray, Tilmann Gneiting, and Nicholas G. Reich.
2021. {``Evaluating Epidemic Forecasts in an Interval Format.''}
\emph{PLOS Computational Biology} 17 (2): e1008618.
\url{https://doi.org/10.1371/journal.pcbi.1008618}.

\bibitem[\citeproctext]{ref-cdc_flusight}
CDC. 2023. {``About Flu Forecasting.''}
\url{https://www.cdc.gov/flu/weekly/flusight/how-flu-forecasting.htm}.

\bibitem[\citeproctext]{ref-clemen1989}
Clemen, Robert T. 1989. {``Combining Forecasts: A Review and Annotated
Bibliography.''} \emph{International Journal of Forecasting} 5 (4):
559--83. \url{https://doi.org/10.1016/0169-2070(89)90012-5}.

\bibitem[\citeproctext]{ref-colon-gonzalez_probabilistic_2021}
Colón-González, Felipe J., Leonardo Soares Bastos, Barbara Hofmann,
Alison Hopkin, Quillon Harpham, Tom Crocker, Rosanna Amato, et al. 2021.
{``Probabilistic Seasonal Dengue Forecasting in {Vietnam}: {A} Modelling
Study Using Superensembles.''} \emph{PLOS Medicine} 18 (3): e1003542.
\url{https://doi.org/10.1371/journal.pmed.1003542}.

\bibitem[\citeproctext]{ref-couch_stacks_2023}
Couch, Simon, and Max Kuhn. 2023. \emph{Stacks: Tidy Model Stacking}.
\url{https://stacks.tidymodels.org/}.

\bibitem[\citeproctext]{ref-cramer2022}
Cramer, Estee Y., Evan L. Ray, Velma K. Lopez, Johannes Bracher, Andrea
Brennen, Alvaro J. Castro Rivadeneira, Aaron Gerding, et al. 2022.
{``Evaluation of Individual and Ensemble Probabilistic Forecasts of
COVID-19 Mortality in the United States.''} \emph{Proceedings of the
National Academy of Sciences} 119 (15): e2113561119.
\url{https://doi.org/10.1073/pnas.2113561119}.

\bibitem[\citeproctext]{ref-hibon2005}
Hibon, Michèle, and Theodoros Evgeniou. 2005. {``To Combine or Not to
Combine: Selecting Among Forecasts and Their Combinations.''}
\emph{International Journal of Forecasting} 21 (1): 15--24.
\url{https://doi.org/10.1016/j.ijforecast.2004.05.002}.

\bibitem[\citeproctext]{ref-howerton2023}
Howerton, Emily, Michael C. Runge, Tiffany L. Bogich, Rebecca K.
Borchering, Hidetoshi Inamine, Justin Lessler, Luke C. Mullany, et al.
2023. {``Context-Dependent Representation of Within- and Between-Model
Uncertainty: Aggregating Probabilistic Predictions in Infectious Disease
Epidemiology.''} \emph{Journal of The Royal Society Interface} 20 (198):
20220659. \url{https://doi.org/10.1098/rsif.2022.0659}.

\bibitem[\citeproctext]{ref-hubverse_docs}
hubverse. 2022. {``The Hubverse: Open Tools for Collaborative
Forecasting.''}
\url{https://hubdocs.readthedocs.io/en/latest/index.html}.

\bibitem[\citeproctext]{ref-johansson2019}
Johansson, Michael A., Karyn M. Apfeldorf, Scott Dobson, Jason Devita,
Anna L. Buczak, Benjamin Baugher, Linda J. Moniz, et al. 2019. {``An
open challenge to advance probabilistic forecasting for dengue
epidemics.''} \emph{Proceedings of the National Academy of Sciences} 116
(48): 24268--74. \url{https://doi.org/10.1073/pnas.1909865116}.

\bibitem[\citeproctext]{ref-lichtendahl2013}
Lichtendahl, Kenneth C., Yael Grushka-Cockayne, and Robert L. Winkler.
2013. {``Is It Better to Average Probabilities or Quantiles?''}
\emph{Management Science} 59 (7): 1594--1611.
\url{https://doi.org/10.1287/mnsc.1120.1667}.

\bibitem[\citeproctext]{ref-mcgowan2019}
McGowan, Craig J., Matthew Biggerstaff, Michael Johansson, Karyn M.
Apfeldorf, Michal Ben-Nun, Logan Brooks, Matteo Convertino, et al. 2019.
{``Collaborative Efforts to Forecast Seasonal Influenza in the United
States, 2015{\textendash}2016.''} \emph{Scientific Reports} 9 (1): 683.
\url{https://doi.org/10.1038/s41598-018-36361-9}.

\bibitem[\citeproctext]{ref-niederreiter1992quasirandom}
Niederreiter, Harald. 1992. \emph{Random Number Generation and
Quasi-Monte Carlo Methods}. Philadelphia PA: Society for Industrial;
Applied Mathematics.

\bibitem[\citeproctext]{ref-paireau_ensemble_2022}
Paireau, Juliette, Alessio Andronico, Nathanaël Hozé, Maylis Layan,
Pascal Crépey, Alix Roumagnac, Marc Lavielle, Pierre-Yves Boëlle, and
Simon Cauchemez. 2022. {``An Ensemble Model Based on Early Predictors to
Forecast {COVID}-19 Health Care Demand in {France}.''} \emph{Proceedings
of the National Academy of Sciences} 119 (18): e2103302119.
\url{https://doi.org/10.1073/pnas.2103302119}.

\bibitem[\citeproctext]{ref-pedregosa_scikit-learn_2011}
Pedregosa, Fabian, Gaël Varoquaux, Alexandre Gramfort, Vincent Michel,
Bertrand Thirion, Olivier Grisel, Mathieu Blondel, et al. 2011.
{``Scikit-Learn: {Machine} {Learning} in {Python}.''} \emph{Journal of
Machine Learning Research} 12 (85): 2825--30.
\url{http://jmlr.org/papers/v12/pedregosa11a.html}.

\bibitem[\citeproctext]{ref-ray_comparing_2023}
Ray, Evan L., Logan C. Brooks, Jacob Bien, Matthew Biggerstaff, Nikos I.
Bosse, Johannes Bracher, Estee Y. Cramer, et al. 2023. {``Comparing
Trained and Untrained Probabilistic Ensemble Forecasts of {COVID}-19
Cases and Deaths in the {United} {States}.''} \emph{International
Journal of Forecasting} 39 (3): 1366--83.
\url{https://doi.org/10.1016/j.ijforecast.2022.06.005}.

\bibitem[\citeproctext]{ref-distfromq}
Ray, Evan L., and Aaron Gerding. 2024. {``Distfromq: Reconstruct a
Distribution from a Collection of Quantiles.''}
\url{http://reichlab.io/distfromq/}.

\bibitem[\citeproctext]{ref-ray_prediction_2018}
Ray, Evan L., and Nicholas G. Reich. 2018. {``Prediction of Infectious
Disease Epidemics via Weighted Density Ensembles.''} \emph{PLoS
Computational Biology} 14 (2): e1005910.
\url{https://doi.org/10.1371/journal.pcbi.1005910}.

\bibitem[\citeproctext]{ref-reich2022}
Reich, Nicholas G., Justin Lessler, Sebastian Funk, Cecile Viboud,
Alessandro Vespignani, Ryan J. Tibshirani, Katriona Shea, et al. 2022.
{``Collaborative Hubs: Making the Most of Predictive Epidemic
Modeling.''} \emph{American Journal of Public Health} 112 (6): 839--42.
\url{https://doi.org/10.2105/AJPH.2022.306831}.

\bibitem[\citeproctext]{ref-reich_accuracy_2019}
Reich, Nicholas G., Craig J. McGowan, Teresa K. Yamana, Abhinav Tushar,
Evan L. Ray, Dave Osthus, Sasikiran Kandula, et al. 2019. {``Accuracy of
Real-Time Multi-Model Ensemble Forecasts for Seasonal Influenza in the
{U}.{S}.''} \emph{PLoS Computational Biology} 15 (11): e1007486.
\url{https://doi.org/10.1371/journal.pcbi.1007486}.

\bibitem[\citeproctext]{ref-stone1961}
Stone, M. 1961. {``The Opinion Pool.''} \emph{The Annals of Mathematical
Statistics} 32 (4): 1339--42. \url{http://www.jstor.org/stable/2237933}.

\bibitem[\citeproctext]{ref-tebaldi2007}
Tebaldi, Claudia, and Reto Knutti. 2007. {``The Use of the Multi-Model
Ensemble in Probabilistic Climate Projections.''} \emph{Philosophical
Transactions: Mathematical, Physical and Engineering Sciences} 365
(1857): 2053--75. \url{https://doi.org/10.1098/rsta.2007.2076}.

\bibitem[\citeproctext]{ref-timmermann2006b}
Timmermann, Allan. 2006. {``Chapter 4 Forecast Combinations.''} In,
edited by G. Elliott, C. W. J. Granger, and A. Timmermann, 1:135--96.
Elsevier. \url{https://doi.org/10.1016/S1574-0706(05)01004-9}.

\bibitem[\citeproctext]{ref-viboud2018}
Viboud, Cécile, Kaiyuan Sun, Robert Gaffey, Marco Ajelli, Laura
Fumanelli, Stefano Merler, Qian Zhang, Gerardo Chowell, Lone Simonsen,
and Alessandro Vespignani. 2018. {``The RAPIDD Ebola Forecasting
Challenge: Synthesis and Lessons Learnt.''} \emph{Epidemics}, The RAPIDD
Ebola Forecasting Challenge, 22 (March): 13--21.
\url{https://doi.org/10.1016/j.epidem.2017.08.002}.

\bibitem[\citeproctext]{ref-vincent1912}
Vincent, Stella Burnham. 1912. {``The Function of the Vibrissae in the
Behavior of the White Rat.''} PhD thesis, Cambridge MA.

\bibitem[\citeproctext]{ref-weiss2019}
Weiss, Christoph,E., Eran Raviv, and Gernot Roetzer. 2019. {``Forecast
Combinations in R Using the ForecastComb Package.''} \emph{The R
Journal} 10 (2): 262. \url{https://doi.org/10.32614/RJ-2018-052}.

\bibitem[\citeproctext]{ref-winkler2015}
Winkler, Robert L. 2015. {``Equal Versus Differential Weighting in
Combining Forecasts.''} \emph{Risk Analysis} 35 (1): 16--18.
\url{https://doi.org/10.1111/risa.12302}.

\bibitem[\citeproctext]{ref-yamana_superensemble_2016}
Yamana, Teresa K., Sasikiran Kandula, and Jeffrey Shaman. 2016.
{``Superensemble Forecasts of Dengue Outbreaks.''} \emph{Journal of The
Royal Society Interface} 13 (123): 20160410.
\url{https://doi.org/10.1098/rsif.2016.0410}.

\end{CSLReferences}



\end{document}
